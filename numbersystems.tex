\chapter{Number Systems}
  \begin{defn}
    A \emph{binary operation} on $A$ is a function $\cdot:A\times A\to A$. We
    usually write $\cdot(a,b)=c$ as $a\cdot b=c$.

    It is \emph{associative} if $(a\cdot b)\cdot c=a\cdot(b\cdot
    c)$ for any $a,\ b,\ c$ in $A$.

    It is \emph{commutative} if $a\cdot b=b\cdot a$ for any $a,\ b$ in $A$.
  \end{defn}
  \begin{defn}
    A \emph{monoid} is an ordered pair $(A,\cdot)$ of a set $A$ and an
    associative binary
    operation $\cdot$ on $A$ such that there exists an element $1$, called the
    \emph{identity}, such that $a\cdot 1=1\cdot a=a$ for all $a$.
  \end{defn}
  \begin{rem}
    There are two main notations for monoid-type structures. These are
    \begin{itemize}
      \item 
        Multiplicative notation, in which the operation is notated $a\cdot b$ or
        simply $ab$, and the identity element is $1$; and
      \item Additive notation, in which the operation is notated $a+b$ and the
        identity element is $0$.
    \end{itemize}
  \end{rem}
  \begin{defn}
    A \emph{group} is a monoid $(A,\cdot)$ such that for each element $a$ of $A$
    there is an element $b$ of $A$ such that
    $ab=1=ba$.
    
    A group is \emph{abelian} if the operation is commutative.
  \end{defn}
  \begin{prop}
    If $ab=ba=1$ and $ac=1$ or $ca=1$ then $b=c$.
  \end{prop}
  \begin{defn}
    The element $b$ of $A$ such that $ab=ba=1$ is called the \emph{inverse} of $a$.
    In multiplicative notation, the inverse of $a$ is notated $a^{-1}$.
    In additive notation, the inverse of $a$ is notated $-a$.
  \end{defn}
  \begin{rem}
    We often define $\frac ab=ab^{-1}$ in multiplicative notation, and
    $a-b=a+(-b)$ in additive notation.
  \end{rem}
  \begin{defn}
    A \emph{ring} is an ordered triple $(A,+,\cdot)$ such that $(A,+)$ is an
    abelian group, $(A\setminus \{0\},\cdot)$ is a monoid, and the
    \emph{distributive laws} hold:
    \[a\cdot(b+c)=ab+ac\quad\text{and}\quad (a+b)\cdot c=ac+bc.\]
    It is \emph{commutative} if $\cdot$ is commutative.

    It is \emph{ordered} if there is a total order $\le$ on $A$ satisfying
    \begin{itemize}
      \item if $a\le b$ then $a+c\le b+c$, and
      \item if $0\le a$ and $0\le b$ then $0\le ab$.
    \end{itemize}
  \end{defn}
  \begin{defn}
    A \emph{field} is a commutative ring $(A,+,\cdot)$ such that $(A\setminus
    \{0\},\cdot)$ is a group.

    An \emph{ordered field} is a field that is also an ordered ring.
  \end{defn}
  \begin{defn}
    In an ordered ring $R$, the \emph{absolute value} $|a|$ of an element $a$
    of $R$ is $a$ if $0\le a$, otherwise $-a$.
  \end{defn}
  \begin{prop}
    $|a+b|\le|a|+|b|$.
  \end{prop}
  \begin{defn}
    Let $X$ and $Y$ be similar well-ordered sets, and let $A$ and $B$ be the
    least elements of $X$ and $Y$ respectively. Assume that all other elements
    of $X$ and $Y$ are operations on $A$ and $B$ respectively, and let $f$ be
    the similarity between $A$ and $B$.

    A function $\varphi:A\to B$ is said
    to be a \emph{homomorphism} if
    for every $a,b\in A$ and every $\cdot\in X\setminus \{A\}$ we
    have \[\varphi(a\cdot b)=\varphi(a) f(\cdot) \varphi(b).\]

    An \emph{isomorphism} is a bijective homorphism.

    If there exists an isomorphism from $A$ to $B$, then we say $A$ and $B$ are
    \emph{isomorphic}.
  \end{defn}
  \begin{prop}
    The property of being isomorphic is reflexive, symmetric and transitive.
  \end{prop}
  \begin{rem}
    We don't say that isomorphism is an equivalence relation, since it would
    imply there exists a set of all well-ordered sets of this type.

    Such a set does not exist because if it did it would contain
    $(S,\mathrm{Id}_S)$ for each set $S$. Then we could use specification to
    extract the set containing exactly those elements, and
    Proposition~\ref{prop:1:orderedpairsinprod} to extract a set of all sets.
  \end{rem}
  \begin{thm}
    There exists a unique ordered ring $\Zz$ (up to isomorphism) such that $\{x\in
      \Zz:x\ge 0\}$ is well-ordered.

    $\Zz$ is commutative.
  \end{thm}
  \begin{defn}
    The \emph{integers}, $\Zz$, are a well-ordered ring. The \emph{non-negative
      integers} $\Zz_{\ge 0}$ are $\{n\in\Zz: n\ge 0\}$. The \emph{positive
      integers} $\Zz^+$ are $\Zz_{\ge 0}\setminus \{0\}$.
  \end{defn}
  \begin{rem}
    We avoid use of the term \emph{natural numbers}, and the symbol $\mathbb N$, since
    some use them to mean the positive integers and others use them to mean the
    nonnegative integers.
  \end{rem}
  \begin{prop}
    $\Zz_{\ge 0}$ is similar to $\omega$.
  \end{prop}
  \begin{rem}
    Thus, we may identify $\omega$ with $\Zz_{\ge 0}$. In particular, the
    cardinality of a finite set is a nonnegative integer.
  \end{rem}
  \begin{prop}
    Every ordered ring contains a unique subring isomorphic to $\Zz$.
  \end{prop}
  \begin{defn}
    In $\Zz\times\Zz^+$, we define the operations
    \[(a,b)+(c,d)=(ad+bc,bd),\qquad (a,b)(c,d)=(ac,bd).\]
    We also define an equivalence relation $\sim$ where
    $(a,b)\sim (c,d)\iff ad=bc$.

    We define the \emph{rational numbers} $\Qq$ as the partition
    of $\Zz\times\Zz^+$ induced by this equivalence relation, with
    $[(a,b)]+[(c,d)]=[(ad+bc,ac+bd)]$ and $[(a,b)]\cdot [(c,d)]=[(ac,bd)]$.
  \end{defn}
  \begin{prop}
    The relation $\sim$ is an equivalence relation. Moreover, the operations $+$
    and $\cdot$ are uniquely defined. With these operations, $\mathbb Q$ is a
    field.
  \end{prop}
  \begin{prop}
    Every ordered field contains a unique subfield isomorphic to $\Qq$.
  \end{prop}
  \begin{defn}
    A partially ordered set $S$ is \emph{complete} if every nonempty subset that has
    an upper bound in $S$ has a least upper bound in $S$.
  \end{defn}
  \begin{prop}
    Let $S$ be a complete partially ordered set. Every nonempty subset that
    has a lower bound in $S$ has a greatest lower bound in $S$.
  \end{prop}
  \begin{thm}
    There exists a unique complete ordered field, up to isomorphism.
  \end{thm}
  \begin{defn}
    We call this field $\Rr$.
  \end{defn}
  \begin{defn}
    We define $\Qq_{\ge 0},\ \Qq^+,\ \Rr_{\ge 0},\ \Rr^+$ in an analogous way to
    $\Zz_{\ge 0}$ and $\Zz^+$.
  \end{defn}
  \begin{defn}
    We define the \emph{complex numbers} $\Cc$ as $\Rr^2$, with the operations
    \[(a,b)+(c,d)=(a+c,b+d),\qquad (a,b)\cdot(c,d)=(ac-bd,ad+bc).\]

    We usually write $(a,b)$ as $a+bi$. We define the \emph{conjugate} of $a+bi$
    to be $\overline{a+bi}=a-bi$.
  \end{defn}
  \begin{prop}
    $\Cc$ is a field under these operations.
  \end{prop}
  \begin{prop}
    There are unique homomorphisms $\Zz\to\Qq$, $\Qq\to\Rr$ and $\Qq\to\Cc$.
    There is also a homomorphism $\Rr\to\Cc$.
  \end{prop}
  \begin{rem}
    Because of this, we usually take $\Zz\subseteq\Qq\subseteq\Rr\subseteq\Cc$.
  \end{rem}
  \begin{prop}
    Let $a\in\Cc$. Then, $a\overline{a}\in\mathbb R_{\ge 0}$.
  \end{prop}
  \begin{prop}
    Let $b\in\Rr_{\ge 0}$. There exists a unique $x\in\Rr_{\ge 0}$ such that
    $x\cdot x=b$. 
  \end{prop}
  \begin{defn}
    We call $x$ the \emph{square root} of $b$, denoted $\sqrt b$.

    We call $\sqrt{a\overline a}$ the \emph{modulus} of $a$, denoted $|a|$.
  \end{defn}
  \begin{prop}
    $|a+b|\le|a|+|b|$.
  \end{prop}
  \begin{thm}
    $|\Zz^+|=|\Zz_{\ge 0}|=|\Zz|=|\Qq|=\omega$, but $|\Rr|=|\Cc|=|\mathcal
    P(\omega)|$.
  \end{thm}

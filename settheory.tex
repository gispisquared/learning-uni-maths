\chapter{Set Theory}
\begin{axiom}[Existence]
  There exists a set.
\end{axiom}
\begin{rem}
  This is implied by the Axiom of Infinity; however, we include it here so that
  we may define the empty set.
\end{rem}
\begin{defn}
  A \emph{sentence} is made by combining assertions of belonging (eg $x\in A$)
  and/or assertions of equality (eg $A=B$) using the usual logical operators:
  \emph{and, or, not, implies, if and only if, there exists, for all}.
\end{defn}
\begin{defn}
  Let $A$ and $B$ be sets.
  If every element of $A$ is an element of $B$, we say that $A$ is a
  \emph{subset} of $B$, denoted $A\subseteq B$.
\end{defn}
\begin{prop}
  If $A\subseteq B$ and $B\subseteq C$ then $A\subseteq C$.
\end{prop}
\begin{axiom}[Extensionality]
  $A=B$ iff $A\subseteq B$ and $B\subseteq A$.
\end{axiom}
\begin{axiom}[Specification]
  For every set $A$ and every sentence $S(x)$ there is a set $B$ whose elements
  are exactly those elements $x$ of $A$ for which $S(x)$ holds.
\end{axiom}
\begin{defn}
  We notate this set $B$ by $\{x\in A: S(x)\}$.
\end{defn}
\begin{prop}
  There exists a unique set $X$ such that for any $x$, the sentence
  $x\in X$ is false.
\end{prop}
\begin{defn}
  We call this set the \emph{empty set}, notated $\emptyset$.
\end{defn}
\begin{prop}
  For every set $A$ there is a set $B$ such that $B\not\in A$.
\end{prop}
\begin{axiom}[Pairing]
  For any two sets $A$ and $B$ there is a set $X$ with $A\in X$ and $B\in X$.
\end{axiom}
\begin{prop}
  There is a unique set $Y$ such that for any $a$, $a$ is in $Y$ iff $a=A$ or $a=B$.
\end{prop}
\begin{defn}
  This set is called the \emph{unordered pair} formed by $A$ and $B$, denoted
  $\{A,B\}$.
\end{defn}
\begin{defn}
  The set $\{A,A\}$ is denoted $\{A\}$, and called the \emph{singleton} of
  $\{A\}$.
\end{defn}
\begin{axiom}[Union]
  For any set $X$ of sets there exists a set $Y$ such that for any $A$ in $X$,
  and any $a$ in $A$, $a$ is in $Y$.
\end{axiom}
\begin{prop}
  For a nonempty set $X$ of sets there is a unique set $Z$ such that $a$ is in $Z$ if and
  only if there exists an $A$ in $X$ such that $a$ is in $A$.
\end{prop}
\begin{defn}
  This set is called the \emph{union} of $X$, denoted
  $\bigcup X$.

  For two sets $A$ and $B$ we define $A\cup B=\bigcup \{A,B\}$.
\end{defn}
\begin{defn}
  Let $A$ and $B$ be sets.
  The \emph{intersection} of $A$ and $B$, notated $A\cap B$, is $\{x\in A:x\in
    B\}$.

  If $A\cap B=\emptyset$ then $A$ and $B$ are called \emph{disjoint}.
\end{defn}
\begin{prop}
  We have
  \begin{itemize}
    \item $A\cup\emptyset=A$,
    \item $A\cup B=B\cup A$ (commutative),
    \item $A\cup (B\cup C)=(A\cup B)\cup C$ (associative),
    \item $A\cup A=A$ (idempotent),
    \item $A\cup (B\cap C)=(A\cup B)\cap(A\cup C)$ (distributive),
    \item $A\subseteq B$ iff $A\cup B=B$,
    \item $A\cap\emptyset=A$,
    \item $A\cap B=B\cap A$ (commutative),
    \item $A\cap (B\cap C)=(A\cap B)\cap C$ (associative),
    \item $A\cap A=A$ (idempotent),
    \item $A\cap (B\cup C)=(A\cap B)\cup(A\cap C)$ (distributive),
    \item $A\subseteq B$ iff $A\cap B=A$.
  \end{itemize}
\end{prop}
\begin{prop}
  For every nonempty set $C$ of sets, there is a unique set $Y$ such that $x\in Y$ iff
  $x\in X$ for each $X$ in $C$.
\end{prop}
\begin{defn}
  This set $Y$ is called the \emph{intersection} of $C$, denoted $\bigcap C$.
\end{defn}
\begin{axiom}[Powers]
  For each set $X$ there is a set that contains all subsets of
  $X$.
\end{axiom}
\begin{prop}
  There is a unique set $Y$ such that $x\in Y$ iff $x\subseteq X$.
\end{prop}
\begin{defn}
  This set $Y$ is called the \emph{power set} of $X$, denoted $\mathcal P(X)$.
\end{defn}
\begin{defn}
  The \emph{ordered pair} of $a$ and $b$ is the set defined as
  \[(a,b)=\{\{a\},\{a,b\}\}.\]
\end{defn}
\begin{prop}
  For any $a,b,c,d$, we have $(a,b)=(c,d)$ iff $a=c$ and $b=d$.
\end{prop}
\begin{defn}
  Let $A$ and $B$ be sets. The \emph{Cartesian product} $A\times B$ is
  \[\{(x,y): x\in A,\ y\in B\}.\]
\end{defn}
\begin{prop}
  For any set $R$ of ordered pairs there are sets $A$ and $B$ such that
  $R\subseteq A\times B$.
\end{prop}
\begin{defn}
  A \emph{binary relation} $R$ over sets $A$ and $B$ is a subset of $A\times
  B$. If $(a,b)$ is in $R$ we write $aRb$.

  If $A=B$ then we call it a \emph{binary relation over} $A$.
\end{defn}
\begin{defn}
  An \emph{equivalence relation} is a binary relation $\sim$ over $A$ such
  that
  \begin{itemize}
    \item $a\sim a$ (reflexive),
    \item $a\sim b\iff b\sim a$ (symmetric), and
    \item if $a\sim b$ and $b\sim c$ then $a\sim c$ (transitive).
  \end{itemize}
  The \emph{equivalence class} of $a$ under $\sim$ is
  \[[a]=\{x\in A:x\sim a\}.\]
\end{defn}
\begin{defn}
  A \emph{partition} of a set $A$ is a disjoint set of subsets of $A$ whose
  union is $A$.

  A partition $X$ of $A$ \emph{induces} a relation $\sim$, where $a\sim b$ iff
  $a$ and $b$ belong to the same element of $X$.
\end{defn}
\begin{prop}
  The set of equivalence classes of an equivalence relation exists and 
  is a partition.
\end{prop}
\begin{defn}
  This partition is called the partition \emph{induced} by the equivalence
  relation $\sim$.
\end{defn}
\begin{prop}
  The equivalence relation induced by a partition induces that partition; the
  partition induced by an equivalence relation induces that relation.
\end{prop}
\begin{defn}
  For any set $X$ we define $X^+=X\cup\{X\}$.
\end{defn}
\begin{axiom}[Infinity]
  There exists a set $S$ containing $\emptyset$ and containing $X^+$ for
  every $X$ in $S$.
\end{axiom}
\begin{prop}
  There exists a unique set $\omega$ which is a subset of all such sets $S$.
\end{prop}
\begin{prop}
  For any $a,b\in\omega$, exactly one of $a\in b,\ a=b,\ b\in a$ is true.
\end{prop}
\begin{prop}
  For any $a\in\omega$ and any $b\in a$, $b\subseteq a$.
\end{prop}
\begin{defn}
  A \emph{function} $f:A\to B$ is a relation $f$ over $A$ and $B$ such that
  for each $a\in A$ there is exactly one $b\in B$ such that $afb$. We usually
  write this as $f(a)=b$.

  A function $f$ is \emph{injective} if for each $b$ in $B$, there is at most one
  $a$ in $A$ such that $f(a)=b$. It is \emph{surjective} if for each $b$ in
  $B$ there is at least one $a$ in $A$ such that $f(a)=b$. A function which is
  both injective and surjective is \emph{bijective}.
\end{defn}
\begin{thm}[Recursion theorem]
  If $a$ is an element of a set $X$, and if $f:X\to X$ is a function, then there
  is a function $g:\omega\to X$ such that $u(0)=a$ and $u(n^+)=f(u(n))$ for all
  $n$ in $\omega$.
\end{thm}
\begin{axiom}[Substitution]
  If $S(a,b)$ is a sentence such that for each $a$ in a set $A$ there exists a
  set $B_a$ such that $b\in B_a\iff S(a,b)$, then there exists a function $F$ with
  domain $A$ such that $F(a)=B_a$ for each $a$ in $A$.
\end{axiom}
\begin{axiom}[Foundation]
  Every set $X$ contains a set $Y$ such that $X$ and $Y$ are disjoint.
\end{axiom}
\begin{axiom}[Choice]
  Let $X$ be a set of sets whose members are all nonempty. Then there exists a
  function $f:X\to\bigcup X$ such that $f(Y)\in Y$ for all $Y\in X$.
\end{axiom}

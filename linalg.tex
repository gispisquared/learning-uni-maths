%chktex-file 3
\chapter{Linear Algebra}
\begin{defn}
  Let $\Ff$ be a field. A \emph{vector space over} $\Ff$ is an abelian group
  $V$ (of \emph{vectors})
  together with a function $\cdot:\Ff\times V\to V$ (\emph{scalar
    multiplication}) such that
  \begin{itemize}
    \item $a(b\mathbf v)=(ab)\mathbf v$ (compatible),
    \item $1\mathbf v=\mathbf v$ (identity), and
    \item $a(\mathbf u+\mathbf v)=a\mathbf u+a\mathbf v$ and $(a+b)\mathbf
      v=a\mathbf v+b\mathbf v$ (distributive).
  \end{itemize}
\end{defn}
\begin{defn}
  Let $S$ be a subset of $V$. A \emph{linear combination} of elements of $S$ is
  a vector of the form \[\sum_{i=1}^n a_i\mathbf s_i,\] where each $s_i$ is a
  distinct element of $S$.
\end{defn}
\begin{defn}
  A \emph{subspace} $W$ of $V$ is a nonempty subset of $V$ which is also a
  vector space over $\Ff$.
\end{defn}
\begin{prop}
  A subset $W$ of $V$ is a subspace iff the following conditions hold:
  \begin{itemize}
    \item $W$ is nonempty;
    \item $u,v\in W$ implies $u+v\in W$ (closed under addition); and
    \item if $a\in\Ff$ and $u\in W$ then $au\in W$ (closed under scalar
      multiplication).
  \end{itemize}
\end{prop}
\begin{defn}
  The \emph{span} of a subset $S$ of $V$ is the intersection of all linear
  subspaces of $V$ that contain $S$.  
\end{defn}
\begin{prop}
  The span of $S$ is the set of linear combinations of elements of $S$. It is
  also the smallest subspace of $V$ that contains $S$.
\end{prop}
\begin{defn}
  A subset $S$ of $V$ is \emph{linearly independent} if any linear combination
  of elements of $S$ that produces $\mathbf 0$ has all coefficients equal to $0$.
  Otherwise, it is \emph{linearly dependent}.
\end{defn}
\begin{defn}
  A subset $S$ of $V$ is a \emph{basis} if it is linearly independent and its
  span is $V$.
\end{defn}
\begin{thm}
  Let $V$ be a vector space.
  \begin{itemize}
    \item $V$ has a basis.
    \item Any two bases of $V$ have the same cardinality.
  \end{itemize}
\end{thm}
\begin{defn}
  The \emph{dimension} of $V$ is the cardinality of a basis of $V$. If $\dim V$
  is an integer, $V$ is said to be \emph{finite-dimensional}; otherwise, it is
  \emph{infinite-dimensional}.
\end{defn}
\begin{prop}
  Let $V$ be finite-dimensional with dimension $d$.
  Let $S$ be a set of vectors in $V$ with $|S|=d$. Then $S$ is linearly
  independent iff it spans $V$.
\end{prop}
\begin{defn}
  An \emph{inner product space} is a vector space $V$ over a field $\Ff$ which
  is either $\Rr$ or $\Cc$, together with a function
  $\langle\cdot,\cdot\rangle:V\times V\to\Ff$ satisfying
  \begin{itemize}
    \item $\langle \mathbf x,\mathbf y\rangle=\overline{\langle \mathbf
      y,\mathbf x\rangle}$ (conjugate
      symmetry)
    \item $\langle a\mathbf x+b\mathbf y,\mathbf z\rangle=a\langle\mathbf
      x,\mathbf z\rangle+b\langle\mathbf y,\mathbf z\rangle$ (linearity in the
      first argument), and
    \item $\langle\mathbf x,\mathbf x\rangle=0\implies\mathbf x=\mathbf 0$.
  \end{itemize}
\end{defn}
\begin{defn}
  A \emph{normed vector space} is a vector space $V$ over $\Rr$ or $\Cc$
  on which there is a \emph{norm}: a function $\|\cdot\|:V\to\Cc$ satisfying
  \begin{itemize}
    \item $\|\mathbf x\|\ge 0$,
    \item $\|\mathbf x\|=0$ implies $\mathbf x=\mathbf 0$,
    \item $\|a\mathbf x\|=|a|\|\mathbf x\|$, and
    \item $\|\mathbf x+\mathbf y\|\le\|\mathbf x\|+\|\mathbf y\|$ (the triangle
      inequality).
  \end{itemize}
\end{defn}
\begin{prop}
  If $V$ is an inner product space, then $\langle\mathbf x,\mathbf x\rangle$ is
  real for all $\mathbf x$.
  Moreover, $\|\mathbf x\|=\sqrt{\langle\mathbf x,\mathbf x\rangle}$ is a norm
  on $V$.
\end{prop}
\begin{defn}
  Two vectors $\mathbf x$ and $\mathbf y$ are \emph{orthogonal} if $\langle
  \mathbf x,\mathbf y\rangle=0$.

  A set of vectors is \emph{orthonormal} if each vector in the set has norm 1
  and is orthogonal to all other vectors in the set.
\end{defn}
\begin{prop}
  Any finite-dimensional vector space has an orthonormal basis.
\end{prop}

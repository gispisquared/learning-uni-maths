\chapter{Cardinality}
\begin{defn}
  Two sets $A$ and $B$ are said to have the same \emph{cardinality} (written
  $|A|=|B|$) if there is a bijection $f:A\to B$.

  A set $A$ has cardinality at most the cardinality of $B$ ($|A|\le|B|$) if
  there is an injection $f:A\to B$.

  A set $A$ has cardinality less than the cardinality of $B$ ($|A|<|B|$) if
  $|A|\le|B|$ and $|A|\ne|B|$.
\end{defn}
\begin{thm}
  If $|A|\le|B|$ and $|B|\le|A|$ then $|A|=|B|$.
\end{thm}
\begin{thm}
  For any set $A$, $|\mathcal P(A)|>|A|$.
\end{thm}
\begin{defn}
  A \emph{cardinal number} is an ordinal number $\alpha$ such that for any
  ordinal number $\beta$ with $|\alpha|=|\beta|$ we have $\alpha\in\beta$.
\end{defn}
\begin{prop}
  For any set $S$, there is a unique cardinal number $\alpha$ with
  $|\alpha|=|S|$.
\end{prop}
\begin{defn}
  For these sets $S$ and $\alpha$ we define $|S|=\alpha$.
\end{defn}
\begin{defn}
  A set $A$ is said to be \emph{finite} if $|A|\in\omega$, and \emph{infinite}
  otherwise.
\end{defn}
\begin{prop}
  A set is infinite if and only if it has the same cardinality as some proper
  subset.
\end{prop}
\begin{defn}
  An infinite set $A$ is said to be \emph{countable} if $|A|=\omega$, and
  \emph{uncountable} otherwise.
\end{defn}
\begin{prop}
  A countable set does not have any uncountable subsets. An uncountable set has
  a countable subset.
\end{prop}

\section{Number Systems}
\begin{defn}
    For a nonnegative integer $n$, we define an \emph{operation of arity} $n$ on
    a set $S$ as a function $f:S^n\to S$. We use the convention $S^0=1=\{0\}$,
    so an operation of arity $0$ selects an element of $S$.
\end{defn}
\begin{defn}
    A \emph{binary operation} $\cdot$ is an operation of arity $2$. We
    usually write $\cdot(a,b)=c$ as $a\cdot b=c$.

    It is \emph{associative} if $(a\cdot b)\cdot c=a\cdot(b\cdot
    c)$ for any $a,\ b,\ c$ in $A$.

    It is \emph{commutative} if $a\cdot b=b\cdot a$ for any $a,\ b$ in $A$.
\end{defn}
\begin{defn}
      An \emph{algebraic structure} is an ordered pair $(S,O)$ where $S$ is a
      set and $O$ is an indexed set of operations on $S$.
\end{defn}
\begin{defn}
    A \emph{semigroup} is a set $A$ together with an associative binary
    operation on $A$.
\end{defn}
\begin{rem}
    There are two main notations for semigroups. These are
    \begin{itemize}
        \item Multiplicative notation, in which the operation is notated $a\cdot b$ or
        simply $ab$, and the identity element (if it exists) is $1$; and
        \item Additive notation, in which the operation is notated $a+b$ and the
        identity element (if it exists) is $0$.
    \end{itemize}
\end{rem}
\begin{defn}
    An element $e\in A$ is called an \emph{identity} for $\cdot$ if for each $x$
    we have $x\cdot e=e\cdot x=x$.
\end{defn}
\begin{defn}
      A \emph{monoid} is a semigroup with a nullary operation that selects
      an identity.
\end{defn}
\begin{defn}
    A partial order $\le$ and a binary operation $\cdot$ are said to be
    \emph{compatible} if $x\le y$ implies $ax\le ay$ and $xa\le ya$.
\end{defn}
\begin{defn}
    Let $X$ be a monoid with a compatible partial order $\ge$.
    The sets $X_{\ge 0}$ and $X_{>0}=X^+$ are defined as
    $\{x\in X:x\ge 0\}$ and $X_{\ge 0}\setminus\{0\}$.
\end{defn}
\begin{defn}
    A \emph{group} is a monoid $A$ with a unary operation $\cdot^{-1}$ such that
    for each $a\in A$ we have $aa^{-1}=a^{-1}a=1$.
    
    A group is \emph{abelian} if its binary operation is commutative.
\end{defn}
\begin{defn}
    A \emph{rng} is an (additive) abelian group $A$ with a binary operation
    $\cdot$ such that $(A,\cdot)$ is a semigroup, and the
    \emph{distributive laws} hold:
    \[a\cdot(b+c)=ab+ac\quad\text{and}\quad (a+b)\cdot c=ac+bc.\]

    It is \emph{commutative} if $\cdot$ is commutative.

    It is \emph{ordered} if there is a total order $\le$ on $A$ compatible with
    $+$ such that $A_{\ge 0}$ is closed under $\cdot$.

    It is a \emph{ring} if $(A,\cdot)$ is a monoid.
  \end{defn}
  \begin{rem}
      Some authors use ``ring'' for what we call a rng, and ``ring with
      identity'' or similar for what we call a ring.
  \end{rem}
  \begin{defn}
    A \emph{field} is a ring $(A,+,\cdot)$ such that $(A\setminus
    \{0\},\cdot)$ is an abelian group.

    An \emph{ordered field} is a field that is also an ordered ring.
  \end{defn}
  \begin{defn}
    Let $(A,X)$ and $(B,Y)$ be algebraic structures such that $X$ and $Y$ are
    indexed by the same set.

    A function $\varphi:A\to B$ is said
    to be a \emph{homomorphism}, or \emph{morphism}, if
    for every $a,b\in A$ and every $i$ we have
    have \[\varphi(a X_i b)=\varphi(a) Y_i \varphi(b).\]
\end{defn}
\begin{defn}
    An \emph{isomorphism} is a bijective homomorphism. An isomorphism from a set
    to itself is an \emph{automorphism}.

    If there exists an isomorphism from $A$ to $B$, then we say $A$ and $B$ are
    \emph{isomorphic}, denoted $A\cong B$.
  \end{defn}
  \begin{prop}
    The property of being isomorphic is reflexive, symmetric and transitive.
  \end{prop}
  \begin{rem}
    We don't say that isomorphism is an equivalence relation, since this would
    imply the existence of a set of all sets. (Consider the union of all
    sets isomorphic to the trivial group.)
  \end{rem}
  \begin{defn}
      Let $A$ be a [group, ring, etc] and let $S$ be a subset of $A$. 
      If $S$ is also a [group, ring, etc], then $S$ is called a \emph{sub}[group,
      ring, etc] of $A$. Conversely, $A$ is a [group, ring, etc]
      \emph{extension} of $S$.
  \end{defn}
  \begin{prop}
      The intersection of any collection of sub[group, ring, etc]s of $G$ is again
      a sub[group, ring, etc] of $G$.
  \end{prop}
\begin{defn}
    The \emph{direct product} of an indexed set $\{G_i\}_{i\in I}$ of algebraic structures is
    the set of sequences $\{g_i\}_{i\in I}$ such that each $g_i$ is in $G_i$, with operations
    defined componentwise.
\end{defn}
\begin{prop}
    The direct product of a set of [groups, rings, etc]\footnote{except for
    fields} is again a [group, ring, etc].
\end{prop}
\begin{prop}
    Let $R$ be a commutative ring and let $x\not\in R$.
    There exists a unique ring extension $R[x]$ of $R$ up to isomorphism such
    that for every ring extension $S$ of $R$ and every element $y$ of $S$,
    there is a unique ring homomorphism $f_y:R[x]\to S$ which fixes $R$ and
    sends $x$ to $y$.
\end{prop}
\begin{defn}
    This ring is called the \emph{polynomial ring} over $R$; if $p\in R[x]$,
    we define $p$ to be a function on $S$ by $f_y(p)=p(y)$.
\end{defn}
\begin{prop}
    Every element in $R[x]$ can be uniquely written as $\sum_{i=1}^\infty r_i
    x^i$, where all but finitely many $r_i$s are $0$.
\end{prop}
\begin{defn}
    The \emph{degree} of a polynomial is the largest $i$ such that $r_i\ne 0$.
\end{defn}
\begin{defn}
    An equivalence relation $\sim$ and a binary operation $\cdot$, both over $A$, are said
    to be \emph{compatible} if $a_1\sim a_2$ and $b_1\sim b_2$ imply $a_1\cdot
    b_1\sim a_2\cdot b_2$. In this case, we may define the operation
    $\cdot$ \emph{induced} on $A/\sim$ by $\cdot$ as $[a]\cdot[b]=[a\cdot b]$.
\end{defn}
\begin{prop}
    If $A$ is a [monoid, group, etc] and $\sim$ is a nontrivial equivalence relation compatible with
    all of its operations, then $A/\sim$ is a [monoid, group, etc] under the operations
    induced on it.
\end{prop}
\begin{defn}
    A partially ordered set $S$ is \emph{complete} if every nonempty subset that has
    an upper bound in $S$ has a least upper bound in $S$.
\end{defn}
\begin{prop}
    Let $S$ be a complete partially ordered set. Every nonempty subset that
    has a lower bound in $S$ has a greatest lower bound in $S$.
\end{prop}
\begin{thm}
    There are structures $\Zz\subseteq\Qq\subseteq\Rr$ satisfying the
    following properties:
    \begin{itemize}
        \item $\Zz$ is a ring, and if $R$ is a ring, there is a unique homomorphism
              $f:\Zz\to R$.
        \item $\Qq$ is a field, and if $F$ is a field containing $\Zz$, there
              is a unique homomorphism $f:\Qq\to F$.
        \item $\Rr$ is a complete totally ordered field.
    \end{itemize}
    Moreover, each of these properties defines the corresponding structure up
    to unique isomorphism.
\end{thm}
\begin{prop}
    There are unique orders on $\Zz$ and $\Qq$ compatible with $+$ such that
    $1\ge 0$.
\end{prop}
\begin{prop}[Principle of Mathematical Induction]
    If $S\subseteq\Zz^+$ such that $1\in S$ and if $x\in S$ then $x+1\in S$,
    then $S=\Zz^+$.
\end{prop}
\begin{defn}
    The structures thus defined are the \emph{nonnegative integers},
    \emph{integers}, \emph{rational numbers} and \emph{real numbers}
    respectively.
\end{defn}
\begin{rem}
    We avoid the symbol $\Nn$ and the term \emph{natural numbers},
    since some sources take it to mean $\Zz_{>0}$ and others take it to mean
    $\Zz_{\ge 0}$.
\end{rem}
\begin{defn}
    We define the \emph{complex numbers} $\Cc$ as $\Rr[x]/\sim$, where $\sim$
    is the equivalence relation given by $a\sim b\iff\exists c:a=b+c(x^2+1)$.
    The equivalence class of $x$ is denoted $i$.
\end{defn}
\begin{prop}
    There is a unique nontrivial automorphism of $\Cc$ fixing $\Rr$.
\end{prop}
\begin{defn}
    This automorphism is called \emph{complex conjugation}; the image of $a$
    is denoted $\overline a$.
\end{defn}
\begin{prop}
    If $a\in\Cc$, then $a\overline a\in\Rr_{\ge 0}$.
\end{prop}
\begin{prop}
    Let $b\in\Rr_{\ge 0}$. There exists a unique $x\in\Rr_{\ge 0}$ such that
    $x\cdot x=b$. 
\end{prop}
\begin{defn}
    We call $x$ the \emph{square root} of $b$, denoted $\sqrt b$.
\end{defn}
\begin{defn}
    We call $\sqrt{a\overline a}$ the \emph{modulus} of $a$, denoted $|a|$.
\end{defn}
\begin{prop}[Triangle Inequality]
    If $a$ and $b$ are complex numbers, then $|a+b|\le|a|+|b|$.
\end{prop}
\begin{thm}[Fundamental Theorem of Algebra]
      For all $P\in\Cc[x]\setminus\Cc$, there is a complex number $z$ such that
      $P(z)=0$.
\end{thm}
\begin{rem}
      We assume this theorem for now, and prove it in the section on complex
      analysis.
\end{rem}
\begin{thm}
    $|\Zz^+|=|\Zz_{\ge 0}|=|\Zz|=|\Qq|=\omega$, but $|\Rr|=|\Cc|=|\mathcal
    P(\omega)|$.
\end{thm}
\subsection*{References}
\begin{itemize}
    \item Landau, \emph{Foundations of Analysis}
    \item Birkhoff and Mac Lane, \emph{A Survey of Modern Algebra}
\end{itemize}

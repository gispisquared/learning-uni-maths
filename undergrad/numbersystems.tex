\section{Number Systems}
  \begin{defn}
    A \emph{binary operation} on $A$ is a function $\cdot:A\times A\to A$. We
    usually write $\cdot(a,b)=c$ as $a\cdot b=c$.

    It is \emph{associative} if $(a\cdot b)\cdot c=a\cdot(b\cdot
    c)$ for any $a,\ b,\ c$ in $A$.

    It is \emph{commutative} if $a\cdot b=b\cdot a$ for any $a,\ b$ in $A$.
  \end{defn}
  \begin{defn}
      An \emph{algebraic structure} is an ordered pair $(S,O)$ where $S$ is a
      set and $O$ is an indexed set of binary operations on $S$.
  \end{defn}
  \begin{defn}
    A \emph{semigroup} is a set $A$ together with an associative binary
    operation on $A$.
  \end{defn}
  \begin{defn}
    A partial order $\le$ and a binary operation $\cdot$ are said to be
    \emph{compatible} if $x\le y$ implies $ax\le ay$ and $xa\le ya$.
  \end{defn}
  \begin{defn}
    An element $e\in A$ is called an \emph{identity} for $\cdot$ if for each $x$
    we have $x\cdot e=e\cdot x=x$.
  \end{defn}
  \begin{rem}
    There are two main notations for semigroups. These are
    \begin{itemize}
      \item 
        Multiplicative notation, in which the operation is notated $a\cdot b$ or
        simply $ab$, and the identity element (if it exists) is $1$; and
      \item Additive notation, in which the operation is notated $a+b$ and the
        identity element (if it exists) is $0$.
    \end{itemize}
  \end{rem}
  \begin{defn}
    A monoid is a semigroup with an identity.
  \end{defn}
  \begin{defn}
      Let $X$ be a monoid. The sets $X_{\ge 0}$ and $X_{>0}=X^+$ are defined as
      $\{x\in X:x\ge 0\}$ and $X_{\ge 0}\setminus\{0\}$.
  \end{defn}
  \begin{defn}
    A \emph{group} is a monoid $A$ such that for each element $a$ of $A$
    there is an element $b$ of $A$ such that
    $ab=1=ba$.
    
    A group is \emph{abelian} if the operation is commutative.
  \end{defn}
  \begin{defn}
    A \emph{ring} is set $A$ together with two operations $+$ and $\cdot$ such
    that $(A,+)$ is an
    abelian group, $(A,\cdot)$ is a semigroup, and the
    \emph{distributive laws} hold:
    \[a\cdot(b+c)=ab+ac\quad\text{and}\quad (a+b)\cdot c=ac+bc.\]

    It is \emph{commutative} if $\cdot$ is commutative.

    It is \emph{ordered} if there is an order $\le$ on $A$ compatible with $+$
    such that if $0\le a$ and $0\le b$ then $0\le ab$.

    It is a \emph{ring with unity} if $\cdot$ has an identity.
  \end{defn}
  \begin{defn}
    Let $(A,X)$ and $(B,Y)$ be algebraic structures such that $X$ and $Y$ are
    indexed by the same set.

    A function $\varphi:A\to B$ is said
    to be a \emph{homomorphism}, or \emph{morphism}, with respect to these operations if
    for every $a,b\in A$ and every $i$ we have
    have \[\varphi(a X_i b)=\varphi(a) Y_i \varphi(b).\]
\end{defn}
\begin{defn}
    An \emph{isomorphism} is a bijective homomorphism. An isomorphism from a set
    to itself is an \emph{automorphism}.

    If there exists an isomorphism from $A$ to $B$, then we say $A$ and $B$ are
    \emph{isomorphic}, denoted $A\cong B$.
  \end{defn}
  \begin{prop}
    The property of being isomorphic is reflexive, symmetric and transitive.
  \end{prop}
  \begin{rem}
    We don't say that isomorphism is an equivalence relation, since this would
    imply the existence of a set of all sets. (Consider the union of all
    sets isomorphic to the trivial group.)
  \end{rem}
  \begin{defn}
      Let $A$ be a [group, ring, etc] and let $S$ be a subset of $A$. 
      If $S$ is also a [group, ring, etc], then $S$ is called a \emph{sub}[group,
      ring, etc] of $A$. Conversely, $A$ is a [group, ring, etc]
      \emph{extension} of $S$.
  \end{defn}
  \begin{prop}
      The intersection of any collection of sub[group ring etc]s of $G$ is again
      a sub[group, ring, etc] of $G$.
  \end{prop}
\begin{defn}
    The \emph{direct product} of an indexed set $G_i$ of algebraic structures is
    the set of sequences $g_i$ such that each $g_i$ is in $G_i$, with operations
    defined componentwise.
\end{defn}
\begin{prop}
    The direct product of a set of [groups, rings, etc]\footnote{except for
    fields} is again a [group, ring, etc].
\end{prop}
  \begin{prop}
      Let $R$ be a commutative ring with unity and let $x\not\in R$.
      There exists a unique ring extension $R[x]$ of $R$ up to isomorphism such
      that for every ring extension $S$ of $R$ and every element $y$ of $S$,
      there is a unique ring homomorphism $f_y:R[x]\to S$ which fixes $R$ and
      sends $x$ to $y$.
  \end{prop}
  \begin{defn}
      This ring is called the \emph{polynomial ring} over $R$; if $p\in R[x]$,
      we define $p$ to be a function on $S$ by $f_y(p)=p(y)$.
  \end{defn}
  \begin{defn}
      An equivalence relation $\sim$ and a binary operation $\cdot$, both over $A$, are said
      to be \emph{compatible} if $a_1\sim a_2$ and $b_1\sim b_2$ imply $a_1\cdot
      b_1\sim a_2\cdot b_2$. In this case, we may define the operation
      $\cdot$ \emph{induced} on $A/\sim$ by $\cdot$ as $[a]\cdot[b]=[a\cdot b]$.
  \end{defn}
  \begin{prop}
      If $A$ is a [monoid, group, etc] and $\sim$ is a nontrivial equivalence relation compatible with
      all of its operations, then $A/\sim$ is a [monoid, group, etc] under the operations
      induced on it.
  \end{prop}
  \begin{defn}
    A \emph{field} is a ring $(A,+,\cdot)$ such that $(A\setminus
    \{0\},\cdot)$ is an abelian group.

    An \emph{ordered field} is a field that is also an ordered ring.
  \end{defn}
  \begin{defn}
    A partially ordered set $S$ is \emph{complete} if every nonempty subset that has
    an upper bound in $S$ has a least upper bound in $S$.
  \end{defn}
  \begin{prop}
    Let $S$ be a complete partially ordered set. Every nonempty subset that
    has a lower bound in $S$ has a greatest lower bound in $S$.
  \end{prop}
  \begin{thm}
      There are structures $\omega=\Zz_{\ge
      0}\subseteq\Zz\subseteq\Qq\subseteq\Rr$ satisfying the
      following properties:
      \begin{itemize}
          \item If $M$ is a monoid and $a\in M$, there is a unique homomorphism
              $f:\Zz_{\ge 0}\to M$ such that $f(1)=a$.
          \item If $R$ is a ring with unity, there is a unique homomorphism
              $f:\Zz\to R$.
          \item If $F$ is a field containing $\Zz$, there is a unique homomorphism $f:\Qq\to F$.
          \item $\Rr$ is a complete totally ordered field.
      \end{itemize}
      Moreover, each of these properties defines the corresponding structure up to unique isomorphism.
  \end{thm}
  \begin{defn}
      The structures thus defined are the \emph{nonnegative integers},
      \emph{integers}, \emph{rational numbers} and \emph{real numbers}
      respectively.
  \end{defn}
  \begin{rem}
      We avoid the symbol $\Nn$ and the term \emph{natural numbers},
      since some sources take it to mean $\Zz_{>0}$ and others take it to mean
      $\Zz_{\ge 0}$.
  \end{rem}
  \begin{defn}
      We define the \emph{complex numbers} $\Cc$ as $\Rr[x]/\sim$, where $\sim$
      is the equivalence relation given by $a\sim b\iff\exists c:a=b+c(x^2+1)$.
  \end{defn}
  \begin{prop}
      There is a unique nontrivial automorphism of $\Cc$ fixing $\Rr$.
  \end{prop}
  \begin{defn}
      This automorphism is called \emph{complex conjugation}; the image of $a$
      is denoted $\overline a$.
  \end{defn}
  \begin{prop}
      If $a\in\Cc$, then $a\overline a\in\Rr_{\ge 0}$.
  \end{prop}
  \begin{prop}
    Let $b\in\Rr_{\ge 0}$. There exists a unique $x\in\Rr_{\ge 0}$ such that
    $x\cdot x=b$. 
  \end{prop}
  \begin{defn}
    We call $x$ the \emph{square root} of $b$, denoted $\sqrt b$.
  \end{defn}
  \begin{defn}
    We call $\sqrt{a\overline a}$ the \emph{modulus} of $a$, denoted $|a|$.
  \end{defn}
  \begin{prop}[Triangle Inequality]
    If $a$ and $b$ are complex numbers, then $|a+b|\le|a|+|b|$.
  \end{prop}
  \begin{thm}[Fundamental Theorem of Algebra]
      For all $P\in\Cc[x]\setminus\Cc$, there is a complex number $z$ such that
      $P(z)=0$.
  \end{thm}
  \begin{rem}
      We assume this theorem for now, and prove it in the section on complex
      analysis.
  \end{rem}
  \begin{thm}
    $|\Zz^+|=|\Zz_{\ge 0}|=|\Zz|=|\Qq|=\omega$, but $|\Rr|=|\Cc|=|\mathcal
    P(\omega)|$.
  \end{thm}
  \begin{prop}
      $|\Rr^\Zz|=|\Zz^\Zz|$.
  \end{prop}

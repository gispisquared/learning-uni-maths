\section{Set Theory}
\begin{defn}
    We define a \emph{set}, or \emph{collection}, as an object which has a
    notion of \emph{elements} ---
    for any set $A$ and any object $B$, either $B$ is an element of $A$ or it
    isn't.
\end{defn}
\begin{axiom}[Existence]
  There exists a set.
\end{axiom}
\begin{defn}
  Let $A$ and $B$ be sets.
  If every element of $A$ is an element of $B$, we say that $A$ is a
  \emph{subset} of $B$, denoted $A\subseteq B$.
\end{defn}
\begin{prop}
  If $A\subseteq B$ and $B\subseteq C$ then $A\subseteq C$.
\end{prop}
\begin{axiom}[Extensionality]
    Let $A$ and $B$ be sets. We have $A\subseteq B$ and $B\subseteq A$ iff $A=B$.
\end{axiom}
\begin{defn}
  A \emph{sentence} is made by combining assertions of belonging (eg $x\in A$)
  and/or assertions of equality (eg $A=B$) using the usual logical operators:
  \emph{and, or, not, implies, if and only if, there exists, for all}.
\end{defn}
\begin{axiom}[Specification]
  For every set $A$, every set $p$ and every sentence $S(x,p)$ there is a set
  $B$ whose elements are exactly those elements $x$ of $A$ for which $S(x,p)$ 
  holds.
\end{axiom}
\begin{defn}
  We notate this set $B$ by $\{x\in A: S(x,p)\}$.
\end{defn}
\begin{prop}
  There exists a unique set $X$ such that for any $x$, the sentence
  $x\in X$ is false.
\end{prop}
\begin{defn}
  We call this set the \emph{empty set}, notated $\emptyset$.
\end{defn}
\begin{prop}
  For every set $A$ there is a set $B$ such that $B\not\in A$.
\end{prop}
\begin{axiom}[Pairing]
  For any two sets $A$ and $B$ there is a set $X$ with $A\in X$ and $B\in X$.
\end{axiom}
\begin{prop}
  There is a unique set $Y$ such that for any $a$, $a$ is in $Y$ iff $a=A$ or $a=B$.
\end{prop}
\begin{defn}
  This set is called the \emph{unordered pair} formed by $A$ and $B$, denoted
  $\{A,B\}$.
\end{defn}
\begin{defn}
  The set $\{A,A\}$ is denoted $\{A\}$, and called the \emph{singleton} of
  $A$.
\end{defn}
\begin{axiom}[Union]
  For any collection $X$ of sets there exists a set $Y$ such that for any $A$ in
  $X$, and any $a$ in $A$, $a$ is in $Y$.
\end{axiom}
\begin{prop}
  For a nonempty collection $X$ of sets there is a unique set $Z$ such that $a$
  is in $Z$ if and only if there exists an $A$ in $X$ such that $a$ is in $A$.
\end{prop}
\begin{defn}
  This set is called the \emph{union} of $X$, denoted
  $\bigcup X$.

  For two sets $A$ and $B$ we define $A\cup B=\bigcup \{A,B\}$.
\end{defn}
\begin{prop}
  For every nonempty collection $C$ of sets, there is a unique set $Y$ such that
  $x\in Y$ iff $x\in X$ for each $X$ in $C$.
\end{prop}
\begin{defn}
  This set $Y$ is called the \emph{intersection} of $C$, denoted $\bigcap C$.
\end{defn}
\begin{defn}
  Let $A$ and $B$ be sets.
  The \emph{intersection} of $A$ and $B$, notated $A\cap B$, is
  $\bigcap\{A,B\}$.

  If $A\cap B=\emptyset$ then $A$ and $B$ are called \emph{disjoint}.
\end{defn}
\begin{axiom}[Powers]
  For each set $X$ there is a collection that contains all subsets of $X$.
\end{axiom}
\begin{prop}
  There is a unique collection $Y$ such that $x\in Y$ iff $x\subseteq X$.
\end{prop}
\begin{defn}
  This set $Y$ is called the \emph{power set} of $X$, denoted $\mathcal P(X)$.
\end{defn}
\begin{defn}
  The \emph{ordered pair} of $a$ and $b$ is the set defined as
  \[(a,b)=\{\{a\},\{a,b\}\}.\]
\end{defn}
\begin{prop}
  For any $a,b,c,d$, we have $(a,b)=(c,d)$ iff $a=c$ and $b=d$.
\end{prop}
\begin{prop}
  For any sets $A$ and $B$, the set
  \[\{(x,y): x\in A,\ y\in B\}\]
  exists.
\end{prop}
\begin{defn}
  This set is called the \emph{Cartesian product} of $A$ and $B$, denoted
  $A\times B$.
\end{defn}
\begin{prop}
  For any set $R$ of ordered pairs there are sets $A$ and $B$ such that
  $R\subseteq A\times B$.
\end{prop}
\begin{defn}
  A \emph{binary relation} $R$ from $A$ to $B$ is a subset of $A\times
  B$. If $(a,b)$ is in $R$ we write $aRb$.

  If $A=B$ then we call it a \emph{binary relation over} $A$.
\end{defn}
\begin{defn}
  An \emph{equivalence relation} is a binary relation $\sim$ over $A$ such
  that
  \begin{itemize}
    \item $a\sim a$ (reflexive),
    \item $a\sim b\iff b\sim a$ (symmetric), and
    \item if $a\sim b$ and $b\sim c$ then $a\sim c$ (transitive).
  \end{itemize}
  The \emph{equivalence class} of $a$ under $\sim$ is
  \[[a]=\{x\in A:x\sim a\}.\]
\end{defn}
\begin{defn}
  A \emph{partition} of a set $A$ is a disjoint collection of nonempty subsets
  of $A$ whose union is $A$.

  A partition $X$ of $A$ \emph{induces} a relation $A/X$, where $a\ A/X\ b$ iff
  $a$ and $b$ belong to the same element of $X$.
\end{defn}
\begin{prop}
  The collection of equivalence classes of an equivalence relation exists and 
  is a partition.
\end{prop}
\begin{defn}
  This partition is called the partition \emph{induced} by the equivalence
  relation $\sim$, denoted $X/\sim$.
\end{defn}
\begin{prop}
  The equivalence relation induced by a partition induces that partition; the
  partition induced by an equivalence relation induces that relation.
\end{prop}
\begin{defn}
    The \emph{natural projection} $\pi:X\to X/\sim$ sends every element to its
    equivalence class.
\end{defn}
\begin{defn}
  For a relation $R$ from $X$ to $Y$ we define the \emph{inverse} relation
  $R^{-1}:Y\to X$ by $xRy\iff yR^{-1}x$.
\end{defn}
\begin{defn}
  A \emph{function} $f:A\to B$ is a relation $f$ over $A$ and $B$ such that
  for each $a\in A$ there is exactly one $b\in B$ such that $afb$. We usually
  write this as $f(a)=b$.
\end{defn}
\begin{prop}
    The set of functions from $A$ to $B$ exists.
\end{prop}
\begin{defn}
    We denote it by $B^A$.
\end{defn}
\begin{prop}
    Let $F:X\to Y$ be a function and let $\sim$ be an equivalence relation on
    $X$. There is a function $G:X/\sim\to Y$ such that $F=G\pi$ iff the image of
    every equivalence class is a singleton.
\end{prop}
\begin{defn}
    In this case $G$ is the function \emph{induced} on $X/\sim$ by $F$.
\end{defn}
\begin{defn}
  The \emph{identity} function $I_A:A\to A$ is defined by $I_A(a)=a$ for each
  $a\in A$.
\end{defn}
\begin{defn}
  Let $f:A\to B$ and $g:B\to A$ be functions. If $f\circ g=I_A$, then $f$ is a
  \emph{left inverse} of $g$ and $g$ is a \emph{right inverse} of $f$. If both
  $f\circ g=I_B$ and $g\circ f=I_B$, then $f$ is called the \emph{two-sided
  inverse} or simply \emph{inverse} of $g$.
\end{defn}
\begin{defn}
  For a set $E\subseteq A$, we define the \emph{image} of $E$ under $f$ as
  $f(E)=\{f(x):x\in E\}$. For a set $E\subseteq B$, we define the \emph{inverse
    image} of $E$ under $F$ as $f^{-1}(E)=\{x\in A:f(x)\in E\}$.
\end{defn}
\begin{defn}
  A function $f$ is \emph{injective} if for each $b$ in $B$, there is at most one
  $a$ in $A$ such that $f(a)=b$. It is \emph{surjective} if for each $b$ in
  $B$ there is at least one $a$ in $A$ such that $f(a)=b$. A function which is
  both injective and surjective is \emph{bijective}.
\end{defn}
\begin{prop}
  A function whose domain is nonempty is injective iff it has a left inverse.
\end{prop}
\begin{prop}
  A function is bijective iff it has an inverse, which equals any left- or
  right-inverse of the function.
\end{prop}
\begin{defn}
  If $A\subseteq B$ and $f:B\to C$, the \emph{restriction} of $f$ to $A$ is
  $f|_A:A\to C,\ f|_A(x)=f(x)$.
\end{defn}
\begin{defn}
  For functions $f:W\to X$ and $g:Y\to Z$, where $Y\subseteq X$, we define the
  \emph{composite} $f\circ g:W\to Z$ as $(f\circ g)(x)=f(g(x))$ for all $x$.
\end{defn}
\begin{defn}
  A function $x$ from a set $I$ (the \emph{index set}) to a set $X$ is called an
  \emph{indexed family} of $X$, and its range is an \emph{indexed set}. We
  notate the indexed set by $\{x_i\}_{i\in I}$.
\end{defn}
\begin{defn}
  For any set $X$ we define $X^+=X\cup\{X\}$.
\end{defn}
\begin{axiom}[Infinity]
  There exists a set $S$ containing $\emptyset$ and containing $X^+$ for
  every $X$ in $S$.
\end{axiom}
\begin{prop}[Peano Axioms]
  There exists a unique set $\omega$ satisfying
  \begin{itemize}
    \item $\emptyset\in\omega$.
    \item If $n\in\omega$ then $n^+\in\omega$.
    \item If $S\subseteq\omega$ such that $\emptyset\in S$ and $n\in S\implies
      n^+\in S$ then $S=\omega$.
    \item $n^+\ne 0$ for all $n\in \omega$.
    \item If $n$ and $m$ are in $\omega$, and if $n^+=m^+$, then $n=m$.
  \end{itemize}
\end{prop}
\begin{thm}[Recursion]
  If $a$ is an element of a set $X$, and if $f:X\to X$ is a function, then there
  is a function $g:\omega\to X$ such that $u(0)=a$ and $u(n^+)=f(u(n))$ for all
  $n$ in $\omega$.
\end{thm}
\begin{prop}
  The set $S^n$, defined by $S^1=S$ and $S^{n+1}=S^n\times S$, exists for each
  $n\in\omega\setminus\{\emptyset\}$.
\end{prop}
\begin{defn}
  A \emph{partial order} is a binary relation $\le$ on a a set $A$ such that
  \begin{itemize}
    \item $a\le a$ (reflexive),
    \item if $a\le b$ and $b\le a$ then $a=b$ (antisymmetric), and
    \item if $a\le b$ and $b\le c$ then $a\le c$ (transitive).
  \end{itemize}
  We define $a<b$ if $a\le b$ and $a\ne b$.

  If for all $a$ and $b$ we have $a\le b$ or $b\le a$ (strongly connected),
  then $\le$ is a \emph{total order}. 

  A \emph{chain} is a totally ordered subset of a partially ordered set.
\end{defn}
\begin{defn}
  If $X$ is a partially ordered set, and if $a\in X$, the set $s(a)=\{x\in
    X:x<a\}$ is called the \emph{initial segment} determined by $a$.
\end{defn}
\begin{defn}
  Two partially ordered sets $X$ and $Y$ are \emph{similar} if there is a
  bijection $f:X\to Y$ such that $a\le b\iff f(a)\le f(b)$. This bijection is
  called a \emph{similarity}.
\end{defn}
\begin{defn}
  Let $S$ be a subset of a partially ordered set $A$, and let $a$ be an element
  of $A$. If $s\le a$ for every $s$ in $S$, then we call $a$ an \emph{upper
    bound} of $S$. If $a\le s$ for every $s$ in $S$, then we call $a$ a
    \emph{lower bound} of $S$. If $a$ is an upper bound of $S$ and a lower
    bound of the set of upper bounds of $S$, then we call $a$ a \emph{least
      upper bound} of $S$.
\end{defn}
\begin{defn}
  A \emph{well-order} on $A$ is a total order $\le$ on $A$ such that every
  nonempty subset $S$ of $A$ has an element $a$ which is a lower bound for $S$.
  The set $A$ together with the relation $\le$ is then called \emph{well-ordered}.
\end{defn}
\begin{thm}[Transfinite Induction]
  Let $S$ be a subset of a well-ordered set $A$ such that for any $x\in A$, if
  $s(x)\subseteq S$ then $x\in S$.
  Then $S=A$.
\end{thm}
\begin{defn}
  If $a$ is an element of a well-ordered set $A$, and $X$ is an arbitrary set,
  then a \emph{sequence of type} $a$ is an family of $X$ indexed by $s(a)$.

  A \emph{sequence function} of type $A$ is a function whose domain consists of
  all sequences of type $a$ for each $a\in A$, and whose codomain is $A$.
\end{defn}
\begin{prop}[Transfinite Recursion]
  If $A$ is a well-ordered set, and if $f$ is a sequence function of type $A$ in
  $X$, then there is a unique function $U:A\to X$ such that $U(a)=f(U|s(a))$ for
  each $a$ in $W$.
\end{prop}
\begin{prop}
  If two well-ordered sets are similar, then the similarity is unique.
\end{prop}
\begin{thm}
  If $X$ and $Y$ are well-ordered, then either $X$ and $Y$ are similar, or one
  is similar to an initial segment of the other.
\end{thm}
\begin{defn}
  An \emph{ordinal number} is a well-ordered set $\alpha$ such that for any
  $\xi\in\alpha$ we have $s(\xi)=\xi$.

  We define the ordinals $0=\emptyset$ and $1=0^+$.
\end{defn}
\begin{prop}
  There is no set of all ordinal numbers.
\end{prop}
\begin{prop}
  $\omega$ is an ordinal number.
\end{prop}
\begin{prop}
  If $\alpha$ is an ordinal number then so is $\alpha^+$, and so is any element
  of $\alpha$.
\end{prop}
\begin{prop}
    If $\alpha$ is an ordinal number, then either $\alpha=(\bigcup\alpha)^+$ or
    $\alpha=\bigcup\alpha$.
\end{prop}
\begin{defn}
    In the first case, $\alpha$ is a \emph{successor ordinal}; in the second, it
    is a \emph{limit ordinal}.
\end{defn}
\begin{thm}
  If two ordinal numbers are similar, then they are equal.

  Otherwise, one is an element of the other.
\end{thm}
\begin{axiom}[Substitution]
  If $p$ is a set and 
  $S(a,b,p)$ is a sentence such that for each $a$ in a set $A$ there exists a
  set $B_a$ such that $b\in B_a\iff S(a,b,p)$, then there exists a function $F$
  with domain $A$ such that $F(a)=B_a$ for each $a$ in $A$.
\end{axiom}
\begin{axiom}[Foundation]
  Every set $X$ contains a set $Y$ such that $X$ and $Y$ are disjoint.
\end{axiom}
\begin{axiom}[Choice]
  Let $X$ be a collection of sets whose members are all nonempty. Then there exists a
  function $f:X\to\bigcup X$ such that $f(Y)\in Y$ for all $Y\in X$.
\end{axiom}
\begin{prop}
  Every relation includes a function with the same domain.
\end{prop}
\begin{thm}[Zorn's Lemma]
  Suppose a partially ordered set $P$ has the property that every chain in $P$
  has an upper bound in $P$. Then there is an element $a\in P$ such that the
  only upper bound for $\{a\}$ is $a$.
\end{thm}
\begin{thm}[Well-Ordering Theorem]
  Every set has a well-ordering.
\end{thm}
\begin{prop}
  Every well-ordered set is similar to a unique ordinal number.
\end{prop}
\begin{prop}
    If $S$ is an ordinal and $A$ is a family of ordinals indexed by $A$,
    consider the set $T$ of ordered pairs $(s,a)$ such that $s\in S$ and $a\in
    A_s$. We define the relation $(s_1,a_1)\le(s_2,a_2)$ if $s_1<s_2$
    or $s_1=s_2$ and $a_1\le a_2$. This relation well-orders $T$.
\end{prop}
\begin{defn}
  The ordinal corresponding to $T$ under this well-ordering is the
  \emph{ordinal sum} of $A$, denoted $\sum A$.
\end{defn}
\begin{prop}
    For any pair of ordinals $(a,b)$ with $a\le b$ there is an ordinal $c$ such
    that $a+c=b$.
\end{prop}
\begin{cor}
    If $a<b$ then for any $c$ we have $c+a<c+b$ and $a+c\le b+c$.
\end{cor}
\begin{prop}
  If $A$ and $B$ are ordinals, the ordering on $A\times B$ where
  $(a,b)\le (c,d)$ iff $b<d$ or both $b=d$ and $a\le c$
  is a well-ordering on $A\times B$.
\end{prop}
\begin{defn}
  The ordinal corresponding to $A\times B$ under this well-ordering is the
  \emph{ordinal product} of $A$ and $B$, denoted $AB$ or $A\cdot B$.
\end{defn}
\begin{prop}
    If $a<b$ then for any $c$ we have $ca<cb$ and $ac\le bc$.
\end{prop}
\begin{prop}
    For every family $\{a_i\}$ of ordinals indexed by an ordinal $b$,
    there exists an ordinal $c$ and a unique function $f:b^+\to c$ such that
    $f(\emptyset)=1$ and 
    \[f(x)=\begin{cases} f(\bigcup x) a_x&\bigcup x\ne x \\ \bigcup_{y\in
    x}f(y)&\bigcup x=x\end{cases}.\]
    The graph of $f$ is the same no matter which ordinal $c$ is used.
\end{prop}
\begin{defn}
  We define $\prod a_i=f(I)$. If all $a_i$ equal $a$, then we define $a^b=f(b)$.
\end{defn}
\begin{prop}
    If $a\le b$, then for any $c$ we have $a^c\le b^c$. If additionally
    $c>1$, then $c^a\le c^b$.
\end{prop}
\begin{prop}
  With ordinal sums, products and exponents as defined,
  \begin{align*}
    a+0=0+a=a \\
    a+1=a^+ \\
    a+(b+c)=(a+b)+c \\
    a(bc)=(ab)c \\
    a\sum B_i=\sum a{B_i} \\
    a^{\sum B_i}=\prod a^{B_i} \\
    a^{bc}=(a^b)^c.
  \end{align*}
  However, ordinal addition and multiplication are not commutative and not
  right-distributive. Also, $(ab)^c$ is generally distinct from $a^c b^c$.
\end{prop}
\begin{prop}
    For $c>1$ and $a\ge 1$, we have $c^a\ge a$.
\end{prop}
\begin{defn}
  Two sets $A$ and $B$ are said to have the same \emph{cardinality} (written
  $|A|=|B|$) if there is a bijection $f:A\to B$.

  A set $A$ has cardinality at most the cardinality of $B$ ($|A|\le|B|$) if
  there is an injection $f:A\to B$.

  A set $A$ has cardinality less than the cardinality of $B$ ($|A|<|B|$) if
  $|A|\le|B|$ and $|A|\ne|B|$.

  A set $A$ is \emph{enumerable} if $|A|=|\omega$, \emph{countable} if it is
  finite or enumerable, and \emph{uncountable} otherwise.
\end{defn}
\begin{prop}
  If there exists a surjection $f:A\to B$ then $|B|\le|A|$.
\end{prop}
\begin{thm}[Schr\"oder-Bernstein]
  If $|A|\le|B|$ and $|B|\le|A|$ then $|A|=|B|$.
\end{thm}
\begin{thm}[Cantor]
  For any set $A$, $|\mathcal P(A)|>|A|$.
\end{thm}
\begin{defn}
    A set $S$ is \emph{dense} if for any $a\in S$, the least upper bound of 
    $s(a)$ is $a$. It is \emph{unbordered} if it has no least upper bound or
    greatest lower bound.
\end{defn}
\begin{prop}
    All enumerable unbordered dense totally ordered sets are similar.
\end{prop}
\begin{prop}
    If $A$ and $B$ are collections of disjoint sets and $f:A\to B$ is a
    bijection such that $|f(a)|=|a|$ for each $a\in A$, then
    $\left|\bigcup A\right|=\left|\bigcup
    B\right|$, and $\left|\prod A\right|=\left|\prod B\right|$.
\end{prop}
\begin{prop}
    For any set $C$ and any indexed family of sets $A$ we have
    \[\left|\prod C^{A_i}\right|=\left|C^{\bigcup A}\right|.\]
\end{prop}
\begin{defn}
    We define $\sum|A_i|=\left|\bigcup A_i\right|$, $\prod|A_i|=\left|\prod
    A_i\right|$, and $|A|^{|B|}=|A^B|$.
\end{defn}
\begin{prop}
    For every set of cardinal numbers there is a cardinal number strictly
    greater than all of them.
\end{prop}
\begin{thm}[K\"onig]
    Let $\{A_i\}$ and $\{B_i\}$ be indexed families of disjoint sets, such that
    for each $i$ we have $|A_i|<|B_i|$. Then, $\left|\bigcup
    A_i\right|<\left|\bigcup B_i\right|$.
\end{thm}
\begin{defn}
  A \emph{cardinal number} is an ordinal number $\alpha$ such that for any
  ordinal number $\beta$ with $|\alpha|=|\beta|$ we have $\alpha\subseteq\beta$.
\end{defn}
\begin{prop}
    If $a$ and $b$ are ordinals, then $|a+b|=|a|+|b|$, $|ab|=|a||b|$ and
    $|a^b|=|a|^{|b|}$. 
    Here, ordinal operations are used on the left side and cardinal operations are
    used on the right.
\end{prop}
\begin{prop}
  Every element of $\omega$, as well as $\omega$ itself, is a cardinal number.
\end{prop}
\begin{prop}
  For any set $S$, there is a unique cardinal number $\alpha$ with
  $|\alpha|=|S|$.
\end{prop}
\begin{defn}
  For these sets $S$ and $\alpha$ we define $|S|=\alpha$.
\end{defn}
\begin{defn}
  A set $A$ is said to be \emph{finite} if $|A|\in\omega$, and \emph{infinite}
  otherwise.
\end{defn}
\begin{prop}
  A set is infinite if and only if it has the same cardinality as some proper
  subset.
\end{prop}
\begin{prop}
  A countable set does not have any uncountable subsets. An uncountable set has
  a subset with cardinality equal to $\omega$.
\end{prop}
\begin{prop}
  A union of countably many countable sets is countable.
\end{prop}
\begin{prop}
  If $a$ and $b$ are cardinal numbers such that $a\ge\omega$ and $a\ge b$,
  then $a+b=a\times b=a$. If $b$ is finite we also have $a^b=a$.
\end{prop}
\begin{cor}
    If $b$ is infinite and $a=c^b$ for some $c$, then $a^b=a$.
\end{cor}
\begin{prop}
    Let $\beta>1$ be an arbitrary ordinal. Every ordinal
    can be represented uniquely as a finite sum
    $\sum_i \beta^{\alpha_i}\gamma_i$, where all $\alpha_i$ are distinct and
    each $\gamma_i$ is smaller than $\beta$.
\end{prop}
\begin{defn}
    For each infinite cardinal $a$, consider the set $c(a)$ of all infinite
    cardinals strictly smaller than $a$. Since $c(a)$ is well-ordered, it is
    similar to some ordinal $\alpha$; we write $a=\aleph_\alpha$.
\end{defn}
\begin{prop}
    The set of ordinals with cardinality $\aleph_\alpha$ has cardinality
    $\aleph_{\alpha+1}$.
\end{prop}
\subsection*{References}
\begin{itemize}
  \item Halmos, \emph{Naive Set Theory}
  \item Kamke, \emph{Theory of Sets}
\end{itemize}

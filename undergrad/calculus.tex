\section{Calculus}
\begin{defn}
    Let $f:E\to Y$ be a function. If there exists a linear map $A:X\to Y$ such that
    \[\lim_{h\to 0}\frac{\|f(x+h)-f(x)-Ah\|}{\|h\|}=0,\]
    then $f$ is said to be \emph{differentiable at} $x$, $A$ is said to be the
    \emph{derivative} of $f$ at $x$, and we write $f'(x)=A$ or $DF_x=A$.
    If $f$ is differentiable at every $x\in E$, we say that $f$ is
    \emph{differentiable}.
\end{defn}
\begin{prop}
    If $f$ is differentiable at $x$, then $f'(x)$ is uniquely defined.
\end{prop}
\begin{prop}
    If $A$ is linear and $f(x)=Ax$, then $f'(x)=A$.
\end{prop}
\begin{prop}[Chain Rule]
    Let $f:E\to F$ and $g:F\to G$ be such that $f$ is differentiable at $x$ and $g$
    is differentiable at $f(x)$. Then \[(g\circ f)'(x)=g'(f(x))f'(x).\]
\end{prop}
\begin{prop}[Sum Rule]
    $(f+g)'=f'+g'$.
\end{prop}
\begin{prop}
    If $A$ is bilinear, then
    \[A(f(x),g(x))'(x)h=A(f'(x)h,g(x))+A(f(x),g'(x)h).\]
\end{prop}
\begin{cor}[Product and Quotient Rule]
    If the codomain of $g$ is $\Rr$, then
    \begin{itemize}
        \item $(fg)'=f'g+fg'$.
        \item $(f/g)'=\frac{f'g-fg'}{g^2}$ (if $g\ne 0$).
    \end{itemize}
\end{cor}
\begin{prop}
    We have:
    \begin{itemize}
        \item $\exp'=\exp$
        \item $\ln'(x)=\frac 1x$
        \item $\sin'=\cos$
        \item $\cos'=-\sin$
        \item $\tan'(x)=1+\tan^2(x)$
        \item $\arcsin'(x)=\frac 1{\sqrt{1-x^2}}$
        \item $\arccos'(x)=\frac{-1}{\sqrt{1-x^2}}$
        \item $\arctan'(x)=\frac 1{1+x^2}$
    \end{itemize}
\end{prop}
\begin{prop}
    Suppose $\{f_n'\}$ converges uniformly and $\{f_n(x_0)\}$ converges. Then
    \[\left(\lim_{n\to\infty} f_n\right)'=\lim_{n\to\infty} f_n'.\]
\end{prop}
\begin{prop}
    Let $f:[a,b]\to\Rr$ be differentiable. Then the image of $f'$ contains
    $[f'(a),f'(b)]$.
\end{prop}
\begin{defn}
    The \emph{directional derivative} $D_u f$ is defined as
    \[D_u f(x)=\lim_{t\to 0}\frac{f(x+tu)-f(x)}{t}.\]
\end{defn}
\begin{prop}
    The derivative $Df$ exists and is continuous iff $D_u f$ exists and is
    continuous for every $u$. In this case we have $D_u f(x)=f'(x)u$.
\end{prop}
\begin{prop}[Mean Value Theorem]
    Let $f:X\to\Ff$, and let $x,y\in X$ such that the line
    segment joining $x$ and $y$ lies in $X$ and $D_{y-x}f$ exists everywhere on
    that line. Then there exists a point $c$ on
    this line segment such that $D_{y-x}(c)=\frac{f(y)-f(x)}{\|y-x\|}$.
\end{prop}
\begin{prop}
    Suppose $X$ is convex and $\|f'(x)\|\le M$ for all $x\in X$. Then, for $a,b\in E$ we have
    $\|f(b)-f(a)\|\le M\|b-a\|$.
\end{prop}
\begin{prop}[L'H\^opital's Rule]
    Let $a$ be a limit point of an open subset $L$ of the extended real number
    system, let $f,g:L\to\Rr$ be differentiable, and suppose $g'\ne 0$ on
    $L$. If $\frac{f'(x)}{g'(x)}\to A$ as $x\to a$, and either
    \begin{itemize}
        \item $f(x),g(x)\to 0$ as $x\to a$, or
        \item $g(x)\to\pm\infty$ as $x\to a$,
    \end{itemize}
    then $\frac{f(x)}{g(x)}\to A$ as $x\to a$.
\end{prop}
\begin{defn}
    A continuous function is $C^0$. If the derivative of a function exists and
    is $C^n$, then the function is $C^{n+1}$. If a function is $C^n$ for all
    $n$, it is $C^\infty$.
\end{defn}
\begin{defn}
    We use the notation $d^n f_x$ for the $n$th order derivative of $f$ at $x$,
    defining $d^0 f_x=x$.
\end{defn}
\begin{prop}
    If $f$ and $g$ are $C^n$, then so are $f+g$, $f\cdot g$ and $f/g$ (assuming
    the codomains match).
\end{prop}
\begin{prop}
    If $D_v f$, $D_u f$ and $D_v D_u f$ exist in an open ball containing $x$,
    and if $D_v D_u f$ is continuous at $x$, then $D_u D_v f(x)=D_v D_u f(x)$.
\end{prop}
\begin{cor}
    If $f$ is $C^n$, then $f^{(n)}(x)$ (the $n$th derivative) is a symmetric
    multilinear map.
\end{cor}
\begin{defn}
    An open set $D$ is \emph{star-shaped} with respect to $A\in D$ if
    for every $B\in D$ the segment $AB$ is in $D$.
\end{defn}
\begin{thm}[Taylor]
    Let $D$ be star-shaped with respect to $A$, and let $f:D\to V$ be $n$-times
    differentiable. Let \[R_{n+1}(x)=f(a+x)-\sum_{i=0}^n \frac{d^i
    f_a(x,x,\ldots,x)}{i!}.\]

    We have \[\lim_{x\to a}\frac{R_{n+1}(x)}{\|x-a\|^n}=0.\] 

    If $f$ is $(n+1)$-times differentiable and its codomain is $\Rr$,
    we have the following expressions for the remainder:
    \begin{itemize}
        \item There is a $t\in[0,1]$ such that
            $R(x)=\frac{d^{n+1}f_a(tx)}{(n+1)!}$.
        \item If $g(t)=f(a+(x-a)t)$, then
            \[R(x)=\frac 1{n!}\int_0^1 (1-t)^n g^{(n+1)}(t)\mathrm dt.\]
    \end{itemize}
\end{thm}
\begin{prop}[Second derivative test]
    Let $f:X\to\Rr$ be $C^2$, where $X\subseteq\Rr^n$.
    Let $a$ be a point of $f$ such that $f'(a)=0$.
    We define the \emph{Hessian matrix} $H_f(a)$ to be the matrix representation
    of $f''(a)$ using the standard bases for $\Rr^n$ and its dual space.
    \begin{itemize}
        \item If $H_f(a)$ is positive (resp.\ negative) definite, then $f$ has a
            local minimum (resp.\ maximum) at $a$.
        \item If $f$ has a local minimum (resp.\ maximum) at $a$, then $H_f(a)$
            is positive (resp.\ negative) semidefinite.
    \end{itemize}
\end{prop}
\begin{defn}
    A collection $\{g_i:M\to[0,1]\}$ of $C^\infty$ functions is a \emph{partition of
    unity} if
    \begin{itemize}
        \item each $g_i$ has compact support,
        \item each $x\in M$ has a neighbourhood $V_x$ such that all but finitely
            many $g_i$ are $0$ on $V_x$, and
        \item $\sum g_i=1$ everywhere on $M$.
    \end{itemize}

    A partition of unity $\{g_i\}$ is \emph{subordinate} to an open cover
    $\{U_j\}$ of $m$ if for every $i$ the support of $g_i$ is contained in some
    $U_j$.
\end{defn}
\begin{thm}
    Let $\{U_i\}$ be an open covering of $M$. There exists a partition of unity
    $\{g_i\}$ subordinate to $\{U_i\}$.
\end{thm}
\begin{defn}
    A \emph{$k$-cell} $S$ is a product of $k$ closed intervals. Its \emph{volume} is
    $v(S)$, the product of the lengths of the intervals.
\end{defn}
\begin{defn}
    A \emph{partition} of a closed interval $[a,b]$ is a sequence
    $t_0,\ldots,t_k$, where $a=t_0\le t_1\ldots\cdots\ldots t_k=b$.

    A \emph{partition} of a $k$-cell
    is a sequence of $k$ partitions $\{P_i\}$, where each $P_i$ is
    a partition of the corresponding $[a_i,b_i]$.
    This partition divides the $k$-cell into a collection of \emph{subcells}.
\end{defn}
\begin{defn}
    For a partition $P$ and a bounded function $f$ we define the \emph{lower} and
    \emph{upper sums} as
    \[L(f,P)=\sum_S\left(v(S)\inf_{x\in S}f(S)\right),\
    U(f,P)=\sum_S\left(v(S)\sup_{x\in S}f(S)\right).\]
\end{defn}
\begin{prop}
    If $P_1$ and $P_2$ are two partitions of the same $k$-cell, then
    $L(f,P_1)\le U(f,P_2)$.
\end{prop}
\begin{defn}
    Let $A$ be a $k$-cell. A function $f:A\to\Rr$ is called \emph{integrable}
    over $A$ if
    $f$ is bounded and $\sup\{L(f,P)\}=\inf\{U(f,P)\}$. In that case, their
    common value is the \emph{integral} of $f$ over $A$, denoted
    \[\int_A f\mathrm dV.\]
\end{defn}
\begin{prop}
    A bounded function $f$ is integrable over $A$ iff for all $\varepsilon>0$ there is a
    partition $P$ of $A$ such that $U(f,P)-L(f,P)\le\varepsilon$.
\end{prop}
\begin{defn}
    A subset $A$ of $\Rr^k$ has \emph{measure} $0$ if for every $\varepsilon>0$
    there is a cover of $A$ by $k$-cells with total volume less than
    $\varepsilon$.
\end{defn}
\begin{thm}[Sard]
    Let $g:A\to\Rr^n$ be continuously differentiable, where $A$ is open. Then
    the subset of $A$ on which $\det g'=0$ has measure $0$.
\end{thm}
\begin{thm}
    A bounded function is integrable over a $k$-cell iff its set of
    discontinuities in the $k$-cell has measure $0$.
\end{thm}
\begin{defn}
    The \emph{support} of a function $f$ is the closure of the set of points at
    which $f$ is nonzero.
\end{defn}
\begin{prop}
    If $f$ has compact support and is integrable over some $k$-cell containing
    its support,
    then for any $k$-cells $A$ and $B$ which contain
    its support we have $\int_A f(x)\mathrm dV=\int_B f(x)\mathrm dV$.
\end{prop}
\begin{defn}
    We define the \emph{characteristic function}
    $\chi_C(x)$ to be $1$ for any $x\in C$, and $0$ elsewhere.
\end{defn}
\begin{cor}
    If $A$ is a $k$-cell containing $C$, and if
    $\int_A\chi_C(x)f(x)\mathrm dV$ exists, then $\int_B\chi_C(x)f(x)\mathrm dV$
    exists and equals this value for any $k$-cell $B$ containing $C$.
\end{cor}
\begin{defn}
    We define $\int_C f(x)\mathrm dV$, to be this value, if it exists. In the
    case where $C$ is an interval $[a,b]$, we also write this as $\int_a^b
    f(x)\mathrm dV$.
\end{defn}
\begin{prop}
    If $f$ is integrable over a $k$-cell $A$, $C$ is a subset of $A$ and the
    boundary of $C$ has measure $0$, then $f$ is integrable over $C$.
\end{prop}
\begin{prop}
    Suppose $\{f_n\}$ converges uniformly and each $f_n$ is integrable over $C$. Then,
    \[\int_C\lim_{n\to\infty} f_n(x)\mathrm dx=\lim_{n\to\infty}\int_C
    f_n(x)\mathrm dx.\]
\end{prop}
\begin{thm}[Fundamental Theorem of Calculus]
    If $f:[a,b]\to\Rr$ is integrable, define
    \[F(x)=\int_a^x f(t)\mathrm dt.\]
    If $f$ is continuous at $c\in[a,b]$, then $F$ is differentiable at $c$, and
    $F'(c)=f(c)$.

    If $F:[a,b]\to\Rr$ is differentiable, define $f(x)=F'(x)$. If $f$ is
    integrable on $[a,b]$, then \[F(x)=F(a)+\int_a^x f(t)\mathrm dt.\]
\end{thm}
\begin{thm}[Integration by Parts]
    Let $F$ and $G$ be differentiable on $[a,b]$ such that $F'=f$ and $G'=g$ are
    integrable. Then,
    \[\int_a^b F(x)g(x)\mathrm dx=F(b)G(b)-F(a)G(a)-\int_a^b f(x)G(x)\mathrm
    dx.\]
\end{thm}
\begin{thm}[Fubini]
    Let $A$ and $B$ be $k$-cells, and let $f:A\times B\to\Rr$ be such that
    $f(x,b)$ is integrable for each $b\in B$ and $f(a,x)$ is integrable for each
    $a\in A$. Then, $f$ is integrable over $A\times B$ and
    \[\int_{A\times B}f\mathrm dV=\int_A\left(\int_B f\mathrm dV\right)\mathrm
    dV=\int_B\left(\int_A f\mathrm dV\right)\mathrm dV.\]
\end{thm}
\begin{thm}[Change of Variables]
    Let $A$ be open in $\Rr^n$, and let $g:A\to\Rr^n$ be $C^1$ and bijective.
    If $f:g(A)\to\Rr$ is integrable, then
    \[\int_{g(A)}f\mathrm dV=\int_A(f\circ g)|\det g'|.\]
\end{thm}
\begin{thm}[Differentiation under the Integral]
    Let $f:[a,b]\times[c,d]$ be such that $f(\cdot,t)$
    is integrable for all $t$, and $D_{(0,1)}f$ is continuous. Then, 
    \[f'(s)=\int_a^b D_{(0,1)}f(x,s)\mathrm dx\]
    for each $s\in(c,d)$.
\end{thm}
\begin{prop}
    We have:
    \begin{itemize}
        \item $\ln(1+x)=\sum (-1)^i\frac{x^{i+1}}{i+1}$ for $-1<x\le 1$
        \item $\frac12\ln\left(\frac{1+x}{1-x}\right)=\sum
            \frac{x^{2i+1}}{2i+1}$ for $-1<x<1$
        \item $\arctan(x)=\sum \frac{(-1)^i x^{2i+1}}{2i+1}$ for $-1\le x\le 1$
    \end{itemize}
\end{prop}
\begin{defn}
    Let $V$ be a normed vector space over $\Rr$. A \emph{curve} is a map $f:[a,b]\to
    V$. We associate to each partition $P=\{x_i\}$ of $[a,b]$ the number
    $L(P)=\sum\|\gamma(x_i)-\gamma(x_{i-1})\|$. If these numbers have a
    supremum, then this supremum is the \emph{length} of $\gamma$ and $\gamma$
    is \emph{rectifiable}. We say that $\gamma$ is \emph{piecewise $C^n$} if
    its domain can be divided into a finite number of intervals such that
    $\gamma$ is $C^n$ on each interval.
\end{defn}
\begin{prop}
    If $\gamma$ is piecewise $C^1$, then $\gamma$ is rectifiable. For each
    $x\in[a,b]$, the length of $\gamma([a,x])$ is
    \[s(x)=\int_a^x\|\gamma'(t)\|\mathrm dt.\]
\end{prop}
\begin{prop}
    There is a function $F$
    such that $\gamma(x)=F(s(x))$ for each $x$. If $F$ is differentiable, then
    $\|F'\|=1$ everywhere.
\end{prop}
\begin{defn}
    The vector $F'(s)$ is called the \emph{unit tangent vector} to $\gamma$.
\end{defn}
\begin{defn}
    Let $D$ be an open subset of an inner product space $V$, and let $F:D\to V$
    be continuous. Let $C:[a,b]\to D$ be piecewise $C^1$.
    The \emph{line integral} of $F$ along $C$ is
    \[\int_C F\cdot\mathrm ds=\int_c^d F(C(t))\cdot C'(t)\mathrm dt.\]
\end{defn}
\begin{prop}
    The line integral is independent of parametrisation.
\end{prop}
\begin{prop}
    Let $D$ be an open subset of $V$, where $V$ is an inner product space over $\Rr$.
    If $f:D\to\Rr$ is differentiable, then for each $x$ there is a vector $\nabla f(x)$ such
    that $f'(x)(y)=\nabla f\cdot y$ for all $y$.
\end{prop}
\begin{defn}
    This vector $\nabla f$ is called the \emph{gradient} of $f$.
\end{defn}
\begin{prop}
    Let $E$ be open in $X$, and let $G:E\to Y$ be $C^1$. Let $A=G^{-1}(0)$.
    Assume $G'(a)$ is surjective for all $a\in A$.
    If $f:E\to\Rr$ is differentiable and the maximum of $f$ on $A$ occurs at $a$,
    then there is a functional $l\in Y^*$ such that $f'(a)=lG'(a)$.
\end{prop}
\begin{cor}[Lagrange Multipliers]
    If the codomain of $G$ is $\Rr$,
    then for some $\lambda$ we have $\nabla f=\lambda\nabla g$.
\end{cor}
\begin{thm}[Fundamental Theorem of Line Integrals]
    Let $f:D\to\Rr$ be $C^1$, and let $\gamma:[a,b]\to D$ be a piecewise $C^1$
    curve such that $\gamma(a)=X_0$ and $\gamma(b)=X_1$. Then,
    \[\int_C\nabla f\cdot\mathrm ds=f(X_1)-f(X_0).\]
\end{thm}
\begin{defn}
    A vector field $F:D\to V$ is \emph{conservative} if $\int_C F\cdot\mathrm ds=0$
    whenever $C$ is a closed rectifiable curve.
\end{defn}
\begin{cor}
    A vector field over a connected open set is conservative iff it is the
    gradient of some function.
\end{cor}
\begin{thm}[Inverse Function]
    Let $f:X\to X$, and let $a$ be such that $f$ is $C^k$ in an open ball
    containing $a$ and $f'(a)$ is invertible.
    Then there is an open ball $E$ containing $a$ and such that $f$ is injective on
    $E$, $F(E)$ is open, and $(F|_E)^{-1}$ is $C^k$.
\end{thm}
\begin{defn}
    A \emph{homeomorphism} between two topological spaces is a continuous
    function with continuous inverse.
\end{defn}
\begin{cor}
    If $f:X\to X$ is differentiable and $|f'|$ is always nonzero, then $f$ is a
    homeomorphism.
\end{cor}
\begin{thm}[Implicit Function]
    Let $\phi:X\times Y\to Y$ be $C^k$ in an open set containing $(x,y)$, and
    assume $\phi(x,y)=0$. Let $\phi'(x,y)=A(x)+B(y)$, and assume $B$ is invertible.
    Then there is an open set $E$ containing $x$ and an open set $F$ containing
    $y$ such that for each $e\in E$ there is a unique $f\in F$ such that
    $\phi(e,f)=0$. The function $e\mapsto f$ is $C^k$.
\end{thm}
\begin{defn}
    The \emph{half-space} $\Hh^k$ is defined as $\Rr^{k-1}\times\Rr_{\ge 0}$.
\end{defn}
\begin{defn}
    Let $U$ and $V$ be normed vector spaces.
    Let $f:S\to V$, where $S\subseteq U$. We say that $f$ is $C^r$ on $S$ if
    there is an open set $W$ containing $U$ and a function $g:W\to V$ such that
    the restriction of $g$ to $S$ is $f$.
\end{defn}
\begin{prop}
    Let $U$ be open in $\Hh^k$, and let $\alpha:U\to V$ be $C^r$. If $\beta_1$
    and $\beta_2$ are extensions of $\alpha$ to open sets in $\Rr^k$ containing
    $U$, then $\beta_1'=\beta_2'$ on $U$.
\end{prop}
\begin{defn}
    Thus, we define $\alpha'=\beta'$ on $U$ for any such extension.
\end{defn}
\begin{defn}
    A subset $M$ of $\Rr^n$ is a $k$-dimensional \emph{manifold with boundary}
    if for each $p\in M$, there is an open subset $V$ of $M$ containing $p$, a
    set $U$ that is open in $\Hh^k$, and a $C^\infty$ homeomorphism $\alpha:U\to
    V$ such that $\alpha'$ has rank $k$ everywhere. Such an $\alpha$ is a
    \emph{patch} around $p$. It is a \emph{manifold} if these
    patches can be chosen such that each $U$ is open in $\Rr^k$.
\end{defn}
\begin{rem}
    In general, one may drop the smoothness condition for $\alpha$ and the
    full-rank condition for $\alpha'$. However, these conditions make doing
    calculus on manifolds more convenient.
\end{rem}
\begin{prop}
    If $(U_1,h_1)$, $(U_2,h_2)$ and $(U_1\cap U_2,h_3)$ are $n$-dimensional
    charts, then $h_1(U_1\cap U_2)$ and $h_2(U_1\cap U_2)$ are
    open in $\Hh^n$ and the function $h_1\circ h_2^{-1}$ is continuous.
\end{prop}
\begin{defn}
    A point $p$ in a $k$-manifold with boundary $M$ is said to lie on the
    \emph{boundary} $\partial M$ of $M$ if no subset of $M$ containing $p$ is a
    $k$-manifold, and in the \emph{interior} $M-\partial M$ otherwise.
\end{defn}
\begin{prop}
    If $M$ is a $k$-manifold with boundary and $\partial M$ is nonempty, then
    $\partial M$ is a $(k-1)$-manifold.
\end{prop}
\begin{thm}[Rank Theorem]
    Let $A$ be an open set in $V$, let $r<\dim W$ be an integer, and let
    $F:A\to W$ be $C^\infty$ such that the rank of $F'$ is $r$ at every point in $A$.
    Then $F(A)$ is an $r$-dimensional smooth manifold in $W$.
\end{thm}
\begin{defn}
    Given $x\in V$, we define a \emph{tangent vector} at $x$ to be a pair
    $(x,v)$ where $v\in V$. The \emph{tangent space} to $V$ at $x$, denoted
    $\mathcal T_x(V)$, is the set of
    all tangent vectors, with the operations $(x,v_1)+(x,v_2)=(x,v_1+v_2)$ and
    $a(x,v)=(x,av)$. Given an inner product and an orientation of $V$, we define
    an \emph{induced} inner product and orientation on $\mathcal T_x(V)$ in the
    obvious way.
\end{defn}
\begin{defn}
    Let $A$ be open in $W$, and let $f:A\to V$ be differentiable. We define
    $f_*:\mathcal T_x(W)\to\mathcal T_{f(x)}(V)$ by 
    $f_*(x,v)=(f(x),D_v f(x))$. 
\end{defn}
\begin{prop}
    Let $M\subseteq W$ be a manifold with boundary, and let $p\in M$. Choose a chart
    $(U,\varphi)$ around $p$, and let $f=\varphi^{-1}$.
    The set $f_*(\mathcal T_{\varphi(p)}(V))$ is a subspace of $T_p(W)$
    independent of the choice of chart.
\end{prop}
\begin{defn}
    This space is the \emph{tangent space} to $M$ at $p$, denoted $\mathcal
    T_p(M)$. The dual space $(T_x(M))^*$ is the \emph{cotangent space} to $M$ at
    $x$, denoted $T_x^*(M)$.
\end{defn}
\begin{defn}
    A $k$-\emph{tensor field} on a manifold with boundary $M$ is a function assigning,
    to each $x\in M$, an element of $\otimes^k \mathcal T_x^*(M)$. A
    \emph{differential form} of order $k$ is a $k$-tensor field such that every
    element of its image is alternating. The \emph{wedge product} of two
    differential forms is taken pointwise.
\end{defn}
\begin{rem}
    Note that a function from $M$ to $\Ff$ is a differential $0$-form.
\end{rem}
\begin{prop}
    If $f$ and $g$ are $C^n$ differential forms defined on an open set, then so
    is $f\wedge g$.
\end{prop}
\begin{defn}
    Let $M_1$ and $M_2$ be manifolds with boundary, and let $\varphi:M_1\to M_2$ be
    continuous. We say that two charts $(U_1,\alpha_1)$ and $(U_2,\alpha_2)$ are
    \emph{compatible} under $\varphi$ if $\varphi(U_1)\subseteq U_2$.
\end{defn}
\begin{defn}
    A continuous function $\varphi:M_1\to M_2$ is \emph{differentiable}
    if for any compatible charts
    $(U_1,\alpha_1)$ and $(U_2,\alpha_2)$, the map
    $\alpha_2\circ\varphi\circ\alpha_1^{-1}$ is differentiable.
\end{defn}
\begin{prop}
    If $\varphi$ is differentiable, there is a unique function $\varphi_*$
    such that for any compatible charts
    $(U_1,\alpha_1)$ and $(U_2,\alpha_2)$ we have
    \[\varphi_*\circ(\alpha_1^{-1})_*=(\alpha_2^{-1})_*\circ
    (\alpha_2\circ\varphi\circ\alpha_1^{-1})_*.\]
\end{prop}
\begin{defn}
    For a differential $k$-form $\omega$ on $M_2$ and a differentiable map 
    $f:M_1\to M_2$, we define the \emph{pullback} 
    $f^*\omega=(\wedge^k f_*)\omega$.
\end{defn}
\begin{prop}
    The function $f^*$ commutes with the operations $+$, $\cdot$ and $\wedge$.
\end{prop}
\begin{defn}
    If $f$ is a differential $k$-form defined on an open set, we define the
    \emph{exterior derivative} $df$ as $df(p)=\Alt(f'(p))$.
\end{defn}
\begin{prop}
    Let $p$ be a point on a manifold with boundary $M$, and let $\omega$ be a differential
    form defined on $M$. Let $(U,\varphi)$ be a chart defined on a neighbourhood of
    $p$, and let $f=\varphi^{-1}$. Then,
    \[\varphi^*(df^*(\omega))(p)\] is independent of the choice of $\varphi$.
\end{prop}
\begin{defn}
    We define this expression to be the \emph{exterior derivative} $d\omega(p)$.
\end{defn}
\begin{prop}
    The exterior derivative has the following properties:
    \begin{itemize}
        \item If $f$ is a $0$-form, then $df=f'$.
        \item If $f$ is a $C^k$ differential $n$-form, then $df$ is a $C^{k-1}$
            differential $(n+1)$-form.
        \item Let $\omega$ be a $k$-differential form. Then,
            $d(\omega\wedge\nu)=d\omega\wedge\nu+(-1)^k\omega\wedge d\nu$.
        \item $d(f^*\omega)=f^*(d\omega)$.
    \end{itemize}
\end{prop}
\begin{defn}
    A form $\omega$ is called \emph{closed} if $d\omega=0$ and \emph{exact} if
    $\omega=d\nu$ for some $\nu$.
    Let $M$ be a manifold, and $\Omega^k(M)$ be the set of $k$-differential forms
    on $M$. Let $C^k(M)$ and $E^k(M)$ be the sets of closed and exact forms on
    $M$, respectively.
\end{defn}
\begin{prop}
    We have $E^k(M)\subseteq C^k(M)\subseteq\Omega^k(M)$, and each of these sets
    is a vector space.
\end{prop}
\begin{defn}
    The \emph{deRham cohomology group} $H^k(M)$ is the
    quotient space $C^k(M)/E^k(M)$.
\end{defn}
\begin{thm}[Poincar\'e Lemma]
    The deRham cohomology group of a star-shaped set is trivial for $k\ge 1$.
\end{thm}
\begin{prop}
    Let $\omega$ be a differential $k$-form in an open $D\subseteq V$, where $V$
    is $k$-dimensional, and let $p$ be a point in $D$. Let $O$ be an orientation
    on $V$. For any orthonormal basis $\{b_i\}$ of $\mathcal T_p(V)$, the expression
    \[k!O(\{b_i\})\omega(\{b_i\})\] is independent of $b_i$.
\end{prop}
\begin{defn}
    We call the value of this expression $\rho^\omega(p)$, and define
    \[\int_D \omega=\int_D\rho^\omega(p)\mathrm dV.\]
\end{defn}
\begin{defn}
    Let $M_1$ and $M_2$ be two manifolds with boundary, and define $\mu_1(p)$ to be an
    orientation of $\mathcal T_p(M_1)$ for each $p\in M_1$. Define $\mu_2$
    similarly on $M_2$. Then a differentiable map $f:M_1\to M_2$ is
    \emph{orientation-preserving} if $\mu_2\circ f_*=\mu_1$. A manifold $M_1$ is
    \emph{oriented} if $\mu_1$ is defined such that for each chart
    $(U,\varphi)$, an orientation can be chosen on the codomain of $\varphi$
    such that $\varphi$ is orientation-preserving.
\end{defn}
\begin{prop}
    Let $\omega$ be a $k$-form defined on an oriented $k$-manifold with boundary
    $M$ which has support $S$. For any orientation-preserving chart $\varphi$
    defined on a superset of $S$, \[\int_{\varphi(S)}\varphi^*(\omega)\]
    is uniquely defined.
\end{prop}
\begin{defn}
    We define this value, if it exists, to equal $\int_M\omega$.
\end{defn}
\begin{prop}
    Let $\omega$ be a $k$-form defined on a $k$-manifold with boundary $M$. Let
    $\{(U_i,\varphi_i)\}$ be an atlas of $M$, and let $\{g_j\}$ be a partition
    of unity subordinate to $\{U_i\}$. If
    \[\sum_j\int_M g_j\omega\]
    converges for some choice of $\{g_j\}$, then it converges to the same value
    for all choices.
\end{prop}
\begin{defn}
    We define this value, if it exists, to be $\int_M\omega$.
\end{defn}
\begin{prop}
    At each point $p\in\partial M$, there are exactly two unit vectors $n_1,n_2$
    in $\mathcal T_M(P)$ which are perpendicular to all vectors in
    $\mathcal T_{\partial M}(P)$. If $\varphi:M\to\Hh^k$ is a patch, then
    $\varphi_*(n_1)$ and $\varphi_*(n_2)$ differ only in the sign of the last
    component; one of them always has positive sign, and the other always has
    negative sign irrespective of the choice of $\varphi$.
\end{prop}
\begin{defn}
    We let the \emph{outward unit normal} at $p$ be whichever of $n_1$ and $n_2$
    has this sign negative.
\end{defn}
\begin{prop}
    Let $M$ be an oriented manifold with boundary, and let $p\in\partial M$.
    Let $n(p)$ denote the outward unit normal at $p$, and let $\mu(p)$ be the
    orientation of $M$ at $p$. The function $\nu(p)$ defined by
    \[\nu(p)(v_2,\ldots,v_k)=\mu(n,v_2,\ldots,v_k)\] orients $\partial M$.
\end{prop}
\begin{defn}
    This is the \emph{induced orientation} on $\partial M$.
\end{defn}
\begin{thm}[Stokes]
    If $D$ is a $C^\infty$ manifold with boundary of dimension $n$ and $\omega$ is
    a $C^1$ differential $(n-1)$-form, then
    \[\int_D d\omega=\int_{\partial D}\omega.\]
\end{thm}
\begin{defn}
    Let $f:V\to V$. The \emph{divergence} of $f$ is $\nabla\cdot f=\tr f'$.
\end{defn}
\begin{defn}
    Let $S\subseteq\Rr^n$ be an $(n-1)$-dimensional manifold. For any function
    $f:\Rr^n\to\Rr^n$, we define a differential $(n-1)$-form $\omega$ by letting
    $\omega(p)(V)$ be the determinant of the linear map which
    takes the standard basis to $(V,f(p))$. Then the \emph{surface integral} of
    $f$ over $S$ is defined as
    \[\int_S f\cdot\mathrm dA=\int_S\omega.\]
\end{defn}
\begin{thm}[Gauss' Divergence]
    Let $S\subseteq\Rr^n$ be an $n$-dimensional
    manifold with boundary. Then, for any $C^1$ function $f$, we have
    \[\int_S \nabla\cdot f\mathrm dV=\int_{\partial S}f\cdot\mathrm dA.\]
\end{thm}
\begin{prop}
    Let $F:V\to V$ be $C^1$, where $V\subseteq\Rr^3$. At each point $(t,F(t))$
    there exists a unique vector $\nabla\times F$ such that for all $y$ we have
    \[(F'(X)-F'(X)^T)y=(\nabla\times F)\times y.\]
\end{prop}
\begin{defn}
    This vector $\nabla\times F$ is the \emph{curl} of $F$.
\end{defn}
\begin{thm}[Stokes]
    Let $M\subseteq\Rr^3$ be an oriented $2$-manifold with boundary, and let
    $\partial M$ have the induced orientation. Let $F:\Rr^3\to\Rr^3$ be $C^1$.
    Then, \[\int_M (\nabla\times F)\cdot \mathrm dA=\int_{\partial
    M}f\cdot\mathrm ds.\]
\end{thm}
\subsection*{References}
\begin{itemize}
  \item Spivak, \emph{Calculus on Manifolds}
  \item Munkres, \emph{Analysis on Manifolds}
  \item Loomis and Sternberg, \emph{Advanced Calculus}
  \item Nickerson, Spencer and Steenrod, \emph{Advanced Calculus}
\end{itemize}

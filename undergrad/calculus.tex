\section{Calculus}
\begin{defn}
    Let $f:E\to Y$ be a function. If there exists a linear map $A:X\to Y$ such that
    \[\lim_{h\to 0}\frac{\|f(x+h)-f(x)-Ah\|}{\|h\|}=0,\]
    then $f$ is said to be \emph{differentiable at} $x$, $A$ is said to be the
    \emph{derivative} of $f$ at $x$, and we write $f'(x)=A$.
    If $f$ is differentiable at every $x\in E$, we say that $f$ is
    \emph{differentiable}.
\end{defn}
\begin{prop}
    If $f$ is differentiable at $x$, then $f'(x)$ is uniquely defined.
\end{prop}
\begin{prop}
    If $A$ is linear and $f(x)=Ax$, then $f'(x)=A$.
\end{prop}
\begin{prop}[Chain Rule]
    Let $f:E\to F$ and $g:F\to G$ be such that $f$ is differentiable at $x$ and $g$
    is differentiable at $f(x)$. Then \[(g\circ f)'(x)=g'(f(x))f'(x).\]
\end{prop}
\begin{prop}
    Suppose $\{f_n'\}$ converges uniformly and $\{f_n(x_0)\}$ converges. Then
    \[\left(\lim_{n\to\infty} f_n\right)'=\lim_{n\to\infty} f_n'.\]
\end{prop}
\begin{prop}
    Let $f:[a,b]\to\Rr$ be differentiable. Then the image of $f'$ contains
    $[f'(a),f'(b)]$.
\end{prop}
\begin{defn}
    The \emph{directional derivative} $D_u f$ is defined as
    \[D_u f(x)=\lim_{t\to 0}\frac{f(x+tu)-f(x)}{t}.\]
\end{defn}
\begin{prop}[Mean Value Theorem]
    Let $f:X\to\Ff$, and let $x,y\in X$ such that the line
    segment joining $x$ and $y$ lies in $X$ and $D_{y-x}f$ exists everywhere on
    that line. Then there exists a point $c$ on
    this line segment such that $D_{y-x}(c)=\frac{f(y)-f(x)}{\|y-x\|}$.
\end{prop}
\begin{prop}
    Suppose $X$ is convex and $\|f'(x)\|\le M$ for all $x\in X$. Then, for $a,b\in E$ we have
    $\|f(b)-f(a)\|\le M\|b-a\|$.
\end{prop}
\begin{prop}[Sum Rule]
    $(f+g)'=f'+g'$.
\end{prop}
\begin{prop}
    If $A$ is bilinear, then
    \[A(f(x),g(x))'(x)h=A(f'(x)h,g(x))+A(f(x),g'(x)h).\]
\end{prop}
\begin{cor}[Product and Quotient Rule]
    If the codomain of $g$ is $\Rr$, then
    \begin{itemize}
        \item $(fg)'=f'g+fg'$.
        \item $(f/g)'=\frac{f'g-fg'}{g^2}$ (if $g\ne 0$).
    \end{itemize}
\end{cor}
\begin{prop}
    We have:
    \begin{itemize}
        \item $\exp'=\exp$
        \item $\ln'(x)=\frac 1x$
        \item $\sin'=\cos$
        \item $\cos'=-\sin$
        \item $\tan'(x)=1+\tan^2(x)$
        \item $\arcsin'(x)=\frac 1{\sqrt{1-x^2}}$
        \item $\arccos'(x)=\frac{-1}{\sqrt{1-x^2}}$
        \item $\arctan'(x)=\frac 1{1+x^2}$
    \end{itemize}
\end{prop}
\begin{defn}
    A continuous function is $C^0$. If the derivative of a function exists and
    is $C^n$, then the function is $C^{n+1}$.
\end{defn}
\begin{prop}
    If $f$ and $g$ are $C^n$, then so are $f+g$, $f\cdot g$ and $f/g$ (assuming
    the codomains match).
\end{prop}
\begin{prop}
    The function $f$ is $C^1$ iff $D_u f$ exists and is continuous for every $u$.
\end{prop}
\begin{prop}
    If $D_v f$, $D_u f$ and $D_v D_u f$ exist in an open ball containing $x$,
    and if $D_v D_u f$ is continuous at $x$, then $D_u D_v f(x)=D_v D_u f(x)$.
\end{prop}
\begin{cor}
    If $f$ is $C^2$, then $D_u D_v f = D_v D_u f$.
\end{cor}
\begin{thm}[Taylor]
    Let $f$ be $n$-times differentiable and defined on an open convex set.
    There exists a unique polynomial $P_n$ of degree $n$
    such that the first $n$ derivatives of $f$ at $a$ equal the first $n$ derivatives
    of $P$ at $a$. Let $R_{n+1}(x)=f(x)-P_n(x)$; then we have
    \[\lim_{x\to a}\frac{R_{n+1}(x)}{\|x-a\|^n}=0.\]

    Further, if $f$ is $(n+1)$-times differentiable and its codomain is $\Rr$,
    let $\gamma(t)=a+(x-a)t$. We have the following expressions for the remainder:
    \begin{itemize}
        \item There is a $t\in[0,1]$ such that
            $R(x)=P_{n+1}(\gamma(t))-P_n(\gamma(t))$.
        \item If $g(t)=f(\gamma(t))$, then
            \[R(x)=\frac 1{n!}\int_0^1 (1-t)^n g^{(n+1)}(t)\mathrm dt.\]
    \end{itemize}
\end{thm}
\begin{prop}[Second derivative test]
    Let $f:X\to\Rr$ be $C^2$, where $X\subseteq\Rr^n$.
    Let $a$ be a point of $f$ such that $f'(a)=0$.
    We define the \emph{Hessian matrix} $H_f(a)$ to be the matrix representation
    of $f''(a)$ using the standard bases for $\Rr^n$ and its dual space.
    \begin{itemize}
        \item If $H_f(a)$ is positive (resp.\ negative) definite, then $f$ has a
            local minimum (resp.\ maximum) at $a$.
        \item If $f$ has a local minimum (resp.\ maximum) at $a$, then $H_f(a)$
            is positive (resp.\ negative) semidefinite.
    \end{itemize}
\end{prop}
\begin{defn}
    A \emph{$k$-cell} $S$ is a product of $k$ closed intervals. Its \emph{volume} is
    $v(S)$, the product of the lengths of the intervals.
\end{defn}
\begin{defn}
    A \emph{partition} of a closed interval $[a,b]$ is a sequence
    $t_0,\ldots,t_k$, where $a=t_0\le t_1\ldots\cdots\ldots t_k=b$.

    A \emph{partition} of a $k$-cell
    is a sequence of $k$ partitions $\{P_i\}$, where each $P_i$ is
    a partition of the corresponding $[a_i,b_i]$.
    This partition divides the $k$-cell into a collection of \emph{subcells}.
\end{defn}
\begin{defn}
    For a partition $P$ and a bounded function $f$ we define the \emph{lower} and
    \emph{upper sums} as
    \[L(f,P)=\sum_S\left(v(S)\inf_{x\in S}f(S)\right),\
    U(f,P)=\sum_S\left(v(S)\sup_{x\in S}f(S)\right).\]
\end{defn}
\begin{prop}
    If $P_1$ and $P_2$ are two partitions of the same $k$-cell, then
    $L(f,P_1)\le U(f,P_2)$.
\end{prop}
\begin{defn}
    Let $A$ be a $k$-cell. A function $f:A\to\Rr$ is called \emph{integrable}
    over $A$ if
    $f$ is bounded and $\sup\{L(f,P)\}=\inf\{U(f,P)\}$. In that case, their
    common value is the \emph{integral} of $f$ over $A$, denoted
    \[\int_A f\mathrm dV.\]
\end{defn}
\begin{prop}
    A bounded function $f$ is integrable over $A$ iff for all $\varepsilon>0$ there is a
    partition $P$ of $A$ such that $U(f,P)-L(f,P)\le\varepsilon$.
\end{prop}
\begin{defn}
    A subset $A$ of $\Rr^k$ has \emph{measure} $0$ if for every $\varepsilon>0$
    there is a cover of $A$ by $k$-cells with total volume less than
    $\varepsilon$.
\end{defn}
\begin{thm}[Sard]
    Let $g:A\to\Rr^n$ be continuously differentiable, where $A$ is open. Then
    the subset of $A$ on which $\det g'=0$ has measure $0$.
\end{thm}
\begin{thm}
    A bounded function is integrable over a $k$-cell iff its set of
    discontinuities in the $k$-cell has measure $0$.
\end{thm}
\begin{defn}
    The \emph{support} of a function $f$ is the closure of the set of points at
    which $f$ is nonzero.
\end{defn}
\begin{prop}
    If $f$ has compact support and is integrable over some $k$-cell containing
    its support,
    then for any $k$-cells $A$ and $B$ which contain
    its support we have $\int_A f(x)\mathrm dV=\int_B f(x)\mathrm dV$.
\end{prop}
\begin{defn}
    We define the \emph{characteristic function}
    $\chi_C(x)$ to be $1$ for any $x\in C$, and $0$ elsewhere.
\end{defn}
\begin{cor}
    If $A$ is a $k$-cell containing $C$, and if
    $\int_A\chi_C(x)f(x)\mathrm dV$ exists, then $\int_B\chi_C(x)f(x)\mathrm dV$
    exists and equals this value for any $k$-cell $B$ containing $C$.
\end{cor}
\begin{defn}
    We define $\int_C f(x)\mathrm dV$, to be this value, if it exists. In the
    case where $C$ is an interval $[a,b]$, we also write this as $\int_a^b
    f(x)\mathrm dV$.
\end{defn}
\begin{prop}
    If $f$ is integrable over a $k$-cell $A$, $C$ is a subset of $A$ and the
    boundary of $C$ has measure $0$, then $f$ is integrable over $C$.
\end{prop}
\begin{prop}
    Suppose $\{f_n\}$ converges uniformly and each $f_n$ is integrable over $C$. Then,
    \[\int_C\lim_{n\to\infty} f_n(x)\mathrm dx=\lim_{n\to\infty}\int_C
    f_n(x)\mathrm dx.\]
\end{prop}
\begin{thm}[Fubini]
    Let $A$ and $B$ be $k$-cells, and let $f:A\times B\to\Rr$ be such that
    $f(x,b)$ is integrable for each $b\in B$ and $f(a,x)$ is integrable for each
    $a\in A$. Then, $f$ is integrable over $A\times B$ and
    \[\int_{A\times B}f\mathrm dV=\int_A\left(\int_B f\mathrm dV\right)\mathrm
    dV=\int_B\left(\int_A f\mathrm dV\right)\mathrm dV.\]
\end{thm}
\begin{thm}[Change of Variables]
    Let $A$ be open in $\Rr^n$, and let $g:A\to\Rr^n$ be $C^1$ and bijective.
    If $f:g(A)\to\Rr$ is integrable, then
    \[\int_{g(A)}f\mathrm dV=\int_A(f\circ g)|\det g'|.\]
\end{thm}
\begin{thm}[Differentiation under the Integral]
    Let $f:[a,b]\times[c,d]$ be such that $f(\cdot,t)$
    is integrable for all $t$, and $D_{(0,1)}f$ is continuous. Then, 
    \[f'(s)=\int_a^b D_{(0,1)}f(x,s)\mathrm dx\]
    for each $s\in(c,d)$.
\end{thm}
\begin{thm}[Fundamental Theorem of Calculus]
    If $f:[a,b]\to\Rr$ is integrable, define
    \[F(x)=\int_a^x f(t)\mathrm dt.\]
    If $f$ is continuous at $c\in[a,b]$, then $F$ is differentiable at $c$, and
    $F'(c)=f(c)$.

    If $F:[a,b]\to\Rr$ is differentiable, define $f(x)=F'(x)$. If $f$ is
    integrable on $[a,b]$, then \[F(x)=F(a)+\int_a^x f(t)\mathrm dt.\]
\end{thm}
\begin{thm}[Integration by Parts]
    Let $F$ and $G$ be differentiable on $[a,b]$ such that $F'=f$ and $G'=g$ are
    integrable. Then,
    \[\int_a^b F(x)g(x)\mathrm dx=F(b)G(b)-F(a)G(a)-\int_a^b f(x)G(x)\mathrm
    dx.\]
\end{thm}
\begin{prop}
    We have:
    \begin{itemize}
        \item $\ln(1+x)=\sum (-1)^i\frac{x^{i+1}}{i+1}$ for $-1<x\le 1$
        \item $\frac12\ln\left(\frac{1+x}{1-x}\right)=\sum
            \frac{x^{2i+1}}{2i+1}$ for $-1<x<1$
        \item $\arctan(x)=\sum \frac{(-1)^i x^{2i+1}}{2i+1}$ for $-1\le x\le 1$
    \end{itemize}
\end{prop}
\begin{defn}
    Let $V$ be a normed vector space over $\Rr$. A \emph{curve} is a map $f:[a,b]\to
    V$. We associate to each partition $P=\{x_i\}$ of $[a,b]$ the number
    $L(P)=\sum\|\gamma(x_i)-\gamma(x_{i-1})\|$. If these numbers have a
    supremum, then this supremum is the \emph{length} of $\gamma$ and $\gamma$
    is \emph{rectifiable}. We say that $\gamma$ is \emph{piecewise smooth} if
    its domain can be divided into a finite number of intervals such that
    $\gamma$ is $C^1$ on each interval.
\end{defn}
\begin{prop}
    If $\gamma$ is piecewise smooth, then $\gamma$ is rectifiable. For each
    $x\in[a,b]$, the length of $\gamma([a,x])$ is
    \[s(x)=\int_a^x\|\gamma'(t)\|\mathrm dt.\]
\end{prop}
\begin{prop}
    There is a function $F$
    such that $\gamma(x)=F(s(x))$ for each $x$. If $F$ is differentiable, then
    $\|F'\|=1$ everywhere.
\end{prop}
\begin{defn}
    The vector $F'(s)$ is called the \emph{unit tangent vector} to $\gamma$.
\end{defn}
\begin{defn}
    Let $D$ be an open subset of an inner product space $V$, and let $F:D\to V$
    be continuous. Let $C:[a,b]\to D$ be piecewise smooth.
    The \emph{line integral} of $F$ along $C$ is
    \[\int_C F\cdot\mathrm ds=\int_c^d F(C(t))\cdot C'(t)\mathrm dt.\]
\end{defn}
\begin{prop}
    The line integral of a curve is independent of parametrisation.
\end{prop}
\begin{prop}
    Let $D$ be an open subset of $V$, where $V$ is an inner product space over $\Rr$.
    If $f:D\to\Rr$ is differentiable, then for each $x$ there is a vector $\nabla f(x)$ such
    that $f'(x)(y)=\nabla f\cdot y$ for all $y$.
\end{prop}
\begin{defn}
    This vector $\nabla f$ is called the \emph{gradient} of $f$.
\end{defn}
\begin{prop}
    Let $E$ be open in $X$, and let $G:E\to Y$ be $C^1$. Let $A=G^{-1}(0)$.
    Assume $G'(a)$ is surjective for all $a\in A$.
    If $f:E\to\Rr$ is differentiable and the maximum of $f$ on $A$ occurs at $a$,
    then there is a functional $l\in Y^*$ such that $f'(a)=lG'(a)$.
\end{prop}
\begin{cor}[Lagrange Multipliers]
    If the codomain of $G$ is $\Rr$,
    then for some $\lambda$ we have $\nabla f=\lambda\nabla g$.
\end{cor}
\begin{thm}[Fundamental Theorem of Line Integrals]
    Let $f:D\to\Rr$ be $C^1$, and let $\gamma:[a,b]\to D$ be a piecewise smooth
    curve such that $\gamma(a)=X_0$ and $\gamma(b)=X_1$. Then,
    \[\int_C\nabla f\cdot\mathrm ds=f(X_1)-f(X_0).\]
\end{thm}
\begin{defn}
    A vector field $F:D\to V$ is \emph{conservative} if $\int_C F\cdot\mathrm ds=0$
    whenever $C$ is a closed curve.
\end{defn}
\begin{cor}
    A vector field over a connected open set is conservative iff it is the
    gradient of some function.
\end{cor}
\begin{defn}
    An $n$-dimensional \emph{patch} is a subset $S$ of an vector space $V$ such
    that there are vector spaces $X$ and $Y$, an isomorphism $\phi:V\to
    X\times Y$, an open subset $E$ of $X$,
    and a function $f:E\to Y$ such that $X$ is $n$-dimensional and
    the graph of $f$ is $\phi(S)$. It is \emph{smooth} if $f$ is $C^1$.

    An $n$-dimensional \emph{manifold} is a subset $S$ of a vector space $V$ such
    that for each $x\in S$, there is an open set $E$ containing $x$ such $S\cap
    E$ is an $n$-dimensional patch. If all these patches are smooth, then $S$ is
    \emph{smooth}.
\end{defn}
\begin{defn}
    A collection $\{g_i:M\to[0,1]\}$ of $C^\infty$ functions is a \emph{partition of
    unity} if
    \begin{itemize}
        \item each $g_i$ has compact support,
        \item each $x\in M$ has a neighbourhood $V_x$ such that all but finitely
            many $g_i$ are $0$ on $V_x$, and
        \item $\sum g_i=1$ everywhere on $M$.
    \end{itemize}

    A partition of unity $\{g_i\}$ is \emph{subordinate} to an open cover
    $\{U_i\}$ of $m$ if for every $j$ the support of $g_j$ is contained in some
    $U_i$.
\end{defn}
\begin{thm}
    Let $\{U_i\}$ be an open covering of $M$. There exists a partition of unity
    $\{g_i\}$ subordinate to $\{U_i\}$.
\end{thm}
\begin{thm}[Inverse Function]
    Let $f:X\to X$, and let $a$ be such that $f$ is $C^1$ in an open ball
    containing $a$ and $f'(a)$ is invertible.
    Then there is an open ball $E$ containing $a$ and such that $f$ is injective on
    $E$, $F(E)$ is open, and $(F|_E)^{-1}$ is $C^1$.
\end{thm}
\begin{cor}
    If $f:X\to X$ is $C^1$, then it sends open sets to open sets.
\end{cor}
\begin{thm}[Implicit Function]
    Let $\phi:X\times Y\to Y$ be $C^1$ in an open set containing $(x,y)$, and
    assume $\phi(x,y)=0$. Let $\phi'(x,y)=A(x)+B(y)$, and assume $B$ is invertible.
    Then there is an open set $E$ containing $x$ and an open set $F$ containing
    $y$ such that for each $e\in E$ there is a unique $f\in F$ such that
    $\phi(e,f)=0$. The function $e\mapsto f$ is differentiable.
\end{thm}
\begin{thm}[Rank Theorem]
    Let $A$ be an open set in $V$, let $r<\dim W$ be an integer, and let
    $F:A\to W$ be $C^1$ such that the rank of $F'$ is $r$ at every point in $A$.
    Then for each point $a\in A$ there is an open set $B$ containing $a$ such
    that $F(B)$ is an $r$-dimensional manifold in $W$.
\end{thm}
% TODO: general stokes' theorem
\begin{defn}
    Let $f:V\to V$. The \emph{divergence} of $f$ is $\nabla\cdot f=\tr f'$.
\end{defn}
% TODO: n-dimensional divergence theorem
\begin{prop}
    Let $F:V\to V$ be $C^1$, where $V\subseteq\Rr^3$. At each point $(t,F(t))$
    there exists a unique vector $\nabla\times f$ such that for all $y$ we have
    \[(F'(X)-F'(X)^T)y=(\nabla\times f)\times y.\]
\end{prop}
\begin{defn}
    This vector $\nabla\times F$ is the \emph{curl} of $F$.
\end{defn}
% TODO: 3d stokes' theorem
\subsection*{References}
\begin{itemize}
  \item Loomis and Sternberg, \emph{Advanced Calculus}
  \item Nickerson, Spencer and Steenrod, \emph{Advanced Calculus}
  \item Spivak, \emph{Calculus on Manifolds}
  \item Munkres, \emph{Analysis on Manifolds}
\end{itemize}

\section{Analysis}
\begin{defn}
    A \emph{topology} on a set $X$ is a collection $T$ of \emph{open sets} in
    $X$ such that $\emptyset, X\in T$, and $T$ is closed under arbitrary unions
    and finite intersections. A set for which a topology has been specified is a
    \emph{topological space}. Given $x\in X$, a \emph{neighbourhood} of $x$ is
    an element $U\in T$ such that $x\in U$.
\end{defn}
\begin{defn}
    If $X$ is a set, a \emph{basis} for a topology on $X$ is a collection $B$ of
    subsets of $X$ such that $\bigcup B=X$, and if $B_1,B_2\in B$ have nonempty
    intersection then there is a $B_3\in B$ contained in their intersection.
\end{defn}
\begin{prop}
    If $B$ is a basis for a topology on $X$, the collection of all unions of
    elements of $B$ defines a topology on $X$.
\end{prop}
\begin{defn}
    This is the topology \emph{generated} by $B$.
\end{defn}
\begin{prop}
    Let $X$ be a topological space with topology $T$. If $Y$ is a subset of $X$,
    the collection $T_Y=\{Y\cap U:U\in T\}$ is a topology on $Y$.
\end{prop}
\begin{defn}
    This is the \emph{subspace topology}.
\end{defn}
\begin{defn}
  Let $E$ be a subset of a topological space $M$.
  \begin{itemize}
    \item A point $p$ is a \emph{limit point} of $E$ if neighbourhood of $p$
      contains a point $q\ne p$ such that $q\in E$.
  \item A point $p$ is a \emph{condensation point} of $E$ if every neighbourhood
      o f$p$ contains uncountably many points of $E$.
    \item A point $p$ is an \emph{interior point} of $E$ if there is a
        neighbourhood of $p$ which is a subset of $E$.
    \item $E$ is \emph{closed} if every limit point of $E$ is a point of $E$.
    \item $E$ is \emph{open} if every point of $E$ is an interior point of $E$.
    \item $E$ is \emph{perfect} if $E$ is closed and every point of $E$ is a
        limit point of $E$.
    \item The \emph{complement} $E^c$ of a set $E$ is the set $M\setminus E$.
    \item The \emph{interior} of $E$ is the set of interior points of $E$.
    \item The \emph{boundary} $\partial E$ of $E$ is the set of points of $M$
      that are limit points of both $E$ and $E^c$.
  \item The \emph{closure} of $E$ is the set $\overline E=E\cup\partial E$.
  \end{itemize}
\end{defn}
\begin{prop}
  The interior and boundary of $E$ are disjoint, and their union is $E$.
\end{prop}
\begin{prop}
  The following are equivalent:
  \begin{itemize}
    \item $E$ is its own interior.
    \item $E\cap\partial E=\emptyset$.
    \item $E^c$ contains its limit points.
    \item $\partial E\subseteq E^c$.
  \end{itemize}
\end{prop}
\begin{prop}
  The closure of $E$ is closed; the interior of $E$ is open.

  Any closed set which contains $E$ contains the closure of $E$. Any open set
  which is contained in $E$ is contained in the interior of $E$.
\end{prop}
\begin{defn}
  A sequence $\{a_n\}$ is \emph{convergent} if there is a point $L$ such that
  every neighbourhood of $L$ contains all but finitely many $a_n$.
  We write $\lim_{n\to\infty} a_n=L$, or $a_n\to L$ as $n\to\infty$.
\end{defn}
\begin{defn}
    A set $E$ is \emph{connected} if the only subsets of $E$ that are both open
    and closed in $E$ are the empty set and $E$ itself; otherwise, it is
    \emph{disconnected}. It is \emph{path-connected} if for all $x,y\in E$ there
    is a continuous function $f:[0,1]\to E$ such that $f(0)=x$ and $f(1)=y$.
    It is \emph{contractible} if there is a continuous map $f:[0,1]\times E\to E$
    such that $f(0,x)=x$ and $f(1,x)$ is constant.
\end{defn}
\begin{prop}
    \begin{itemize}
        \item Every open connected set is path-connected.
        \item Every path-connected set is connected.
        \item Every contractible set is path-connected.
    \end{itemize}
\end{prop}
\begin{prop}
    A topological space is disconnected iff it is the union of two 
    disjoint nonempty open sets.
\end{prop}
\begin{defn}
  We say that $f$ is \emph{continuous} at $p$ if for each neighbourhood $B$ of
  $f(p)$, there is a neighbourhood $A$ of $p$ such that $f(A)\subseteq B$.

  We say that $f$ is \emph{continuous} on $X$, or simply \emph{continuous},
  if it is continuous at every point in $X$.
\end{defn}
\begin{prop}
  A function is continuous iff the inverse image of every open set is open.
\end{prop}
\begin{prop}
    A continuous image of a connected set is connected.
\end{prop}
\begin{prop}
    A monotonic function $f:[a,b]\to\Rr$ has at most countably many
    discontinuities.
\end{prop}
\begin{defn}
  An \emph{open cover} of a topological space $E$ is a set of open sets whose union contains $E$.
\end{defn}
\begin{defn}
    A set $E$ is said to be \emph{compact} if every open cover of $E$ contains a
    finite subcover.
\end{defn}
\begin{prop}
  Suppose $X\subseteq M$. A subset $E$ of $X$ is compact
  relative to $X$ (under the subspace topology) iff $E$ is compact relative to $M$.
\end{prop}
\begin{prop}
  If $S$ is a collection of closed subsets of a compact space such that any
  finite intersection of elements of $S$ is nonempty, then $\bigcap S$ is
  nonempty.
\end{prop}
\begin{prop}
    A continuous image of a compact set is compact.
\end{prop}
\begin{defn}
    A topological space is \emph{Hausdorff} if for every pair $(a,b)$ of
    distinct points, there are disjoint neighbourhoods $A$ of $a$ and $B$ of $b$.
\end{defn}
\begin{prop}
    If $f:X\to Y$ is a continuous bijection, $X$ is compact, and $Y$ is
    Hausdorff, then $f^{-1}$ is continuous.
\end{prop}
\begin{defn}
  A \emph{metric space} is a nonempty set $M$ together with a function
  $d:M\times M\to\Rr$ (the \emph{metric}) such that
  \begin{itemize}
    \item $d(x,y)=0\iff x=y$,
    \item $d(x,y)=d(y,x)$ (symmetry), 
    \item $d(x,z)\le d(x,y)+d(y,z)$ (triangle inequality).
  \end{itemize}
\end{defn}
\begin{defn}
  In a metric space, the \emph{open ball} $B_r(x)$ with centre $x$ and radius $r$ is the
  set of all points $y$ with $d(x,y)<r$.

  The \emph{closed ball} $\overline{B_r(x)}$ with centre $x$ and radius $r$ is
  the set of all points $y$ with $d(x,y)\le r$.
\end{defn}
\begin{prop}
  In a normed vector space, the function $d(x,y)=\|x-y\|$ is a metric.
\end{prop}
\begin{defn}
  We call this the \emph{induced metric}.
\end{defn}
\begin{prop}
    The set of open balls on a metric space is a basis for a topology.
\end{prop}
\begin{prop}
  Suppose $\{a_n\}$ and $\{b_n\}$ are sequences of complex numbers which
  converge to $a$ and $b$ respectively. Then the sequences $\{a_n+b_n\},
  \{a_n b_n\},\{\frac{a_n}{b_n}\}$ converge to $a+b,ab,\frac ab$
  respectively (where in the last one we require $b_n\ne 0$ for each $n$, and
  $b\ne 0$).
\end{prop}
\begin{defn}
    Let $E$ be a subset of a metric space $X$.
    \begin{itemize}
        \item $E$ is \emph{bounded} if it is contained in some open ball.
        \item $E$ is \emph{totally bounded} if for any $\varepsilon>0$ there are a
            finite number of open balls of radius $\varepsilon$ which cover $E$.
    \end{itemize}
\end{defn}
\begin{thm}[Monotone Convergence]
    A monotonic sequence in $\Rr$ converges iff it is bounded.
\end{thm}
\begin{defn}
  A sequence $\{p_n\}$ in a metric space is \emph{Cauchy} if for every
  $\varepsilon>0$ there is an integer $N$ such that $d(p_n,p_m)<\varepsilon$ if
  $m,n\ge N$.

  A metric space is \emph{complete} if every Cauchy sequence converges.
\end{defn}
\begin{prop}
  Every convergent sequence is Cauchy.
\end{prop}
\begin{prop}
  A sequence in $\Rr^n$ or $\Cc^n$ converges iff it converges coordinatewise.
\end{prop}
\begin{cor}
  $\Rr^n$ and $\Cc^n$ are complete.
\end{cor}
\begin{prop}
    Let $M$ be a metric space. The following are equivalent:
    \begin{itemize}
        \item $M$ is compact.
        \item $M$ is complete and totally bounded.
        \item Every infinite set in $M$ contains a limit point.
        \item Every sequence of open sets $S_1\subseteq S_2\subseteq\cdots$
            fails to cover $M$.
    \end{itemize}
\end{prop}
\begin{cor}[Weierstrass]
  Every bounded infinite subset of $\Rr^n$ has a limit point.
\end{cor}
\begin{cor}[Extreme Value Theorem]
    If the domain of $f$ is compact and the codomain of $f$ is $\Rr$, then the
    image of $f$ is closed and bounded. In particular, $f$ attains its minimum
    and maximum.
\end{cor}
\begin{prop}
    Let $S\subseteq\Rr$. The following are equivalent:
    \begin{itemize}
        \item $S$ is connected.
        \item If $a,b\in S$ and $a\le c\le b$ then $c\in S$.
        \item There are real numbers $a,b$ such that $S$ contains all elements
            $c$ with $a<c<b$ and no elements with $a>c$ or $b<c$.
    \end{itemize}
\end{prop}
\begin{defn}
    Such a set is called an \emph{interval}. 
\end{defn}
\begin{cor}[Intermediate Value Theorem]
    If the domain of $f$ is connected and the codomain of $f$ is $\Rr$, then the
    image of $f$ is an interval.
\end{cor}
\begin{prop}
    For every uncountable subset $A$ of $\Rr^n$, the set $A'$ of its condensation
    points is nonempty and perfect. Further, $A\setminus A'$ is at most
    countable.
\end{prop}
\begin{prop}
    There exists a perfect set in $\Rr$ which contains no segment.
\end{prop}
\begin{prop}
    Every nonempty perfect set in $\Rr^n$ has cardinality $|\Rr|$.
\end{prop}
\begin{thm}[Baire]
    To each countable ordinal $\alpha$, assign a closed set
    $S_\alpha\subseteq\Rr^n$, such that if $\alpha<\beta$ then $S_\beta\subseteq S_\alpha$.
    The intersection of all sets thus defined equals $S_\gamma$ for some
    countable ordinal $\gamma$.
\end{thm}
\begin{cor}[Cantor-Bendixson]
    Let $S$ be a closed set, and define $S^{(\alpha)}$ for every ordinal
    $\alpha$ such that
    \begin{itemize}
        \item $S^{(0)}=S$
        \item $S^{(\alpha+1)}$ is the set of limit points of $S^{(\alpha)}$
        \item $S^{(\alpha)}$ is the intersection of all $S^{(\beta)}$ for
            $\beta<\alpha$, if $\alpha$ is a limit ordinal.
    \end{itemize}
    Then there is an ordinal $\gamma$ which is at most countable such that
    $S^{(\gamma)}$ is the set of condensation points of $S$.
\end{cor}
\begin{defn}
  A function $f$ is \emph{uniformly continuous} if for every $\varepsilon>0$
  there exists a $\delta>0$ such that if $d(a,b)<\delta$ then
  $d(f(a),f(b))<\varepsilon$.
\end{defn}
\begin{thm}
    Every continuous function on a compact metric space is uniformly continuous.
\end{thm}
\begin{defn}
  It is \emph{Lipschitz continuous} if for every
  there is a real constant $K$ such that $d(f(a),f(b))\le Kd(a,b)$ for all $a$
  and $b$. If $K<1$ then $f$ is a \emph{contraction}.
\end{defn}
\begin{prop}[Contraction Principle]
    A contraction on a complete metric space has a unique fixed point.
\end{prop}
\begin{defn}
    Let $\{a_i\}$ be a sequence in $A$, where $A$ is a normed vector space.
    If the limit on the right exists,
    \[\sum_{i=1}^\infty a_i=\lim_{n\to\infty}\sum_{i=1}^n a_i.\]
    Otherwise, the series \emph{diverges}.
\end{defn}
\begin{prop}
    If $\sum a_n$ converges, then $\lim_{n\to\infty} a_n=0$.
\end{prop}
\begin{prop}
    The series $\sum x^n$ converges iff $|x|<1$, in which case its value is
    $\frac{1}{1-x}$.
\end{prop}
\begin{prop}[Comparison test]
    If $\|a_n\|\le c_n$ for all sufficiently large $n$, and $\sum c_n$ converges,
    then so does $\sum a_n$.
\end{prop}
\begin{cor}
    If $\sum\|a_n\|$ converges then so does $\sum a_n$.
\end{cor}
\begin{defn}
    Such a sequence is said to \emph{converge absolutely}.
\end{defn}
\begin{prop}[Cauchy condensation test]
    If $\{a_n\}$ is nonincreasing, then $\sum a_n$ converges iff $\sum 2^n
    a_{2^n}$ converges.
\end{prop}
\begin{cor}
    $\sum n^{-s}$ converges iff $s>1$.
\end{cor}
\begin{prop}
    $\sum\frac 1{n!}=\lim_{n\to\infty}\left(1+\frac 1n\right)^n$.
\end{prop}
\begin{defn}
    We let this limit be $e$.
\end{defn}
\begin{defn}
    We define the \emph{extended real number system} as
    $\Rr\cup\{-\infty,\infty\}$ and extend the operations by
    $x+\infty=\infty,x-\infty=-\infty,\frac x{\infty}=0$
    and for $x>0$, $x\infty=\infty,x(-\infty)=-\infty$.

    If $S\subseteq\Rr$ is unbounded above, then $\sup S=\infty$. If it is
    unbounded below, then $\inf S=-\infty$.

    We may regard the set $\{x\in\Rr:x>a\}$ as an open ball around $\infty$, and
    $\{x\in\Rr:x<-a\}$ as an open ball around $-\infty$. Clearly, the set of
    open balls still forms a basis for a topological vector space.
\end{defn}
\begin{defn}
    We define \[\limsup_{n\to\infty}
    s_n=\lim_{n\to\infty}\sup_{m\ge n} s_m.\]
    We define $\liminf_{n\to\infty} s_n$ similarly.
\end{defn}
\begin{prop}[Root test]
    Let $\alpha=\limsup_{n\to\infty}\sqrt[n]{|a_n|}$. If $\alpha<1$ then $\sum
    a_n$ converges; if $\alpha_1$ then $\sum a_n$ diverges.
\end{prop}
\begin{prop}
    If $\{a_i\}$ is a sequence of positive reals,
\begin{align*}
    \liminf_{n\to\infty}\frac{a_{n+1}}{a_n}&\le\liminf_{n\to\infty}\sqrt[n]{a_n} \\
    \limsup_{n\to\infty}\frac{a_{n+1}}{a_n}&\ge\limsup_{n\to\infty}\sqrt[n]{a_n}.
\end{align*}
\end{prop}
\begin{cor}[Ratio test]
    Let $b_n=\frac{\|a_{n+1}\|}{\|a_n\|}$. If $\limsup_{n\to\infty} b_n<1$,
    then $\sum a_n$ converges. If $\sum a_n$ converges, then $b_n<1$ for
    infinitely many $n$.
\end{cor}
\begin{defn}
    Given a sequence $\{c_n\}$ of complex numbers, the series
    \[\sum_{n=0}^\infty c_n z^n\] is a \emph{power series} in a complex number
    $z$.
\end{defn}
\begin{prop}
    Let $\alpha=\limsup_{n\to\infty}\sqrt[n]{|c_n|}$. Then $\sum c_n z^n$
    converges if $|z|<\frac1\alpha$ and diverges if $|z|>\frac1\alpha$.
\end{prop}
\begin{defn}
    We call $\alpha$ the \emph{radius of convergence} for $\sum c_n z^n$.
\end{defn}
\begin{prop}
    Suppose the partial sums of $\sum a_n$ are bounded, $\{b_n\}$ is decreasing,
    and $\lim_{n\to\infty} b_n=0$. Then $\sum a_n b_n$ converges.
\end{prop}
\begin{cor}
    If $\{c_i\}$ is decreasing and converges to $0$, $\sum c_n z^n$ has
    radius of convergence $r$, and $|z|=r\ne z$, then $\sum c_n z^n$ converges.
\end{cor}
\begin{prop}
    If $\sum a_n$ converges absolutely and $\sum b_n$ converges, then
    \[\sum_n\sum_k a_k b_{n-k}=\left(\sum a_n\right)\left(\sum b_n\right).\]
\end{prop}
\begin{prop}
    If $\sum a_n$ converges absolutely then so does $\sum b_n$, where $\{b_n\}$
    is any permutation of $\{a_n\}$.
\end{prop}
\begin{prop}
    If $\{a_n\}$ are real and $\sum a_n$ converges but does not converge
    absolutely, then $\sum b_n$
    can be made to diverge or converge to any value, where $\{b_n\}$ is a
    permutation of $a_n$.
\end{prop}
\begin{defn}
    Two norms on a vector space are said to be \emph{equivalent} if they induce
    the same topology.
\end{defn}
\begin{prop}
    Any two norms on a finite-dimensional vector space are equivalent.
\end{prop}
\begin{defn}
  Let $L$ be a linear transformation on a normed vector space $V$. We say that
  $L$ is \emph{bounded} if there is some real $K$ such that $\|Lx\|\le K\|x\|$
  for all $x\in V$. The greatest lower bound of all such $K$ is denoted $\|L\|$.
\end{defn}
\begin{prop}
  Let $V$ and $W$ be finite-dimensional normed vector spaces. The function
  $\|\cdot\|:\mathcal L(V,W)\to\Rr$ is a norm on $\mathcal L(V,W)$; any two such
  norms are equivalent.
\end{prop}
\begin{prop}
  Let $A$ be a self-adjoint linear map on a normed vector space $V$, and let
  $S=\{\langle Ax,x\rangle:\|x\|=1\}$. Then, $\|A\|=\max\{|\inf S|,|\sup
  S|\}$.
\end{prop}
\begin{prop}[Min-Max Theorem]
  Let $A$ be a self-adjoint linear map on an $n$-dimensional inner product space
  $V$. Let $\lambda_1\ge\lambda_2\ge\cdots\ge\lambda_n$ be the eigenvalues of
  $A$ according to multiplicity. Then for each $k$ we have
  \[\lambda_k=\inf\{\|A|_U\|:\dim U=n-k+1\}.\]
\end{prop}
\begin{thm}[Ergodic Theorem]
  If $U$ is an isometry on a finite-dimensional inner product space, and if $M$
  is the subspace of all solutions of $Ux=x$, then the sequence
  \[\frac1n(1+U+\cdots+U^{n-1})\] converges to $P_M$.
\end{thm}
\begin{prop}
  Let $f(x)=\sum_{n=0}^\infty a_n x^n$ be a power series with radius of
  convergence $r$. If $A$ is such that $\|A\|<r$, then $f(A)$ exists.
\end{prop}
\begin{prop}
  Let $A$ be a contractive operator on a finite-dimensional normed vector space. Then,
  $(I-A)\sum_{n=0}^\infty A^n=I$.
\end{prop}
\begin{defn}
    Let $X$ and $Y$ be metric spaces. We define $Y^X$ as the space of
    functions $f:X\to Y$, and give it the \emph{uniform metric}
    $\|f\|=\sup_x\max(1,f(x))$.
\end{defn}
\begin{prop}
    Under this definition, $Y^X$ is a normed vector space.
\end{prop}
\begin{prop}
    If $Y$ is complete, then so is $Y^X$.
\end{prop}
\begin{defn}
    Suppose $\{f_n\}$ is a sequence in $Y^X$ such that $\{f_n(x)\}$
    converges for every $x$. We then say that $\{f_n\}$ \emph{converges
    pointwise} to the function $f(x)=\lim f_n(x)$.

    If $f=\lim f_n$, we say the sequence \emph{converges uniformly} to $f$.
\end{defn}
\begin{defn}
  Let $f:X\to Y$ be a function, where $X$ and $Y$ are metric spaces.
  Let $p$ be a limit point of $X$. We say that
  \[\lim_{x\to p}f(x)=q\] if for every sequence $\{x_n\}$ which converges
  to $p$ but does not contain $p$, $f(x_n)$ converges to $q$.
\end{defn}
\begin{prop}
    Assume $\{f_n\}$ converges uniformly and $\lim_{t\to x}f_n(t)$ is defined for
    all $n$. Then, $\lim_{t\to x}\lim_{n\to\infty}
    f_n(x)=\lim_{n\to\infty}\lim_{t\to x}f_n(x)$.
\end{prop}
\begin{prop}
    If $X$ and $Y$ are compact, then every sequence in $Y^X$ has a subsequence
    that converges pointwise.
\end{prop}
\begin{cor}
    The limit of a uniformly convergent sequence of continuous functions is
    continous.
\end{cor}
\begin{defn}
    A sequence $\{f_n\}\in Y^X$ is said to be \emph{equicontinuous} if for
    every $\varepsilon>0$ there exists a $\delta>0$ such that if $d(a,b)<\delta$
    then $d(f_n(a),f_n(b))<\varepsilon$ for all $n$.
\end{defn}
\begin{prop}
    If $X$ is compact, all $\{f_n\}$ are continuous, and
    $\{f_n\}$ converges uniformly, then $\{f_n\}$ is
    equicontinuous.
\end{prop}
\begin{prop}
    If $X$ and $Y$ are compact, then every equicontinuous sequence in
    $Y^X$ has a convergent subsequence.
\end{prop}
\begin{defn}
    Let $A\subseteq \Rr^K$ be an algebra. If for
    each $x_1,x_2\in K$ there is a function $f\in A$ such that $f(x_1)\ne
    f(x_2)$, then $A$ is said to \emph{separate points}. If for each $x\in
    K$ there is an $f\in A$ such that $f(x)\ne 0$, then $A$ is said to
    \emph{vanish at no point}.
\end{defn}
\begin{thm}[Stone-Weierstrass]
    Let $A\subseteq \Rr^K$ be an algebra of continuous functions on a
    compact set $K$ which separates points and vanishes at no point. Then the
    closure of $A$ is the set of real continuous functions on $K$.
\end{thm}
\begin{cor}
    Let $A$ be an algebra of complex continuous functions
    on a compact set $K$, such that $A$ separates points and vanishes at no
    point. If $A$ is closed under complex conjugation, the closure of $A$ is the
    set of complex continuous functions on $K$.
\end{cor}
\begin{cor}[Weierstrass]
    Every continuous function $f:D\to\Cc$, where $D$ is a compact subset of
    $\Rr$, is a limit of a sequence of polynomials.
\end{cor}
\begin{defn}
    An \emph{analytic function} is a function of the form \[f(x)=\sum c_n
    (x-a)^n,\] where the radius of convergence is positive (and possibly
    infinite).
\end{defn}
\begin{prop}
    If $\varepsilon>0$ and $f$ is analytic with radius of convergence $R$, then
    $\sum c_n (x-a)^n$ converges uniformly for $x\in B_{R-\varepsilon}(a)$.
\end{prop}
\begin{thm}[Abel]
    An analytic function is continuous.
\end{thm}
\begin{cor}
    Assuming all three sums converge,
    \[\sum_n\sum_k a_k b_{n-k}=\left(\sum a_n\right)\left(\sum b_n\right).\]
\end{cor}
\begin{thm}[Taylor]
    If $f$ is analytic with radius $R$ around $0$, and $a<R$, then $f$ is
    analytic with radius at least $|R|-|a|$ around $a$.
\end{thm}
\begin{prop}
    If two analytic functions are equal on a set $S$ which
    has a limit point, then they are equal.
\end{prop}
\begin{defn}
    If $A$ is a complete normed commutative algebra over $\Rr$, we define the \emph{exponential}
    \[\exp(z)=\sum_{n=0}^\infty\frac{z^n}{n!}.\]
\end{defn}
\begin{prop}
    We have $\exp(z+w)=\exp(z)\exp(w)$
    and $\exp(a/b)=\sqrt[b]{e^a}$ for any integer $a$ and positive integer $b$.
\end{prop}
\begin{prop}
  Let $A$ be an operator on a complex finite-dimensional normed vector space.
  The eigenvalues of $\exp A$, together with their multiplicities, are
  equal to the exponentials of the eigenvalues of $A$.
\end{prop}
\begin{prop}
  Let $A$ and $B$ be operators on a finite-dimensional normed vector space such
  that $AB=BA$. Then, $\exp(A+B)=\exp A\exp B$.
\end{prop}
\begin{prop}
    The function $\exp|_\Rr$ has image $\Rr^+$ and is strictly increasing.
\end{prop}
\begin{defn}
    We define the \emph{natural logarithm} $\ln:\Rr^+\to\Rr$ as the left-inverse
    of $\exp|_\Rr$.
\end{defn}
\begin{prop}
    The function $\exp(ix)|_\Rr$ attains exactly the values on the unit circle.
\end{prop}
\begin{defn}
    We let $\pi$ be the smallest positive real such that $\exp(i\pi)=-1$. We let
    $\cos(x)=\frac{\exp(ix)+\exp(-ix)}2$,
    $\sin(x)=\frac{\exp(ix)-\exp(-ix)}{2i}$, and
    $\tan(x)=\frac{\sin(x)}{\cos(x)}$.
\end{defn}
\begin{prop}
    The functions $\sin$ and $\cos$ have image $[-1,1]$; the function $\tan$ has
    image $\Rr$. The functions $\sin|_{[-\pi/2,\pi/2]}$, $\cos|_{[0,\pi]}$ and
    $\tan|_{(-\pi/4,\pi/4)}$ have the same images and are injective.
\end{prop}
\begin{defn}
    The left-inverses of these three functions are denoted $\arcsin$,
    $\arccos$ and $\arctan$.
\end{defn}
\subsection*{References}
\begin{itemize}
    \item Rudin, \emph{Principles of Mathematical Analysis}
    \item Tao, \emph{Analysis II}
\end{itemize}

\documentclass{amsbook}
\usepackage{amsmath,amssymb,amsthm,tcolorbox,hyperref,booktabs,epigraph}
\hypersetup{colorlinks=true}
\theoremstyle{plain}
  \newtheorem{thm}{Theorem}
  \newtheorem{lem}[thm]{Lemma}
  \newtheorem{prop}[thm]{Proposition}
  \newtheorem{cor}[thm]{Corollary}
  \newtheorem{axiom}[thm]{Axiom}
\theoremstyle{definition}
  \newtheorem{defn}[thm]{Definition}
  \newtheorem{eg}[thm]{Example}
\theoremstyle{remark}
  \newtheorem{rem}[thm]{Remark}
  \newtheorem{note}[thm]{Note}
  \newtheorem{case}[thm]{Case}
  \newtheorem{claim}[thm]{Claim}
\newcommand{\LHS}{\text{LHS}}
\newcommand{\RHS}{\text{RHS}}
\newcommand{\Zz}{\mathbb{Z}}
\newcommand{\Qq}{\mathbb{Q}}
\newcommand{\Rr}{\mathbb{R}}
\newcommand{\Cc}{\mathbb{C}}
\newcommand{\Ff}{\mathbb{F}}
\newcommand{\lcm}{\mathrm{lcm}}
\title{Learning Uni Maths}
\author{gispisquared}
\begin{document}
\maketitle
\epigraph{If only I had the theorems! Then I should find the proofs easily
enough.}{Bernhard Riemann}
\tableofcontents
\chapter{Set Theory}
\begin{axiom}[Existence]
  There exists a set.
\end{axiom}
\begin{rem}
  This is implied by the Axiom of Infinity; however, we include it here so that
  we may define the empty set.
\end{rem}
\begin{defn}
  A \emph{sentence} is made by combining assertions of belonging (eg $x\in A$)
  and/or assertions of equality (eg $A=B$) using the usual logical operators:
  \emph{and, or, not, implies, if and only if, there exists, for all}.
\end{defn}
\begin{defn}
  Let $A$ and $B$ be sets.
  If every element of $A$ is an element of $B$, we say that $A$ is a
  \emph{subset} of $B$, denoted $A\subseteq B$.
\end{defn}
\begin{prop}
  If $A\subseteq B$ and $B\subseteq C$ then $A\subseteq C$.
\end{prop}
\begin{axiom}[Extensionality]
  $A=B$ iff $A\subseteq B$ and $B\subseteq A$.
\end{axiom}
\begin{axiom}[Specification]
  For every set $A$ and every sentence $S(x)$ there is a set $B$ whose elements
  are exactly those elements $x$ of $A$ for which $S(x)$ holds.
\end{axiom}
\begin{defn}
  We notate this set $B$ by $\{x\in A: S(x)\}$.
\end{defn}
\begin{prop}
  There exists a unique set $X$ such that for any $x$, the sentence
  $x\in X$ is false.
\end{prop}
\begin{defn}
  We call this set the \emph{empty set}, notated $\emptyset$.
\end{defn}
\begin{prop}
  For every set $A$ there is a set $B$ such that $B\not\in A$.
\end{prop}
\begin{axiom}[Pairing]
  For any two sets $A$ and $B$ there is a set $X$ with $A\in X$ and $B\in X$.
\end{axiom}
\begin{prop}
  There is a unique set $Y$ such that for any $a$, $a$ is in $Y$ iff $a=A$ or $a=B$.
\end{prop}
\begin{defn}
  This set is called the \emph{unordered pair} formed by $A$ and $B$, denoted
  $\{A,B\}$.
\end{defn}
\begin{defn}
  The set $\{A,A\}$ is denoted $\{A\}$, and called the \emph{singleton} of
  $A$.
\end{defn}
\begin{axiom}[Union]
  For any set $X$ of sets there exists a set $Y$ such that for any $A$ in $X$,
  and any $a$ in $A$, $a$ is in $Y$.
\end{axiom}
\begin{prop}
  For a nonempty set $X$ of sets there is a unique set $Z$ such that $a$ is in $Z$ if and
  only if there exists an $A$ in $X$ such that $a$ is in $A$.
\end{prop}
\begin{defn}
  This set is called the \emph{union} of $X$, denoted
  $\bigcup X$.

  For two sets $A$ and $B$ we define $A\cup B=\bigcup \{A,B\}$.
\end{defn}
\begin{defn}
  Let $A$ and $B$ be sets.
  The \emph{intersection} of $A$ and $B$, notated $A\cap B$, is $\{x\in A:x\in
    B\}$.

  If $A\cap B=\emptyset$ then $A$ and $B$ are called \emph{disjoint}.
\end{defn}
\begin{prop}
  We have
  \begin{itemize}
    \item $A\cup\emptyset=A$,
    \item $A\cup B=B\cup A$ (commutative),
    \item $A\cup (B\cup C)=(A\cup B)\cup C$ (associative),
    \item $A\cup A=A$ (idempotent),
    \item $A\cup (B\cap C)=(A\cup B)\cap(A\cup C)$ (distributive),
    \item $A\subseteq B$ iff $A\cup B=B$,
    \item $A\cap\emptyset=A$,
    \item $A\cap B=B\cap A$ (commutative),
    \item $A\cap (B\cap C)=(A\cap B)\cap C$ (associative),
    \item $A\cap A=A$ (idempotent),
    \item $A\cap (B\cup C)=(A\cap B)\cup(A\cap C)$ (distributive),
    \item $A\subseteq B$ iff $A\cap B=A$.
  \end{itemize}
\end{prop}
\begin{prop}
  For every nonempty set $C$ of sets, there is a unique set $Y$ such that $x\in Y$ iff
  $x\in X$ for each $X$ in $C$.
\end{prop}
\begin{defn}
  This set $Y$ is called the \emph{intersection} of $C$, denoted $\bigcap C$.
\end{defn}
\begin{axiom}[Powers]
  For each set $X$ there is a set that contains all subsets of $X$.
\end{axiom}
\begin{prop}
  There is a unique set $Y$ such that $x\in Y$ iff $x\subseteq X$.
\end{prop}
\begin{defn}
  This set $Y$ is called the \emph{power set} of $X$, denoted $\mathcal P(X)$.
\end{defn}
\begin{defn}
  The \emph{ordered pair} of $a$ and $b$ is the set defined as
  \[(a,b)=\{\{a\},\{a,b\}\}.\]
\end{defn}
\begin{prop}
  For any $a,b,c,d$, we have $(a,b)=(c,d)$ iff $a=c$ and $b=d$.
\end{prop}
\begin{defn}
  Let $A$ and $B$ be sets. The \emph{Cartesian product} $A\times B$ is
  \[\{(x,y): x\in A,\ y\in B\}.\]
\end{defn}
\begin{prop}\label{prop:1:orderedpairsinprod}
  For any set $R$ of ordered pairs there are sets $A$ and $B$ such that
  $R\subseteq A\times B$.
\end{prop}
\begin{defn}
  A \emph{binary relation} $R$ over sets $A$ and $B$ is a subset of $A\times
  B$. If $(a,b)$ is in $R$ we write $aRb$.

  If $A=B$ then we call it a \emph{binary relation over} $A$.
\end{defn}
\begin{defn}
  An \emph{equivalence relation} is a binary relation $\sim$ over $A$ such
  that
  \begin{itemize}
    \item $a\sim a$ (reflexive),
    \item $a\sim b\iff b\sim a$ (symmetric), and
    \item if $a\sim b$ and $b\sim c$ then $a\sim c$ (transitive).
  \end{itemize}
  The \emph{equivalence class} of $a$ under $\sim$ is
  \[[a]=\{x\in A:x\sim a\}.\]
\end{defn}
\begin{defn}
  A \emph{partition} of a set $A$ is a disjoint set of subsets of $A$ whose
  union is $A$.

  A partition $X$ of $A$ \emph{induces} a relation $\sim$, where $a\sim b$ iff
  $a$ and $b$ belong to the same element of $X$.
\end{defn}
\begin{prop}
  The set of equivalence classes of an equivalence relation exists and 
  is a partition.
\end{prop}
\begin{defn}
  This partition is called the partition \emph{induced} by the equivalence
  relation $\sim$.
\end{defn}
\begin{prop}
  The equivalence relation induced by a partition induces that partition; the
  partition induced by an equivalence relation induces that relation.
\end{prop}
\begin{defn}
  For any set $X$ we define $X^+=X\cup\{X\}$.
\end{defn}
\begin{axiom}[Infinity]
  There exists a set $S$ containing $\emptyset$ and containing $X^+$ for
  every $X$ in $S$.
\end{axiom}
\begin{prop}
  There exists a unique set $\omega$ which is a subset of all such sets $S$.
\end{prop}
\begin{prop}
  For any $a,b\in\omega$, exactly one of $a\in b,\ a=b,\ b\in a$ is true.
\end{prop}
\begin{prop}
  For any $a\in\omega$ and any $b\in a$, $b\subseteq a$.
\end{prop}
\begin{defn}
  A \emph{function} $f:A\to B$ is a relation $f$ over $A$ and $B$ such that
  for each $a\in A$ there is exactly one $b\in B$ such that $afb$. We usually
  write this as $f(a)=b$.

  A function $f$ is \emph{injective} if for each $b$ in $B$, there is at most one
  $a$ in $A$ such that $f(a)=b$. It is \emph{surjective} if for each $b$ in
  $B$ there is at least one $a$ in $A$ such that $f(a)=b$. A function which is
  both injective and surjective is \emph{bijective}.
\end{defn}
\begin{thm}[Recursion theorem]
  If $a$ is an element of a set $X$, and if $f:X\to X$ is a function, then there
  is a function $g:\omega\to X$ such that $u(0)=a$ and $u(n^+)=f(u(n))$ for all
  $n$ in $\omega$.
\end{thm}
\begin{axiom}[Substitution]
  If $S(a,b)$ is a sentence such that for each $a$ in a set $A$ there exists a
  set $B_a$ such that $b\in B_a\iff S(a,b)$, then there exists a function $F$ with
  domain $A$ such that $F(a)=B_a$ for each $a$ in $A$.
\end{axiom}
\begin{axiom}[Foundation]
  Every set $X$ contains a set $Y$ such that $X$ and $Y$ are disjoint.
\end{axiom}
\begin{axiom}[Choice]
  Let $X$ be a set of sets whose members are all nonempty. Then there exists a
  function $f:X\to\bigcup X$ such that $f(Y)\in Y$ for all $Y\in X$.
\end{axiom}
\begin{defn}
  A \emph{partial order} is a binary relation $\le$ on a a set $A$ such that
  \begin{itemize}
    \item $a\le a$ (reflexive),
    \item if $a\le b$ and $b\le a$ then $a=b$ (antisymmetric), and
    \item if $a\le b$ and $b\le c$ then $a\le c$ (transitive).
  \end{itemize}
  We define $a<b$ if $a\le b$ and $a\ne b$.

  If for all $a$ and $b$ we have $a\le b$ or $b\le a$ (strongly connected),
  then $\le$ is a \emph{total order}. 

  A \emph{chain} is a totally ordered subset of a partially ordered set.
\end{defn}
\begin{defn}
  If $X$ is a partially ordered set, and if $a\in X$, the set $s(a)=\{x\in
    X:x<a\}$ is called the \emph{initial segment} determined by $a$.
\end{defn}
\begin{defn}
  Two partially ordered sets $X$ and $Y$ are \emph{similar} if there is a
  bijection $f:X\to Y$ such that $a\le b\iff f(a)\le f(b)$. This bijection is
  called a \emph{similarity}.
\end{defn}
\begin{defn}
  Let $S$ be a subset of a partially ordered set $A$, and let $a$ be an element
  of $A$. If $s\le a$ for every $s$ in $S$, then we call $a$ an \emph{upper
    bound} of $S$. If $a\le s$ for every $s$ in $S$, then we call $a$ a
    \emph{lower bound} of $S$. If $a$ is an upper bound of $S$ and a lower
    bound of the set of upper bounds of $S$, then we call $a$ a \emph{least
      upper bound} of $S$.
\end{defn}
\begin{defn}
  A \emph{well-order} on $A$ is a total order $\le$ on $A$ such that every
  nonempty subset $S$ of $A$ has an element $a$ which is a lower bound for $S$.
  The set $A$ together with the relation $\le$ is then called \emph{well-ordered}.
\end{defn}
\begin{prop}
  If two well-ordered sets are similar, then the similarity is unique.
\end{prop}
\begin{thm}
  If $X$ and $Y$ are well-ordered, then either $X$ and $Y$ are similar, or one
  is similar to an initial segment of the other.
\end{thm}
\begin{defn}
  An \emph{ordinal number} is a well-ordered set $\alpha$ such that for any
  $\xi\in\alpha$ we have $s(\xi)=\xi$.
\end{defn}
\begin{prop}
  $\omega$ is an ordinal number.
\end{prop}
\begin{prop}
  If $\alpha$ is an ordinal number then so is $\alpha^+$, and so is any element
  of $\alpha$.
\end{prop}
\begin{thm}
  If two ordinal numbers are similar, then they are equal.

  Otherwise, one is an element of the other.
\end{thm}
\begin{prop}
  If a set $\alpha$ can be well-ordered such that it is an ordinal, then the
  ordering is unique.
\end{prop}
\begin{prop}
  Every well-ordered set is similar to a unique ordinal number.
\end{prop}
\begin{prop}
  There is no set of all ordinal numbers.
\end{prop}
\begin{thm}[Zorn's Lemma]
  Suppose a partially ordered set $P$ has the property that every chain in $P$
  has an upper bound in $P$. Then there is an element $a\in P$ such that the
  only upper bound for $\{a\}$ is $a$.
\end{thm}
\begin{thm}[Well-Ordering Theorem]
  Every set has a well-ordering.
\end{thm}
\begin{defn}
  Two sets $A$ and $B$ are said to have the same \emph{cardinality} (written
  $|A|=|B|$) if there is a bijection $f:A\to B$.

  A set $A$ has cardinality at most the cardinality of $B$ ($|A|\le|B|$) if
  there is an injection $f:A\to B$.

  A set $A$ has cardinality less than the cardinality of $B$ ($|A|<|B|$) if
  $|A|\le|B|$ and $|A|\ne|B|$.
\end{defn}
\begin{thm}
  If $|A|\le|B|$ and $|B|\le|A|$ then $|A|=|B|$.
\end{thm}
\begin{thm}
  For any set $A$, $|\mathcal P(A)|>|A|$.
\end{thm}
\begin{defn}
  A \emph{cardinal number} is an ordinal number $\alpha$ such that for any
  ordinal number $\beta$ with $|\alpha|=|\beta|$ we have $\alpha\subseteq\beta$.
\end{defn}
\begin{prop}
  For any set $S$, there is a unique cardinal number $\alpha$ with
  $|\alpha|=|S|$.
\end{prop}
\begin{defn}
  For these sets $S$ and $\alpha$ we define $|S|=\alpha$.
\end{defn}
\begin{defn}
  A set $A$ is said to be \emph{finite} if $|A|\in\omega$, and \emph{infinite}
  otherwise.
\end{defn}
\begin{prop}
  A set is infinite if and only if it has the same cardinality as some proper
  subset.
\end{prop}
\begin{defn}
  An infinite set $A$ is said to be \emph{countable} if $|A|=\omega$, and
  \emph{uncountable} otherwise.
\end{defn}
\begin{prop}
  A countable set does not have any uncountable subsets. An uncountable set has
  a countable subset.
\end{prop}
\subsection*{References}
\begin{itemize}
  \item \emph{Naive Set Theory}, Halmos
  \item \emph{Set Theory}, Jech
\end{itemize}

\chapter{Number Systems}
  \begin{defn}
    A \emph{binary operation} on $A$ is a function $\cdot:A\times A\to A$. We
    usually write $\cdot(a,b)=c$ as $a\cdot b=c$.

    It is \emph{associative} if $(a\cdot b)\cdot c=a\cdot(b\cdot
    c)$ for any $a,\ b,\ c$ in $A$.

    It is \emph{commutative} if $a\cdot b=b\cdot a$ for any $a,\ b$ in $A$.
  \end{defn}
  \begin{defn}
    A \emph{monoid} is an ordered pair $(A,\cdot)$ of a set $A$ and an
    associative binary
    operation $\cdot$ on $A$ such that there exists an element $1$, called the
    \emph{identity}, such that $a\cdot 1=1\cdot a=a$ for all $a$.
  \end{defn}
  \begin{rem}
    There are two main notations for monoid-type structures. These are
    \begin{itemize}
      \item 
        Multiplicative notation, in which the operation is notated $a\cdot b$ or
        simply $ab$, and the identity element is $1$; and
      \item Additive notation, in which the operation is notated $a+b$ and the
        identity element is $0$.
    \end{itemize}
  \end{rem}
  \begin{defn}
    A \emph{group} is a monoid $(A,\cdot)$ such that for each element $a$ of $A$
    there is an element $b$ of $A$ such that
    $ab=1=ba$.
    
    A group is \emph{abelian} if the operation is commutative.
  \end{defn}
  \begin{prop}
    If $ab=ba=1$ and $ac=1$ or $ca=1$ then $b=c$.
  \end{prop}
  \begin{defn}
    The element $b$ of $A$ such that $ab=ba=1$ is called the \emph{inverse} of $a$.
    In multiplicative notation, the inverse of $a$ is notated $a^{-1}$.
    In additive notation, the inverse of $a$ is notated $-a$.
  \end{defn}
  \begin{rem}
    We often define $\frac ab=ab^{-1}$ in multiplicative notation, and
    $a-b=a+(-b)$ in additive notation.
  \end{rem}
  \begin{defn}
    A \emph{ring} is an ordered triple $(A,+,\cdot)$ such that $(A,+)$ is an
    abelian group, $(A\setminus \{0\},\cdot)$ is a monoid, and the
    \emph{distributive laws} hold:
    \[a\cdot(b+c)=ab+ac\quad\text{and}\quad (a+b)\cdot c=ac+bc.\]
    It is \emph{commutative} if $\cdot$ is commutative.

    It is \emph{ordered} if there is a total order $\le$ on $A$ satisfying
    \begin{itemize}
      \item if $a\le b$ then $a+c\le b+c$, and
      \item if $0\le a$ and $0\le b$ then $0\le ab$.
    \end{itemize}
  \end{defn}
  \begin{defn}
    A \emph{field} is a commutative ring $(A,+,\cdot)$ such that $(A\setminus
    \{0\},\cdot)$ is a group.

    An \emph{ordered field} is a field that is also an ordered ring.
  \end{defn}
  \begin{defn}
    In an ordered ring $R$, the \emph{absolute value} $|a|$ of an element $a$
    of $R$ is $a$ if $0\le a$, otherwise $-a$.
  \end{defn}
  \begin{prop}
    $|a+b|\le|a|+|b|$.
  \end{prop}
  \begin{defn}
    Let $X$ and $Y$ be similar well-ordered sets, and let $A$ and $B$ be the
    least elements of $X$ and $Y$ respectively. Assume that all other elements
    of $X$ and $Y$ are operations on $A$ and $B$ respectively, and let $f$ be
    the similarity between $A$ and $B$.

    A function $\varphi:A\to B$ is said
    to be a \emph{homomorphism} if
    for every $a,b\in A$ and every $\cdot\in X\setminus \{A\}$ we
    have \[\varphi(a\cdot b)=\varphi(a) f(\cdot) \varphi(b).\]

    An \emph{isomorphism} is a bijective homorphism.

    If there exists an isomorphism from $A$ to $B$, then we say $A$ and $B$ are
    \emph{isomorphic}.
  \end{defn}
  \begin{prop}
    The property of being isomorphic is reflexive, symmetric and transitive.
  \end{prop}
  \begin{rem}
    We don't say that isomorphism is an equivalence relation, since it would
    imply there exists a set of all well-ordered sets of this type.

    Such a set does not exist because if it did it would contain
    $(S,\mathrm{Id}_S)$ for each set $S$. This would imply the existence of a
    set of all sets.
  \end{rem}
  \begin{thm}
    There exists a unique ordered ring $\Zz$ (up to isomorphism) such that
    $\{x\in\Zz:x\ge 0\}$ is well-ordered.

    $\Zz$ is commutative.
  \end{thm}
  \begin{defn}
    We call this set $\Zz$ the \emph{integers}. The \emph{non-negative
    integers} $\Zz_{\ge 0}$ are $\{n\in\Zz: n\ge 0\}$. The \emph{positive
    integers} $\Zz^+$ are $\Zz_{\ge 0}\setminus \{0\}$.
  \end{defn}
  \begin{rem}
    As a byproduct of our construction, we get a canonical bijection between
    $\omega$ and $\Zz_{\ge 0}$.
    In particular, the cardinality of a finite set is a nonnegative integer.
  \end{rem}
  \begin{rem}
    We avoid use of the term \emph{natural numbers}, and the symbol $\mathbb N$,
    since
    some use them to mean the positive integers and others use them to mean the
    nonnegative integers.
  \end{rem}
  \begin{prop}
    Every ordered ring contains a unique subring isomorphic to $\Zz$.
  \end{prop}
  \begin{defn}
    In $\Zz\times\Zz^+$, we define the operations
    \[(a,b)+(c,d)=(ad+bc,bd),\qquad (a,b)(c,d)=(ac,bd).\]
    We also define an equivalence relation $\sim$ where
    $(a,b)\sim (c,d)\iff ad=bc$.

    We define the \emph{rational numbers} $\Qq$ as the partition
    of $\Zz\times\Zz^+$ induced by this equivalence relation, with
    $[(a,b)]+[(c,d)]=[(ad+bc,ac+bd)]$ and $[(a,b)]\cdot [(c,d)]=[(ac,bd)]$.
  \end{defn}
  \begin{prop}
    The relation $\sim$ is an equivalence relation. Moreover, the operations $+$
    and $\cdot$ are independent of the representatives of each equivalence
    class. With these operations, $\mathbb Q$ is a field.
  \end{prop}
  \begin{prop}
    Every ordered field contains a unique subfield isomorphic to $\Qq$.
  \end{prop}
  \begin{defn}
    A partially ordered set $S$ is \emph{complete} if every nonempty subset that has
    an upper bound in $S$ has a least upper bound in $S$.
  \end{defn}
  \begin{prop}
    Let $S$ be a complete partially ordered set. Every nonempty subset that
    has a lower bound in $S$ has a greatest lower bound in $S$.
  \end{prop}
  \begin{thm}
    There exists a unique complete ordered field, up to isomorphism.
  \end{thm}
  \begin{defn}
    We call this field $\Rr$.
  \end{defn}
  \begin{defn}
    We define $\Qq_{\ge 0},\ \Qq^+,\ \Rr_{\ge 0},\ \Rr^+$ in an analogous way to
    $\Zz_{\ge 0}$ and $\Zz^+$.
  \end{defn}
  \begin{defn}
    We define the \emph{complex numbers} $\Cc$ as $\Rr^2$, with the operations
    \[(a,b)+(c,d)=(a+c,b+d),\qquad (a,b)\cdot(c,d)=(ac-bd,ad+bc).\]

    We usually write $(a,b)$ as $a+bi$. We define the \emph{conjugate} of $a+bi$
    to be $\overline{a+bi}=a-bi$.
  \end{defn}
  \begin{prop}
    $\Cc$ is a field under these operations.
  \end{prop}
  \begin{prop}
    There are unique homomorphisms $\Zz\to\Qq$, $\Qq\to\Rr$ and $\Qq\to\Cc$.
    There is also an isomorphism $\Rr\to\{x\in\Cc: x=\overline x\}$.
  \end{prop}
  \begin{rem}
    Because of this, we usually take $\Zz\subseteq\Qq\subseteq\Rr\subseteq\Cc$.
  \end{rem}
  \begin{prop}
    Let $a\in\Cc$. Then, $a\overline{a}\in\mathbb R_{\ge 0}$.
  \end{prop}
  \begin{prop}
    Let $b\in\Rr_{\ge 0}$. There exists a unique $x\in\Rr_{\ge 0}$ such that
    $x\cdot x=b$. 
  \end{prop}
  \begin{defn}
    We call $x$ the \emph{square root} of $b$, denoted $\sqrt b$.

    We call $\sqrt{a\overline a}$ the \emph{modulus} of $a$, denoted $|a|$.
  \end{defn}
  \begin{prop}
    $|a+b|\le|a|+|b|$.
  \end{prop}
  \begin{thm}
    $|\Zz^+|=|\Zz_{\ge 0}|=|\Zz|=|\Qq|=\omega$, but $|\Rr|=|\Cc|=|\mathcal
    P(\omega)|$.
  \end{thm}
  \begin{defn}
    A \emph{polynomial} over $S$ is an expression of the form
    \[p(z)=a_0+a_1z+a_2 z^2+\cdots+a_m z^m,\]
    for some integer $m$ and coefficients $a_i\in S$.

    We say the \emph{degree} of $p$ is $d$, where $d$ is the largest integer
    such that $a_d\ne 0$. If no such $d$ exists, the degree is $-\infty$.
  \end{defn}
  \begin{prop}[Division Algorithm]
    Suppose $p$ and $s$ are polynomials over a field $\Ff$
    with $s\ne 0$. There exist unique
    polynomials $q,r$ over $\Ff$ such that $p=sq+r$ and $\deg r<\deg s$.
  \end{prop}
  \begin{defn}
    A number $r\in\Ff$ is a root of a polynomial $p$ over $\Ff$ if $p(r)=0$.
  \end{defn}
  \begin{prop}
    A polynomial over a field $\Ff$ has at most as many roots as its degree.
  \end{prop}
  \begin{thm}[Fundamental Theorem of Algebra]
    Every nonconstant polynomial over $\Cc$ has a root.
  \end{thm}
  \begin{prop}
    If $p$ is a polynomial over $\Cc$ then it has a unique factorisation of the
    form $p(z)=c(z-r_1)\cdots(z-r_m)$, where all constants are complex numbers.
  \end{prop}
  \begin{prop}
    If $p$ is a polynomial over $\Rr$ then it has a unique factorisation of the
    form
    \[p(x)=c(x-r_1)\cdots(x-r_m)(x^2+b_1x+c_1)\cdots(x^2+b_n x+c_n),\]
    where all constants are real numbers such that $b_j^2<4c_j$ for each $j$.
  \end{prop}

\section{Linear Algebra}
\begin{defn}
  Let $\Ff$ be a field. A \emph{vector space over} $\Ff$ is an abelian group
  $V$ (of \emph{vectors})
  together with a function $\cdot:\Ff\times V\to V$ (\emph{scalar
    multiplication}) such that
  \begin{itemize}
    \item $a(b v)=(ab) v$ (compatible),
    \item $1 v= v$ (identity), and
    \item $a( u+ v)=a u+a v$ and $(a+b)
      v=a v+b v$ (distributive).
  \end{itemize}
\end{defn}
\begin{defn}
  Let $S$ be a subset of $V$. A \emph{linear combination} of elements of $S$ is
  a vector of the form \[\sum_{i=1}^n a_i s_i,\] where each $s_i$ is a
  distinct element of $S$.

  A \emph{basis} of a vector space $V$ is a set $S\subseteq V$ such that each
  element of $V$ can be uniquely represented as a linear combination of elements
  of $S$.
\end{defn}
\begin{rem}
  For an infinite-dimensional vector space, there are multiple different notions
  of a basis. This one is usually called a \emph{Hamel basis}.
\end{rem}
\begin{thm}
  Let $V$ be a vector space.
  \begin{itemize}
    \item $V$ has a Hamel basis.
    \item Any two Hamel bases of $V$ have the same cardinality.
  \end{itemize}
\end{thm}
\begin{defn}
  The \emph{dimension} of $V$ is the cardinality of a basis of $V$. If $\dim V$
  is an integer, $V$ is said to be \emph{finite-dimensional}; otherwise, it is
  \emph{infinite-dimensional}.
\end{defn}
\begin{defn}
  A \emph{subspace} $W$ of $V$ is a nonempty subset of $V$ which is also a
  vector space over $\Ff$.
\end{defn}
\begin{prop}
  A subset $W$ of $V$ is a subspace iff the following conditions hold:
  \begin{itemize}
    \item $W$ is nonempty;
    \item $u,v\in W$ implies $u+v\in W$ (closed under addition); and
    \item if $a\in\Ff$ and $u\in W$ then $au\in W$ (closed under scalar
      multiplication).
  \end{itemize}
\end{prop}
\begin{prop}
  The intersection of any collection of subspaces of $V$ is again a subspace of
  $V$.
\end{prop}
\begin{defn}
  The \emph{span} of a subset $S$ of $V$ is the intersection of all subspaces of
  $V$ which contain $S$.
\end{defn}
\begin{prop}
  The span of $S$ is the set of all linear combinations of $S$.
\end{prop}
\begin{defn}
  Given two subspaces $X$ and $Y$ of $V$, their \emph{sum} $X+Y$ is the
  intersection of all subspaces of $V$ which contain both $X$ and $Y$.

  If $X+Y=V$ and $X\cap Y=\{0\}$ then $X$ is said to be a \emph{complement} of
  $Y$.
\end{defn}
\begin{prop}
  $X+Y=\{x+y:x\in X,y\in Y\}$.
\end{prop}
\begin{defn}
  A subset $S$ of $V$ is \emph{linearly independent} if any linear combination
  of elements of $S$ that produces $ 0$ has all coefficients equal to $0$.
  Otherwise, it is \emph{linearly dependent}.
\end{defn}
\begin{prop}
  A subset $S$ of $V$ is a \emph{basis} iff it is linearly independent and its
  span is $V$.
\end{prop}
\begin{prop}
  Let $V$ be finite-dimensional with dimension $d$.
  Let $S$ be a set of vectors in $V$ with $|S|=d$. Then $S$ is linearly
  independent iff it spans $V$.
\end{prop}
\begin{defn}
  A \emph{linear map}, or \emph{linear transformation}, from $V$ to $W$ is a group homomorphism
  $T:V\to W$ such that $T(\lambda v)=\lambda T(v)$ for all $\lambda\in\Ff$. A
  linear map from a vector space to itself is an \emph{operator}.

  The \emph{product} of linear maps $S$ and $T$ is $ST=S\circ T$.
\end{defn}
\begin{prop}
  The set $\mathcal L(V,W)$ of linear maps from $V$ to $W$ is a vector space.
  Right-multiplication by a linear map $T:U\to V$ defines a linear map from $\mathcal
  L(V,W)$ to $\mathcal L(U,W)$. Left-multiplication by $T$ defines a linear map
  from $\mathcal L(W,U)$ to $\mathcal L(W,V)$.
\end{prop}
\begin{defn}
    An \emph{algebra} is a set $A$ over a field $K$ with operations of addition,
    multiplication and scalar multiplication which is both a vector space and a
    ring, such that multiplication is bilinear.
\end{defn}
\begin{prop}
    The set of operators from a vector space to itself is an algebra.
\end{prop}
\begin{defn}
    The \emph{null space} of a linear map $T$ is the subset of its domain that $T$
    maps to 0.
\end{defn}
\begin{prop}
    The null space and image of a linear map are both vector spaces.
\end{prop}
\begin{prop}
    The null space of a linear map is $\{0\}$ iff the map is injective.
\end{prop}
\begin{prop}
    If a linear map is injective, then its left inverse is linear. If a linear
    map is surjective, then it has a linear right inverse.
\end{prop}
\begin{prop}
    Let $V$ be finite-dimensional.
    A linear map $T:V\to V$ is injective iff it is surjective.
\end{prop}
\begin{defn}
  The \emph{product} of vector spaces is the Cartesian product, where addition
  and scalar multiplication are defined componentwise.
\end{defn}
\begin{prop}
  The product of a collection $S$ of vector spaces
  is a vector space whose dimension is the sum of the dimensions of the elements
  of $S$.
\end{prop}
\begin{cor}
  The product $\Ff^n=\Ff\times\Ff\times\cdots\times\Ff$ is a vector space over
  $\Ff$.
\end{cor}
\begin{prop}
  Suppose $U$ is a subspace of $V$. Define the relation $a\sim b\iff b-a\in V$.
  Then $\sim$ is an equivalence relation compatible with addition and scalar
  multiplication. The partition induced by this relation is a vector
  space. If $V$ is finite-dimensional, this vector space has dimension $\dim
  V-\dim U$.
\end{prop}
\begin{defn}
  This vector space is called the \emph{quotient space} of $V$ over $U$, denoted
  $V/U$.
\end{defn}
\begin{prop}
  Suppose $T$ is a linear transformation with domain $V$, and let $U$ be the
  null space of $T$.
  Then $T$ induces an isomorphism from $V/U$ to the image of $T$.
\end{prop}
\begin{cor}[Rank-Nullity]
  Let $V$ be finite-dimensional, and
  let $T:V\to W$ be a linear transformation. Then the null space of $T$ is a
  subspace of $V$, the image of $T$ is a subspace of $W$, and the sum of the
  dimensions of these two subspaces equals $\dim V$.
\end{cor}
\begin{defn}
  A \emph{linear functional} on $V$ is a linear map from $V$ to $\Ff$.

  The space of linear functionals on $V$ is the \emph{dual space} of $V$,
  denoted $V'$.
\end{defn}
\begin{prop}
  If $V$ is infinite-dimensional, $\dim V'>\dim V$.
\end{prop}
\begin{prop}
  If $v_1,\ldots,v_n$ is a finite basis of $V$, then there exists a basis of $n$
  elements $\varphi_j$ of $V'$, where $\varphi_j v_k$ is $1$ if $j=k$ and $0$
  otherwise.
\end{prop}
\begin{defn}
  This basis is called the \emph{dual basis} of
  $v_1,\ldots,v_n$.
\end{defn}
\begin{prop}
  If $V$ is finite-dimensional, then for every $z\in (V')'$ there is an $x\in V$
  such that for every $y\in V'$ we have $z(y)=y(x)$. The correspondence $x\to z$
  is an isomorphism.
\end{prop}
\begin{rem}
  Thus, $(V')'$ and $V$ are often identified for finite-dimensional vector
  spaces.
\end{rem}
\begin{defn}
  For $U\subseteq V$, the \emph{annihilator} of $U$ is
  \[U^0=\{\varphi\in V': \varphi(u)=0\ \forall u\in U\}.\]
\end{defn}
\begin{prop}
  $\dim U+\dim U^0=\dim V$.
\end{prop}
\begin{prop}
  $(U^0)^0=U$.
\end{prop}
\begin{prop}
  If $M$ and $N$ are complementary subspaces of $V$, then $M^0$ and $N^0$ are
  complementary subspaces of $V'$. The restriction $|_M$ is an isomorphism
  between $N^0$ and $M'$.
\end{prop}
\begin{defn}
  The \emph{dual map} of $T$ is the linear map $T':W'\to V'$ defined by
  $T'\varphi=\varphi T$ for each $\varphi\in W'$.
\end{defn}
\begin{prop}
    The image of $T'$ is the annihilator of the null space of $T$. The null
    space of $T'$ is the annihilator of the image of $T$.
\end{prop}
\begin{defn}
  Suppose $V$ and $W$ have finite bases $\{v_i\}_1^m$ and $\{w_i\}_1^n$
  respectively. The \emph{matrix} $A$ of $T$ with respect to these bases is
  defined by
  \[Tv_k=\sum_{i=1}^n A_{i,k}w_i.\]

  We also identify $1\times n$ and $n\times 1$ matrices with elements of
  $\Ff^n$.
\end{defn}
\begin{prop}
  This defines a bijection between the space of $m\times n$ matrices and
  $\mathcal L(\Ff^n,\Ff^m)$.
\end{prop}
\begin{defn}
  Thus, we identify the two, and can therefore talk of the image, null space,
  etc of a matrix.
  Matrix addition and multiplication are defined in the same way as addition and
  multiplication of linear transformations.
\end{defn}
\begin{defn}
  The \emph{rank} of a matrix is the dimension of its image.

  The \emph{transpose} of a matrix is the matrix obtained by swapping rows and
  columns: $A^T_{j,k}=A_{k,j}$.
\end{defn}
\begin{prop}
  Let $T:V\to W$ be a linear transformation, where $V$ and $W$ are
  finite-dimensional. Pick bases $\{v_i\}$ and $\{w_i\}$ 
  of $V$ and $W$. The matrix of $T'$ with respect
  to the dual bases of $\{w_i\}$ and $\{v_i\}$ 
  is the transpose of the matrix of $T$ with respect to $\{v_i\}$ and $\{w_i\}$.
\end{prop}
\begin{prop}
    The image of $A$ equals the image of $AA^T$.
\end{prop}
\begin{cor}
    The ranks of the matrices $A$, $A^T$, $AA^T$ and $A^T A$ are equal.
\end{cor}
\begin{defn}
  Let $A:U\to W$ and
  $B:V\to W$ be linear maps. We \emph{augment} $A$ with $B$ to get the linear
  map
  \[(A|B):U\times V\to W,\ (A|B)(x,y)=Ax+By.\]
\end{defn}
\begin{prop}
  For any $x:V\to U$ we have $Ax=B\iff (A|B)(x,-I)=0$.
\end{prop}
\begin{rem}
  Thus, to solve the linear system $Ax=B$ it suffices to find the null space of
  $(A|B)$. Notice also that the matrix of $(A|B)$ is simply the matrix formed by
  concatenating the matrices of $A$ and $B$.
\end{rem}
\begin{prop}
  Let $T$ and $S$ be linear maps from $V$ to $W$. The following are equivalent:
  \begin{itemize}
    \item The null spaces of $T$ and $S$ are the same.
    \item The images of $T'$ and $S'$ are the same.
    \item There is an invertible linear map $A:V\to V$ such that $AT=S$.
  \end{itemize}
\end{prop}
\begin{defn}
  Such linear maps are called \emph{equivalent}.
\end{defn}
\begin{defn}
  A linear transformation $T:V\to V$ is a \emph{projection} onto $U$ if $U$ is
  its image and $Tu=u$ for each $u\in U$.
\end{defn}
\begin{prop}
  Every linear transformation is equivalent to a projection.
\end{prop}
\begin{cor}
  Every linear transformation $T$ is a sum of $r$ transformations of rank one,
  where $r$ is the rank of $T$.
\end{cor}
\begin{defn}
  A \emph{pivot} is the first nonzero entry in a row of a matrix.

  A matrix is in \emph{row echelon form (REF)} if all rows consisting
  of only zeroes are at the bottom and the pivot of a nonzero row is strictly to
  the right of the pivot of the row above it.

  A matrix is in \emph{reduced row echelon form (RREF)} if it is in REF, all
  pivots are $1$, and each column containing a pivot has zeroes everywhere else
  in the column.
\end{defn}
\begin{prop}
  Every matrix is equivalent to a unique matrix in RREF\@.
\end{prop}
\begin{prop}[$LU$ Factorisation]
    If a matrix $A$ is square, then there are a permutation matrix $P$, an
    invertible lower-triangular matrix $L$ and a matrix $U$ in REF
    such that $PA=LU$.
\end{prop}
\begin{rem}
    The null space of a matrix in REF is easy to find by
    \emph{back-substitution}.
    Thus, to find the null space of a matrix, we factorise it into into
    $P^{-1}LU$ and find the null space of $U$. This process is known as
    \emph{Gaussian elimination}.
\end{rem}
\begin{prop}
  Let $T$ be a matrix which is equivalent to a matrix $S$ in REF\@. Then,
  \begin{itemize}
    \item The rows of $S$ with pivots form a basis for the span of the rows of $T$.
    \item Consider the columns of $S$ with pivots. The corresponding columns of
      $T$ form a basis for the span of the columns of $T$.
  \end{itemize}
\end{prop}
\begin{defn}
    Let $T:V\to V$ be a linear transformation. A subspace $U$ of $V$ is called
    \emph{invariant} under $T$ if $u\in U\implies Tu\in U$.
\end{defn}
\begin{defn}
  A nonzero vector $v\in V$ is called an \emph{eigenvector} of $T$ if there is
  some $\lambda\in\Ff$ such that $Tv=\lambda v$. We call $\lambda$ an
  \emph{eigenvalue} of $T$.
\end{defn}
\begin{prop}
  $\lambda$ is an eigenvalue of $T$ if and only if $T-\lambda I$ is not
  invertible.
\end{prop}
\begin{prop}
  Any set of eigenvectors of $T$ with distinct eigenvalues is linearly
  independent.
\end{prop}
\begin{prop}
  Suppose $T:V\to V$ is linear, $U$ is a subspace of $V$ invariant under
  $T$, and $\pi:V\to V/U$ is the natural projection. There is a linear map $T/U:V/U\to
  V/U$ such that $T/U\circ\pi=\pi\circ T$.
\end{prop}
\begin{defn}
    This map is the \emph{quotient map} $T/U$.
\end{defn}
\begin{defn}
  Suppose $T:V\to V$ is a linear transformation and
  \[p(z)=\sum a_i z^i,\] where each $a_i\in\Ff$. Then $p(T)=\sum a_i T^i$.
\end{defn}
\begin{thm}
  Every operator on a finite-dimensional nonzero complex vector space has an
  eigenvalue.
\end{thm}
\begin{defn}
  In defining the \emph{matrix} of an operator, we choose the same basis for the
  domain and codomain.
\end{defn}
\begin{prop}
  Suppose $V$ is a finite-dimensional vector space and $T:V\to V$ is an
  operator. Then $T$ has an upper-triangular matrix with respect to some basis
  of $V$.
\end{prop}
\begin{prop}
  Suppose $T:V\to V$ has an upper-triangular matrix with respect to some basis
  of $V$. Then the eigenvalues of $T$ are precisely the entries on the diagonal
  of that matrix.
\end{prop}
\begin{defn}
  An operator is \emph{diagonalisable} if it has a diagonal
  matrix with respect to some basis of the space.
\end{defn}
\begin{prop}
  Let $T:V\to V$ be an operator over a finite-dimensional vector space.
  Then $T$ is diagonalisable iff $V$ has a basis
  consisting of eigenvectors of $T$.
\end{prop}
\begin{defn}
  An \emph{inner product space} is a vector space $V$ over a field $\Ff$ which
  is either $\Rr$ or $\Cc$, together with a function
  $\langle\cdot,\cdot\rangle:V\times V\to\Ff$ satisfying
  \begin{itemize}
    \item $\langle  x, y\rangle=\overline{\langle 
      y, x\rangle}$ (conjugate
      symmetry)
    \item $\langle a x+b y, z\rangle=a\langle
      x, z\rangle+b\langle y, z\rangle$ (linearity in the
      first argument), and
    \item $\langle x, x\rangle=0\implies x= 0$.
  \end{itemize}
\end{defn}
\begin{prop}
  The dot product, defined by 
  \[(a_1,\ldots,a_n)\cdot(b_1,\ldots,b_n)=\sum a_i\overline{b_i},\]
  is an inner product over both $\Rr^n$ and $\Cc^n$.
\end{prop}
\begin{prop}[Cauchy-Schwarz]
  $|\langle u,v\rangle|\le \|u\|\|v\|$.
\end{prop}
\begin{defn}
  A \emph{normed vector space} is a vector space $V$ over $\Rr$ or $\Cc$
  on which there is a \emph{norm}: a function $\|\cdot\|:V\to\Rr$ satisfying
  \begin{itemize}
    \item $\| x\|\ge 0$, with $\|x\|=0\iff x=0$,
    \item $\|a x\|=|a|\| x\|$, and
    \item $\| x+ y\|\le\| x\|+\| y\|$ (the triangle
      inequality).
  \end{itemize}
\end{defn}
\begin{prop}
  If $V$ is an inner product space, then $\langle x, x\rangle$ is
  real for all $ x$.
  Moreover, $\| x\|=\sqrt{\langle x, x\rangle}$ is a norm
  on $V$.
\end{prop}
\begin{defn}
  Two vectors $ x$ and $ y$ are \emph{orthogonal} if $\langle
   x, y\rangle=0$.

  A set of vectors is \emph{orthonormal} if each vector in the set has norm 1
  and is orthogonal to all other vectors in the set.
\end{defn}
\begin{prop}[Gram-Schmidt]
  Suppose $V$ is finite-dimensional. Then every orthonormal list of vectors in
  $V$ can be extended to an orthonormal basis of $V$.
\end{prop}
\begin{rem}
  Thus, we may identify a finite-dimensional inner product space over $\Ff$ with
  $\Ff^n$ under the usual dot product.
\end{rem}
\begin{thm}[Schur]
  An operator over a finite-dimensional inner product space has an
  upper-triangular matrix with respect to an orthonormal basis of the space.
\end{thm}
\begin{thm}[Riesz Representation]
  Any linear functional $f$ on a finite-dimensional inner product space can be
  written as $f(x)=\langle x,v\rangle$ for some fixed vector $v$.
\end{thm}
\begin{rem}
  Thus, on a finite-dimensional inner product space we may canonically identify
  the dual space with the space itself.
\end{rem}
\begin{prop}
  Let $T:V\to W$ be linear. There exists a unique function $T^*:W\to V$ such
  that
  \[\langle Tv,w\rangle=\langle v,T^* w\rangle\] for every $v\in V$ and every
  $w\in W$. The function $T^*$ is linear, and we have $(T^*)^*=T$.
\end{prop}
\begin{defn}
  We call $T^*$ the \emph{adjoint} of $T$.
\end{defn}
\begin{prop}
    The images of $T$ and $TT^*$ are the same; the images of $T^*$ and $T^* T$
    are the same.
\end{prop}
\begin{cor}
    The ranks of $T$ and $T^*$ are the same.
\end{cor}
\begin{prop}
  Let $T:V\to W$ be linear, where $V$ and $W$ are real or complex vector spaces.
  Let $\{v_i\}$ be an orthonormal basis for $V$, and
  let $\{w_i\}$ be an orthonormal basis for $W$. Then, the matrix of $T^*$ with
  respect to $\{w_i\}$ and $\{v_i\}$ is the conjugate transpose of the matrix of
  $T$ with respect to $\{v_i\}$ and $\{w_i\}$.
\end{prop}
\begin{defn}
  Let $T$ be an operator. If $T^*=T$, then $T$ is \emph{self-adjoint}. If
  $TT^*=T^* T$, then $T$ is \emph{normal}.
\end{defn}
\begin{prop}
  Every eigenvalue of a self-adjoint operator is real.
\end{prop}
\begin{thm}[Spectral]
  Let $T:V\to V$ be normal, where $V$ is finite-dimensional. Then
  $T$ has a diagonal matrix with respect to some orthonormal basis of $V$.
\end{thm}
\begin{rem}
  Thus, we may write $T=UBU^*$, where $UU^*=U^*U=I$ and $B$ is diagonal.
\end{rem}
\begin{prop}
  If $A$ is normal, then $B$ commutes with $A$ iff $B$ commutes with $A^*$.
\end{prop}
\begin{defn}
  Let $U$ and $V$ be vector spaces over $\Ff$. A function $w:U\times V\to\Ff$
  is called a \emph{bilinear form} if $w(u,v_0)$ and $w(u_0,v)$ are linear for
  fixed $u_0$ and $v_0$.
\end{defn}
\begin{prop}
  The dimension of the space of bilinear forms on $U\times V$ is the product of
  the dimensions of $U$ and $V$.
\end{prop}
\begin{prop}
    For any two vector spaces $V$ and $W$ there is a unique (up to isomorphism)
    vector space $V\otimes W$ and a map $\otimes:V\times W\to V\otimes W$
    such that for any bilinear function $h:V\times W\to Z$ there is a linear
    function $\bar h:V\otimes W\to Z$ such that $h(v,w)=\bar h(v\otimes w)$.
\end{prop}
\begin{defn}
    The space $U\otimes V$ is the \emph{tensor product} of $U$ and $V$, and for
    $u\in U$ and $v\in V$ we call $u\otimes v$ the \emph{tensor product} of $u$
    and $v$.
\end{defn}
\begin{prop}
    If $X$ and $Y$ are bases in $U$ and $V$, then the set $\{x\otimes y:x\in
    X,y\in Y\}$ is a basis in $U\otimes V$.
\end{prop}
\begin{prop}
    Let $U$ and $V$ be finite-dimensional vector spaces.
    There is an isomorphism $f:\mathcal
    L(U)\otimes\mathcal L(V)\to\mathcal L(U\otimes V)$ satisfying
    \[f(A\otimes B)(u\otimes v)=Au\otimes Bv.\]
\end{prop}
\begin{defn}
    Thus, we identify these two spaces and speak of the \emph{tensor product} of
    two operators as an operator on the tensor product of the underlying spaces.
\end{defn}
\begin{prop}
  For each bilinear form $f$ on a finite-dimensional inner product space
  there is a unique linear map $A$
  such that $Q(x,y)=\langle Ax,y\rangle$ for all $x,y\in V$. The form $f$ is
  conjugate symmetric --- that is, $f(x,y)=\overline{f(y,x)}$ --- iff $A$ is
  self-adjoint.
\end{prop}
\begin{defn}
  A \emph{quadratic form} $Q$ on $V$ is defined by $Q(x)=f(x,x)$, where
  $f:V\times V\to\Ff$ is bilinear.
\end{defn}
\begin{prop}
  If $Q$ is a quadratic form on a complex vector space such that its image is
  contained in $\Rr$, then there exists a unique conjugate symmetric bilinear
  form $f$ such that $Q(x)=f(x,x)$.
\end{prop}
\begin{defn}
  Two self-adjoint linear maps $X$ and $Y$ are \emph{congruent} if there is some
  invertible linear map $S$ such that $X=SYS^*$.
\end{defn}
\begin{prop}
  If $Q=\langle Ax,x\rangle$ and $R(y)=\langle By,y\rangle$ are two quadratic
  forms on $V$, then there is an invertible linear map $L$ such that
  $Q(Lx)=R(x)$ iff $A$ and $B$ are congruent.
\end{prop}
\begin{prop}[Sylvester's Law of Inertia]
  If two diagonal matrices are congruent, the numbers of positive, negative, and
  zero entries are equal in each.
\end{prop}
\begin{defn}
  A quadratic form $Q(x)=f(x,x)$ on an inner product space, where $f$ is
  conjugate symmetric, is called
  \begin{itemize}
    \item \emph{Positive definite} if $Q(x)>0$ for all $x\ne 0$;
    \item \emph{Positive semidefinite} if $Q(x)\ge 0$ for all $x$;
    \item \emph{Negative definite} if $Q(x)<0$ for all $x\ne 0$;
    \item \emph{Negative semidefinite} if $Q(x)\le 0$ for all $x$; and
    \item \emph{Indefinite} otherwise.
  \end{itemize}
  A self-adjoint linear map $A$ is called \emph{positive definite} (etc) if the
  corresponding quadratic form $\langle Ax,x\rangle$ is positive definite (etc).
\end{defn}
\begin{prop}
  $T$ is positive semidefinite iff there exists an operator $R$ such that
  $T=R^*R$. The matrix $R$ may be taken to be upper triangular, and is
  invertible iff $T$ is positive definite.
\end{prop}
\begin{defn}
  An operator $F$ is a \emph{square root} of an operator $T$ if $R^2=T$.
\end{defn}
\begin{prop}
  Every positive semidefinite operator has a unique positive semidefinite square root.
\end{prop}
\begin{defn}
  If $T$ is a positive semidefinite operator, then $\sqrt T$ denotes the unique positive
  semidefinite square root of $T$.
\end{defn}
\begin{defn}
  Let $U$ be a finite-dimensional subspace of $V$. The \emph{orthogonal
  projection} of $V$ onto $U$ is the operator $P_U:V\to V$ defined by $P_U v=u$
  where $u\in U$ and $\langle v-u,x\rangle=0\ \forall x\in U$.
\end{defn}
\begin{prop}
  The orthogonal projection is well defined, and satisfies
  \[\|P_u v\|\le \|v\|.\] For any $u\in U$, we have
  \[\|v-P_U v\|\le \|v-u\|.\]
\end{prop}
\begin{prop}
    Let $A$ be injective. Then $A^* A$ is invertible, and the projection
    onto the image of $A$ is $A(A^* A)^{-1}A^*$.
\end{prop}
\begin{cor}[Least Squares Regression]
    For any vector $b$, the vector $x$ that minimises $\|b-Ax\|$ is $(A^*
    A)^{-1} A^* b$.
\end{cor}
\begin{defn}
  A linear transformation is an \emph{isometry} if it preserves norms. An
  operator which is also an isometry is \emph{unitary}.
\end{defn}
\begin{prop}
    An linear map $T$ is an isometry iff $T^*T=I$.
\end{prop}
\begin{thm}[QR Decomposition]
    Let $A:V\to W$ be linear, and pick orthonormal bases on $V$ and $W$. There
    There exists a unitary operator $Q$ and an upper triangular matrix $R$ such
    that $A=QR$.
\end{thm}
\begin{prop}
    If $A=QR$ as above and $A$ is injective, then $(A^* A)x=A^*b$ implies $Rx=Q^*
    b$.
\end{prop}
\begin{thm}[Polar Decomposition]
  For each operator $T$, there exists a unitary operator $S$ such that
  $T=S\sqrt{T^*T}$.
\end{thm}
\begin{defn}
  The \emph{singular values} of $T$ are the eigenvalues of $\sqrt{T^*T}$,
  where each eigenvalue $\lambda$ is counted the same number of times as the
  dimension of its eigenspace.
\end{defn}
\begin{prop}
  The nonzero singular values of $T$ and of $T^*$ coincide.
\end{prop}
\begin{thm}[Singular Value Decomposition]
  Suppose $T:V\to W$ has singular values $s_1,\ldots,s_n$. Then there exist
  orthonormal bases $e_1,\ldots,e_n$ of $V$ and $f_1,\ldots,f_n$ of $W$ such that
  \[Tv=\sum_i s_i\langle v,e_i\rangle f_i\]
  for all $v\in V$.
\end{thm}
\begin{prop}
  Let $T:V\to W$ be a linear transformation. There exists a unique linear
  transformation $T^+:W\to V$ such that
  \begin{itemize}
    \item $TT^+T=T$;
    \item $T^+TT^+=T^+$;
    \item $TT^+$ and $T^+T$ are self-adjoint.
  \end{itemize}
\end{prop}
\begin{defn}
  This transformation $T^+$ is known as the \emph{pseudoinverse} of $T$.
\end{defn}
\begin{prop}
    If $T$ is injective, then $T^+T=I$. If $T$ is surjective, then $TT^+=I$.
\end{prop}
\begin{prop}
    The perpendicular projection onto the image of $T$ is $TT^+$.
\end{prop}
\begin{defn}
  A vector $v$ is called a \emph{generalised eigenvector} of $T$ corresponding
  to an eigenvalue $\lambda$ if $v\ne 0$ and $(T-\lambda I)^j v=0$ for some
  positive integer $j$.

  The \emph{generalised eigenspace} of $T$ corresponding to $\lambda$ is the set
  of all generalised eigenvectors of $T$ corresponding to $\lambda$, along with
  the $0$ vector.
\end{defn}
\begin{prop}
  For finite-dimensional $V$, 
  $v$ is a generalised eigenvector of $T$ iff $(T-\lambda I)^{\dim V}v=0$.
\end{prop}
\begin{prop}
  Generalised eigenvectors corresponding to distinct eigenvalues are linearly
  independent.
\end{prop}
\begin{prop}
  Suppose $V$ is a finite-dimensional complex vector space, and $T$ is an
  operator on $V$. Then there is a basis of $V$ consisting of generalised
  eigenvectors of $T$.
\end{prop}
\begin{defn}
  The \emph{multiplicity} of an eigenvalue $\lambda$ of $T$ is the dimension of
  the corresponding generalised eigenspace.
\end{defn}
\begin{prop}
  If $T$ is diagonalisable, then the multiplicity of $\lambda$ equals the number
  of times that $\lambda$ appears in the diagonal matrix of $T$ with respect to
  any basis.
\end{prop}
\begin{prop}
  Every operator on a nonzero finite-dimensional real vector space has an invariant
  subspace of dimension $1$ or $2$.
\end{prop}
\begin{defn}
  A linear transformation $T$ is \emph{nilpotent} if $T^q=0$ for some positive
  integer $q$. The least positive integer $q$ such that this is true is called
  the \emph{index} of $T$.
\end{defn}
\begin{defn}
  If $X$ is a complement of $Y$, and $X$ and $Y$ are both invariant under $T$,
  then $T$ is said to be \emph{decomposed} by $X$ and $Y$.
\end{defn}
\begin{prop}
  For every linear transformation $A$ on a finite-dimensional vector space $V$,
  there are unique subspaces $X$ and $Y$ on $V$ such that $A$ is decomposed by
  $X$ and $Y$, $A|_X$ is nilpotent, and $A|_Y$ is invertible.
\end{prop}
\begin{prop}
  If $A$ is nilpotent with index $q$ on a finite-dimensional vector space $V$,
  then there exist positive integers $r,q=q_1\ge\cdots\ge q_r$ and vectors
  $x_1,\ldots,x_r$ such that $\{A^j x_i:1\le i\le r,j<q_r\}$ is a basis for $V$
  and $A^{q_i}x_i=0$ for all $i$.
\end{prop}
\begin{defn}
  A \emph{block diagonal matrix} is a square matrix of the form
  \[\begin{pmatrix} A_1&&0 \\ &\ddots& \\0&&A_m \end{pmatrix},\]
  where each $A_i$ is a square matrix lying along the diagonal and all other
  entries of the matrix are $0$.
\end{defn}
\begin{thm}[Jordan Form]
  If $T$ is an operator on a finite-dimensional complex vector space, then there is a
  basis such that the matrix of $T$ with respect to this basis is block diagonal
  with blocks of the form
  \[\begin{pmatrix} \lambda_i&1&&0 \\
  &\ddots&\ddots& \\
  &&\ddots&1 \\
  0&&&\lambda_i\end{pmatrix},\]
  where each $\lambda_i$ is a distinct eigenvalue of $T$.
\end{thm}
\begin{defn}
  The \emph{trace} of a square matrix $A$ is the sum of the diagonal entries of
  $A$.
\end{defn}
\begin{prop}
  If $T$ is an operator over a finite-dimensional vector space $V$, and
  $\{a_i\}$ and $\{b_i\}$ are two bases for $V$, then the trace of the matrix of
  $T$ with respect to $\{a_i\}$ equals the trace of the matrix of $T$ with
  respect to $\{b_i\}$.
\end{prop}
\begin{defn}
  We call this quantity the \emph{trace} of $T$.
\end{defn}
\begin{prop}
  The trace is additive; further, the traces of $AB+kI$ and of $BA+kI$ are equal.
\end{prop}
\begin{prop}
  If $V$ is complex, then the trace of $T$ equals the sum of the eigenvalues of
  $T$ counted according to multiplicity.
\end{prop}
\begin{defn}
  Let $\{V_i\}_1^k$ be vector spaces over $\Ff$. A \emph{$k$-linear form} is a
  function $f:\prod_1^k V_i\to\Ff$ such that, if all except for one argument is
  kept fixed, then the function is linear in the remaining argument. If all
  $V_i$ are equal to $V$, then $f$ is a \emph{$k$-linear form on $V$}.
\end{defn}
\begin{defn}
  A $k$-linear form $f$ on $V$ is \emph{alternating} if $f(x_1,\ldots,x_k)=0$
  whenever two of the $x_i$s are equal.
\end{defn}
\begin{prop}
  If $x_1,\ldots,x_k$ are linearly dependent vectors and $w$ is an alternating
  $k$-linear form, then $w(x_1,\ldots,x_k)=0$.
\end{prop}
\begin{prop}
  If $V$ is an $n$-dimensional vector space for $n>0$, then the vector space of
  alternating $n$-linear forms on $V$ is one-dimensional.
\end{prop}
\begin{prop}
  Let $A$ be an operator on $V$. To each nonzero alternating $n$-linear form $w$
  on $V$ we associate the form $\bar Aw$ defined by $(\bar Aw)(x_1,\ldots,x_n)
  =w(Ax_1,\ldots,Ax_n)$. Then there exists a scalar $|A|\in\Ff$ such that
  $\bar Aw=|A|w$.
\end{prop}
\begin{defn}
  We call this scalar $|A|$ the \emph{determinant} of $A$, also denoted $\det
  A$.
\end{defn}
\begin{prop}
  The determinant is multiplicative.
\end{prop}
\begin{cor}
  $A$ is invertible iff $\det A\ne 0$.
\end{cor}
\begin{prop}
    $\det A=\det A'$.
\end{prop}
\begin{prop}
    Let $a,b\in\Rr^3$. There exists a unique vector $a\times b$ such that for
    for any vector $c$, $(a\times b)\cdot c$ is the determinant of the operator
    which sends the standard basis to $a,b,c$.
\end{prop}
\begin{defn}
    This vector is the \emph{cross product} of $a$ and $b$.
\end{defn}
\begin{prop}
    We have the following identities:
    \begin{itemize}
        \item $a\times b+b\times a=0$
        \item $a\times(b\times c)+b\times(c\times a)+c\times(a\times b)=0$.
    \end{itemize}
\end{prop}
\begin{defn}
  If $\pi$ is a permutation of $\{1,2,\ldots,n\}$, the sign of $\pi$ is $-1^k$,
  where $k=|\{(a,b)\in\{1,2,\ldots,n\}:a<b,\ \pi(a)>\pi(b)\}|$.
\end{defn}
\begin{prop}
  Let $A$ be $n\times n$. The determinant of the linear transformation defined
  by $A$ equals
  \[\sum_\pi\sign(\pi)\prod_{i=1}^n A_{\pi(i),i},\] 
  where the sum is taken over all permutations $\pi$ of $\{1,2,\ldots,n\}$.
\end{prop}
\begin{defn}
  For an $n\times n$ matrix $A$, let $A_{j,k}$ denote the $(n-1)\times(n-1)$
  matrix obtained from $A$ by crossing out row number $j$ and column number $k$.
  The numbers $C_{j,k}=(-1)^{j+k}\det A_{j,k}$ are the \emph{cofactors} of $A$.
  Let $C$ be the matrix whose entries are the cofactors of a given matrix $A$.
\end{defn}
\begin{thm}[Cofactor expansion]
  $AC^T=(\det A)I$.
\end{thm}
\begin{cor}[Cramer's rule]
  For an invertible matrix $A$, the entry number $k$ of the solution of the
  equation $Ax=b$ is given by
  \[x_k=\frac{\det B_k}{\det A},\]
  where the matrix $B_k$ is obtained from $A$ by replacing column number $k$ of
  $A$ by $B$.
\end{cor}
\begin{prop}
  If $V$ is a complex vector space, then the determinant of $T$ equals the product of the
  eigenvalues of $T$ counted according to multiplicity.
\end{prop}
\begin{defn}
  Let $T$ be an operator on a finite-dimensional vector space. The
  \emph{characteristic polynomial} $p$ of $T$ is defined by
  \[p(\lambda)=\det(T-\lambda I).\]
\end{defn}
\begin{thm}[Cayley-Hamilton]
  Let $p$ be the characteristic polynomial of $T$. Then $p(T)=0$.
\end{thm}
\begin{prop}
  If $T$ is an operator on a finite-dimensional complex vector space, then the
  characteristic polynomial $p$ of $T$ satisfies
  \[p(z)=\prod(\lambda_i-z),\]
  where $\lambda_i$ are the eigenvalues of $T$ counted according to
  multiplicity.
\end{prop}
\begin{cor}
  All eigenvalues of $A$ are positive iff $A$ is positive definite.
\end{cor}
\begin{prop}[Sylvester's Criterion]
  Let $A$ be self-adjoint. Then $A$ is positive definite iff 
    for each $k$, the determinant of the top-left $k\times k$ submatrix of
      $A$ is positive.
\end{prop}
\subsection*{References}
\begin{itemize}
    \item Strang, \emph{Linear Algebra and its Applications}
    \item Treil, \emph{Linear Algebra Done Wrong}
    \item Axler, \emph{Linear Algebra Done Right}
    \item Halmos, \emph{Finite-Dimensional Vector Spaces}
\end{itemize}

\chapter{Metric Spaces}
\begin{defn}
  A \emph{metric space} is a nonempty set $M$ together with a function
  $d:M\times M\to\Rr$ (the \emph{metric}) such that
  \begin{itemize}
    \item $d(x,y)=0\iff x=y$,
    \item $d(x,y)=d(y,x)$ (symmetry), 
    \item $d(x,z)\le d(x,y)+d(y,z)$ (triangle inequality).
  \end{itemize}
\end{defn}
\begin{prop}
  In a normed vector space, the function $d(x,y)=\|x-y\|$ is a metric.
\end{prop}
\begin{defn}
  We call this the \emph{induced metric}.
\end{defn}
\begin{defn}
  In a metric space, the \emph{open ball} $B_r(x)$ with centre $x$ and radius $r$ is the
  set of all points $y$ with $d(x,y)<r$.

  The \emph{closed ball} $\overline{B_r(x)}$ with centre $x$ and radius $r$ is
  the set of all points $y$ with $d(x,y)\le r$.
\end{defn}
\begin{defn}
  Let $E$ be a subset of a metric space $M$.
  \begin{itemize}
    \item A point $p$ is a \emph{limit point} of $E$ if every open ball centred
      at $p$ contains a point $q\ne p$ such that $q\in E$.
    \item A point $p$ is an \emph{interior point} of $E$ if there is an open
      ball centred at $p$ which is a subset of $E$.
    \item $E$ is \emph{closed} if every limit point of $E$ is a point of $E$.
    \item $E$ is \emph{open} if every point of $E$ is an interior point of $E$.
    \item $E$ is \emph{bounded} if it is contained in some open ball.
    \item The \emph{complement} $E^c$ of a set $E$ is the set $M\setminus E$.
    \item The \emph{interior} of $E$ is the set of interior points of $E$.
    \item The \emph{boundary} $\partial E$ of $E$ is the set of points of $M$
      that are limit points of both $E$ and $E^c$.
  \end{itemize}
\end{defn}
\begin{prop}
  The interior and boundary of $E$ are disjoint, and their union is $E$.
\end{prop}
\begin{prop}
  The following are equivalent:
  \begin{itemize}
    \item $E$ is open.
    \item $E\cap\partial E=\emptyset$.
    \item $E^c$ is closed.
    \item $\partial E\subseteq E^c$.
  \end{itemize}
\end{prop}
\begin{prop}
  Every open ball is open; every closed ball is closed.
\end{prop}
\begin{prop}
  If $p$ is a limit point of $E$, then every open ball centred around $p$
  contains infinitely many points of $E$.
\end{prop}
\begin{prop}
  Any union of open sets is open; a finite intersection of open sets is open.

  Any intersection of closed sets is closed; a finite union of closed sets is
  closed.
\end{prop}
\begin{defn}
  The \emph{closure} of $E$ is the set $E\cup\partial E$.
\end{defn}
\begin{prop}
  The closure of $E$ is closed; the interior of $E$ is open.

  Any closed set which contains $E$ contains the closure of $E$. Any open set
  which is contained in $E$ is contained in the interior of $E$.
\end{prop}
\begin{prop}
  Suppose $X\subseteq M$ inherits the metric. A subset $E$ of $X$ is open
  relative to $X$ iff $E=X\cap Y$ for some open set $Y$.
\end{prop}
\begin{defn}
  An \emph{open cover} of $E$ is a set of open sets whose union contains $E$.
\end{defn}
\begin{prop}
  The following are equivalent:
  \begin{itemize}
    \item 
      Every open cover of $E$ contains a finite
      subset which is still an open cover of $E$.
    \item Every infinite subset of $E$ contains a limit point in $E$.
  \end{itemize}
\end{prop}
\begin{defn}
  Such a set is called \emph{compact}.
\end{defn}
\begin{prop}
  Suppose $X\subseteq M$ inherits the metric. A subset $E$ of $X$ is open
  relative to $X$ iff $E$ is compact relative to $M$.
\end{prop}
\begin{prop}
  A compact subset of a metric space is closed and bounded; a closed subset of a compact
  metric space is compact.
\end{prop}
\begin{prop}
  If $S$ is a collection of compact subsets of a metric space such that any
  finite intersection of elements of $S$ is nonempty, then $\bigcap S$ is
  nonempty.
\end{prop}
\begin{thm}[Heine-Borel]
  A subset of $\Rr^n$ is compact iff it is closed and bounded.
\end{thm}
\begin{thm}[Weierstrass]
  Every bounded infinite subset of $\Rr^n$ has a limit point.
\end{thm}
\begin{defn}
  Two subsets $A$ and $B$ of a metric space $X$ are \emph{separated} if both
  $A\cap\overline B$ and $B\cap\overline A$.

  A set $E$ is \emph{disconnected} if it is the union of two nonempty separated
  sets, and connected otherwise.
\end{defn}
\begin{prop}
  A metric space $M$ is connected iff the only sets which are both open and closed
  are the empty set and $M$.
\end{prop}
\begin{prop}
  A subset of $\Rr^1$ is connected iff it is an interval.
\end{prop}
\begin{defn}
  A sequence $\{a_n\}$ is \emph{convergent} if there is a point $L$ such that
  for any $\varepsilon>0$ there is an $N\in\Zz^+$ such that $n\ge N$ implies
  $d(a_n,L)<\varepsilon$. We write \[\lim_{n\to\infty} a_n=L.\]
\end{defn}
\begin{prop}
  Suppose $\{a_n\}$ and $\{b_n\}$ are sequences of complex numbers which
  converge to $a$ and $b$ respectively. Then the sequences $\{a_n+b_n\},\
  \ \{a_n b_n\},\ \{\frac{a_n}{b_n}\}$ converge to $a+b,\ ab,\ \frac ab$
  respectively (where in the last one we require $b_n\ne 0$ for each $n$).
\end{prop}
\begin{prop}
  A sequence in $\Rr^n$ or $\Cc^n$ converges iff it converges coordinatewise.
\end{prop}
\begin{defn}
  A sequence $\{p_n\}$ is \emph{Cauchy} if for every $\varepsilon>0$ there is an
  integer $N$ such that $d(p_n,p_m)<\varepsilon$ if $m,n\ge N$.

  A metric space is \emph{complete} if every Cauchy sequence converges.
\end{defn}
\begin{prop}
  Every convergent sequence is Cauchy.
\end{prop}
\begin{prop}
  Every compact metric space is complete.
\end{prop}
\begin{prop}
  $\Rr^n$ and $\Cc^n$ are complete.
\end{prop}
\begin{defn}
  Let $f:X\to Y$ be a function, where $Y$ is a metric space and $X$ is a 
  subset of a metric space $E$. Let $p$ be a limit point of $X$. We say that
  \[\lim_{x\to p}f(x)=q\] if for every sequence $\{x_n\}$ in $E$ which converges
  to $p$ but does not contain $p$, $f(x_n)$ converges to $q$.
\end{defn}
\begin{defn}
  We say that $f$ is \emph{continuous} at $p$
  if for every sequence $\{x_n\}$ in $E$ which converges
  to $p$, $f(x_n)$ converges to $f(p)$.

  We say that $f$ is \emph{continuous} on $X$, or simply \emph{continuous},
  if it is continuous at every point in $X$.
\end{defn}
\begin{prop}
  A function $f$ is continuous iff the inverse image of every open set is open.
\end{prop}
\begin{prop}
  If $f$ is continuous, then
  \begin{itemize}
    \item The image of a compact set is compact.
    \item The image of a connected set is connected.
  \end{itemize}
\end{prop}
\begin{cor}[Intermediate Value Theorem]
  If the codomain of $f$ is $\Rr$, then it is an interval. If the domain of $f$
  is a compact set, then the interval is closed.
\end{cor}
\begin{defn}
  A function $f$ is \emph{uniformly continuous} if for every $\varepsilon>0$
  there exists a $\delta>0$ such that if $d(a,b)<\delta$ then
  $d(f(a),f(b))<\varepsilon$.
\end{defn}
\begin{thm}
  Every continuous function on a compact set is uniformly continuous.
\end{thm}
\begin{thm}
  Let $S$ be an open subset of $\Rr^n$, and let $f:S\to\Rr^n$ be an injective
  continuous function. Then the image of $f$ is open.
\end{thm}
\begin{thm}[Fundamental Theorem of Algebra]
  Let $p:\Cc\to\Cc$ be a polynomial. Then the image of $p$ is $\Cc$.
\end{thm}
\subsection*{References}
\begin{itemize}
  \item \emph{Principles of Mathematical Analysis}, Rudin
\end{itemize}

\appendix
\chapter{Proofs}

\end{document}

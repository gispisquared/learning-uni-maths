\documentclass{amsbook}
\usepackage{amsmath,amssymb,amsthm,tcolorbox,hyperref,booktabs}
\hypersetup{colorlinks=true}
\theoremstyle{plain}
  \newtheorem{thm}{Theorem}
  \newtheorem{lem}[thm]{Lemma}
  \newtheorem{prop}[thm]{Proposition}
  \newtheorem{cor}[thm]{Corollary}
  \newtheorem{axiom}[thm]{Axiom}
\theoremstyle{definition}
  \newtheorem{defn}[thm]{Definition}
  \newtheorem{eg}[thm]{Example}
\theoremstyle{remark}
  \newtheorem{rem}[thm]{Remark}
  \newtheorem{note}[thm]{Note}
  \newtheorem{case}[thm]{Case}
  \newtheorem{claim}[thm]{Claim}
\newcommand{\LHS}{\text{LHS}}
\newcommand{\RHS}{\text{RHS}}
\newcommand{\Zz}{\mathbb{Z}}
\newcommand{\Qq}{\mathbb{Q}}
\newcommand{\Rr}{\mathbb{R}}
\newcommand{\Cc}{\mathbb{C}}
\newcommand{\Ff}{\mathbb{F}}
\newcommand{\lcm}{\mathrm{lcm}}
\title{Learning Uni Maths}
\author{gispisquared}
\begin{document}
\maketitle
\tableofcontents
\chapter{Set Theory}
\begin{axiom}[Existence]
  There exists a set.
\end{axiom}
\begin{rem}
  This is implied by the Axiom of Infinity; however, we include it here so that
  we may define the empty set.
\end{rem}
\begin{defn}
  A \emph{sentence} is made by combining assertions of belonging (eg $x\in A$)
  and/or assertions of equality (eg $A=B$) using the usual logical operators:
  \emph{and, or, not, implies, if and only if, there exists, for all}.
\end{defn}
\begin{defn}
  Let $A$ and $B$ be sets.
  If every element of $A$ is an element of $B$, we say that $A$ is a
  \emph{subset} of $B$, denoted $A\subseteq B$.
\end{defn}
\begin{prop}
  If $A\subseteq B$ and $B\subseteq C$ then $A\subseteq C$.
\end{prop}
\begin{axiom}[Extensionality]
  $A=B$ iff $A\subseteq B$ and $B\subseteq A$.
\end{axiom}
\begin{axiom}[Specification]
  For every set $A$ and every sentence $S(x)$ there is a set $B$ whose elements
  are exactly those elements $x$ of $A$ for which $S(x)$ holds.
\end{axiom}
\begin{defn}
  We notate this set $B$ by $\{x\in A: S(x)\}$.
\end{defn}
\begin{prop}
  There exists a unique set $X$ such that for any $x$, the sentence
  $x\in X$ is false.
\end{prop}
\begin{defn}
  We call this set the \emph{empty set}, notated $\emptyset$.
\end{defn}
\begin{prop}
  For every set $A$ there is a set $B$ such that $B\not\in A$.
\end{prop}
\begin{axiom}[Pairing]
  For any two sets $A$ and $B$ there is a set $X$ with $A\in X$ and $B\in X$.
\end{axiom}
\begin{prop}
  There is a unique set $Y$ such that for any $a$, $a$ is in $Y$ iff $a=A$ or $a=B$.
\end{prop}
\begin{defn}
  This set is called the \emph{unordered pair} formed by $A$ and $B$, denoted
  $\{A,B\}$.
\end{defn}
\begin{defn}
  The set $\{A,A\}$ is denoted $\{A\}$, and called the \emph{singleton} of
  $A$.
\end{defn}
\begin{axiom}[Union]
  For any set $X$ of sets there exists a set $Y$ such that for any $A$ in $X$,
  and any $a$ in $A$, $a$ is in $Y$.
\end{axiom}
\begin{prop}
  For a nonempty set $X$ of sets there is a unique set $Z$ such that $a$ is in $Z$ if and
  only if there exists an $A$ in $X$ such that $a$ is in $A$.
\end{prop}
\begin{defn}
  This set is called the \emph{union} of $X$, denoted
  $\bigcup X$.

  For two sets $A$ and $B$ we define $A\cup B=\bigcup \{A,B\}$.
\end{defn}
\begin{defn}
  Let $A$ and $B$ be sets.
  The \emph{intersection} of $A$ and $B$, notated $A\cap B$, is $\{x\in A:x\in
    B\}$.

  If $A\cap B=\emptyset$ then $A$ and $B$ are called \emph{disjoint}.
\end{defn}
\begin{prop}
  We have
  \begin{itemize}
    \item $A\cup\emptyset=A$,
    \item $A\cup B=B\cup A$ (commutative),
    \item $A\cup (B\cup C)=(A\cup B)\cup C$ (associative),
    \item $A\cup A=A$ (idempotent),
    \item $A\cup (B\cap C)=(A\cup B)\cap(A\cup C)$ (distributive),
    \item $A\subseteq B$ iff $A\cup B=B$,
    \item $A\cap\emptyset=A$,
    \item $A\cap B=B\cap A$ (commutative),
    \item $A\cap (B\cap C)=(A\cap B)\cap C$ (associative),
    \item $A\cap A=A$ (idempotent),
    \item $A\cap (B\cup C)=(A\cap B)\cup(A\cap C)$ (distributive),
    \item $A\subseteq B$ iff $A\cap B=A$.
  \end{itemize}
\end{prop}
\begin{prop}
  For every nonempty set $C$ of sets, there is a unique set $Y$ such that $x\in Y$ iff
  $x\in X$ for each $X$ in $C$.
\end{prop}
\begin{defn}
  This set $Y$ is called the \emph{intersection} of $C$, denoted $\bigcap C$.
\end{defn}
\begin{axiom}[Powers]
  For each set $X$ there is a set that contains all subsets of $X$.
\end{axiom}
\begin{prop}
  There is a unique set $Y$ such that $x\in Y$ iff $x\subseteq X$.
\end{prop}
\begin{defn}
  This set $Y$ is called the \emph{power set} of $X$, denoted $\mathcal P(X)$.
\end{defn}
\begin{defn}
  The \emph{ordered pair} of $a$ and $b$ is the set defined as
  \[(a,b)=\{\{a\},\{a,b\}\}.\]
\end{defn}
\begin{prop}
  For any $a,b,c,d$, we have $(a,b)=(c,d)$ iff $a=c$ and $b=d$.
\end{prop}
\begin{defn}
  Let $A$ and $B$ be sets. The \emph{Cartesian product} $A\times B$ is
  \[\{(x,y): x\in A,\ y\in B\}.\]
\end{defn}
\begin{prop}\label{prop:1:orderedpairsinprod}
  For any set $R$ of ordered pairs there are sets $A$ and $B$ such that
  $R\subseteq A\times B$.
\end{prop}
\begin{defn}
  A \emph{binary relation} $R$ over sets $A$ and $B$ is a subset of $A\times
  B$. If $(a,b)$ is in $R$ we write $aRb$.

  If $A=B$ then we call it a \emph{binary relation over} $A$.
\end{defn}
\begin{defn}
  An \emph{equivalence relation} is a binary relation $\sim$ over $A$ such
  that
  \begin{itemize}
    \item $a\sim a$ (reflexive),
    \item $a\sim b\iff b\sim a$ (symmetric), and
    \item if $a\sim b$ and $b\sim c$ then $a\sim c$ (transitive).
  \end{itemize}
  The \emph{equivalence class} of $a$ under $\sim$ is
  \[[a]=\{x\in A:x\sim a\}.\]
\end{defn}
\begin{defn}
  A \emph{partition} of a set $A$ is a disjoint set of subsets of $A$ whose
  union is $A$.

  A partition $X$ of $A$ \emph{induces} a relation $\sim$, where $a\sim b$ iff
  $a$ and $b$ belong to the same element of $X$.
\end{defn}
\begin{prop}
  The set of equivalence classes of an equivalence relation exists and 
  is a partition.
\end{prop}
\begin{defn}
  This partition is called the partition \emph{induced} by the equivalence
  relation $\sim$.
\end{defn}
\begin{prop}
  The equivalence relation induced by a partition induces that partition; the
  partition induced by an equivalence relation induces that relation.
\end{prop}
\begin{defn}
  For any set $X$ we define $X^+=X\cup\{X\}$.
\end{defn}
\begin{axiom}[Infinity]
  There exists a set $S$ containing $\emptyset$ and containing $X^+$ for
  every $X$ in $S$.
\end{axiom}
\begin{prop}
  There exists a unique set $\omega$ which is a subset of all such sets $S$.
\end{prop}
\begin{prop}
  For any $a,b\in\omega$, exactly one of $a\in b,\ a=b,\ b\in a$ is true.
\end{prop}
\begin{prop}
  For any $a\in\omega$ and any $b\in a$, $b\subseteq a$.
\end{prop}
\begin{defn}
  A \emph{function} $f:A\to B$ is a relation $f$ over $A$ and $B$ such that
  for each $a\in A$ there is exactly one $b\in B$ such that $afb$. We usually
  write this as $f(a)=b$.

  A function $f$ is \emph{injective} if for each $b$ in $B$, there is at most one
  $a$ in $A$ such that $f(a)=b$. It is \emph{surjective} if for each $b$ in
  $B$ there is at least one $a$ in $A$ such that $f(a)=b$. A function which is
  both injective and surjective is \emph{bijective}.
\end{defn}
\begin{thm}[Recursion theorem]
  If $a$ is an element of a set $X$, and if $f:X\to X$ is a function, then there
  is a function $g:\omega\to X$ such that $u(0)=a$ and $u(n^+)=f(u(n))$ for all
  $n$ in $\omega$.
\end{thm}
\begin{axiom}[Substitution]
  If $S(a,b)$ is a sentence such that for each $a$ in a set $A$ there exists a
  set $B_a$ such that $b\in B_a\iff S(a,b)$, then there exists a function $F$ with
  domain $A$ such that $F(a)=B_a$ for each $a$ in $A$.
\end{axiom}
\begin{axiom}[Foundation]
  Every set $X$ contains a set $Y$ such that $X$ and $Y$ are disjoint.
\end{axiom}
\begin{axiom}[Choice]
  Let $X$ be a set of sets whose members are all nonempty. Then there exists a
  function $f:X\to\bigcup X$ such that $f(Y)\in Y$ for all $Y\in X$.
\end{axiom}
\begin{defn}
  A \emph{partial order} is a binary relation $\le$ on a a set $A$ such that
  \begin{itemize}
    \item $a\le a$ (reflexive),
    \item if $a\le b$ and $b\le a$ then $a=b$ (antisymmetric), and
    \item if $a\le b$ and $b\le c$ then $a\le c$ (transitive).
  \end{itemize}
  We define $a<b$ if $a\le b$ and $a\ne b$.

  If for all $a$ and $b$ we have $a\le b$ or $b\le a$ (strongly connected),
  then $\le$ is a \emph{total order}. 

  A \emph{chain} is a totally ordered subset of a partially ordered set.
\end{defn}
\begin{defn}
  If $X$ is a partially ordered set, and if $a\in X$, the set $s(a)=\{x\in
    X:x<a\}$ is called the \emph{initial segment} determined by $a$.
\end{defn}
\begin{defn}
  Two partially ordered sets $X$ and $Y$ are \emph{similar} if there is a
  bijection $f:X\to Y$ such that $a\le b\iff f(a)\le f(b)$. This bijection is
  called a \emph{similarity}.
\end{defn}
\begin{defn}
  Let $S$ be a subset of a partially ordered set $A$, and let $a$ be an element
  of $A$. If $s\le a$ for every $s$ in $S$, then we call $a$ an \emph{upper
    bound} of $S$. If $a\le s$ for every $s$ in $S$, then we call $a$ a
    \emph{lower bound} of $S$. If $a$ is an upper bound of $S$ and a lower
    bound of the set of upper bounds of $S$, then we call $a$ a \emph{least
      upper bound} of $S$.
\end{defn}
\begin{defn}
  A \emph{well-order} on $A$ is a total order $\le$ on $A$ such that every
  nonempty subset $S$ of $A$ has an element $a$ which is a lower bound for $S$.
  The set $A$ together with the relation $\le$ is then called \emph{well-ordered}.
\end{defn}
\begin{prop}
  If two well-ordered sets are similar, then the similarity is unique.
\end{prop}
\begin{thm}
  If $X$ and $Y$ are well-ordered, then either $X$ and $Y$ are similar, or one
  is similar to an initial segment of the other.
\end{thm}
\begin{defn}
  An \emph{ordinal number} is a well-ordered set $\alpha$ such that for any
  $\xi\in\alpha$ we have $s(\xi)=\xi$.
\end{defn}
\begin{prop}
  $\omega$ is an ordinal number.
\end{prop}
\begin{prop}
  If $\alpha$ is an ordinal number then so is $\alpha^+$, and so is any element
  of $\alpha$.
\end{prop}
\begin{thm}
  If two ordinal numbers are similar, then they are equal.

  Otherwise, one is an element of the other.
\end{thm}
\begin{prop}
  If a set $\alpha$ can be well-ordered such that it is an ordinal, then the
  ordering is unique.
\end{prop}
\begin{prop}
  Every well-ordered set is similar to a unique ordinal number.
\end{prop}
\begin{prop}
  There is no set of all ordinal numbers.
\end{prop}
\begin{thm}[Zorn's Lemma]
  Suppose a partially ordered set $P$ has the property that every chain in $P$
  has an upper bound in $P$. Then there is an element $a\in P$ such that the
  only upper bound for $\{a\}$ is $a$.
\end{thm}
\begin{thm}[Well-Ordering Theorem]
  Every set has a well-ordering.
\end{thm}
\begin{defn}
  Two sets $A$ and $B$ are said to have the same \emph{cardinality} (written
  $|A|=|B|$) if there is a bijection $f:A\to B$.

  A set $A$ has cardinality at most the cardinality of $B$ ($|A|\le|B|$) if
  there is an injection $f:A\to B$.

  A set $A$ has cardinality less than the cardinality of $B$ ($|A|<|B|$) if
  $|A|\le|B|$ and $|A|\ne|B|$.
\end{defn}
\begin{thm}
  If $|A|\le|B|$ and $|B|\le|A|$ then $|A|=|B|$.
\end{thm}
\begin{thm}
  For any set $A$, $|\mathcal P(A)|>|A|$.
\end{thm}
\begin{defn}
  A \emph{cardinal number} is an ordinal number $\alpha$ such that for any
  ordinal number $\beta$ with $|\alpha|=|\beta|$ we have $\alpha\subseteq\beta$.
\end{defn}
\begin{prop}
  For any set $S$, there is a unique cardinal number $\alpha$ with
  $|\alpha|=|S|$.
\end{prop}
\begin{defn}
  For these sets $S$ and $\alpha$ we define $|S|=\alpha$.
\end{defn}
\begin{defn}
  A set $A$ is said to be \emph{finite} if $|A|\in\omega$, and \emph{infinite}
  otherwise.
\end{defn}
\begin{prop}
  A set is infinite if and only if it has the same cardinality as some proper
  subset.
\end{prop}
\begin{defn}
  An infinite set $A$ is said to be \emph{countable} if $|A|=\omega$, and
  \emph{uncountable} otherwise.
\end{defn}
\begin{prop}
  A countable set does not have any uncountable subsets. An uncountable set has
  a countable subset.
\end{prop}
\subsection*{References}
\begin{itemize}
  \item \emph{Naive Set Theory}, Halmos
  \item \emph{Set Theory}, Jech
\end{itemize}

\chapter{Ordinals}
\begin{defn}
  A \emph{partial order} is a binary relation $\le$ on a a set $A$ such that
  \begin{itemize}
    \item $a\le a$ (reflexive),
    \item if $a\le b$ and $b\le a$ then $a=b$ (antisymmetric), and
    \item if $a\le b$ and $b\le c$ then $a\le c$ (transitive).
  \end{itemize}
  We define $a<b$ if $a\le b$ and $a\ne b$.

  If for all $a$ and $b$ we have $a\le b$ or $b\le a$ (strongly connected),
  then $\le$ is a \emph{total order}. 

  A \emph{chain} is a totally ordered subset of a partially ordered set.
\end{defn}
\begin{defn}
  If $X$ is a partially ordered set, and if $a\in X$, the set $s(a)=\{x\in
    X:x<a\}$ is called the \emph{initial segment} determined by $a$.
\end{defn}
\begin{defn}
  Two partially ordered sets $X$ and $Y$ are \emph{similar} if there is a
  bijection $f:X\to Y$ such that $a\le b\iff f(a)\le f(b)$. This bijection is
  called a \emph{similarity}.
\end{defn}
\begin{defn}
  Let $S$ be a subset of a partially ordered set $A$, and let $a$ be an element
  of $A$. If $s\le a$ for every $s$ in $S$, then we call $a$ an \emph{upper
    bound} of $S$. If $a\le s$ for every $s$ in $S$, then we call $a$ a
    \emph{lower bound} of $S$. If $a$ is an upper bound of $S$ and a lower
    bound of the set of upper bounds of $S$, then we call $a$ a \emph{least
      upper bound} of $S$.
\end{defn}
\begin{defn}
  A \emph{well-order} on $A$ is a total order $\le$ on $A$ such that every
  nonempty subset $S$ of $A$ has an element $a$ which is a lower bound for $S$.
  The set $A$ together with the relation $\le$ is then called \emph{well-ordered}.
\end{defn}
\begin{prop}
  If two well-ordered sets are similar, then the similarity is unique.
\end{prop}
\begin{thm}
  If $X$ and $Y$ are well-ordered, then either $X$ and $Y$ are similar, or one
  is similar to an initial segment of the other.
\end{thm}
\begin{defn}
  An \emph{ordinal number} is a well-ordered set $\alpha$ such that for any
  $\xi\in\alpha$ we have $s(\xi)=\xi$.
\end{defn}
\begin{prop}
  $\omega$ is an ordinal number.
\end{prop}
\begin{prop}
  If $\alpha$ is an ordinal number then so is $\alpha^+$, and so is any element
  of $\alpha$.
\end{prop}
\begin{thm}
  If two ordinal numbers are similar, then they are equal.

  Otherwise, one is an element of the other.
\end{thm}
\begin{prop}
  If a set $\alpha$ can be well-ordered such that it is an ordinal, then the
  ordering is unique.
\end{prop}
\begin{prop}
  Every well-ordered set is similar to a unique ordinal number.
\end{prop}
\begin{prop}
  There is no set of all ordinal numbers.
\end{prop}
\begin{thm}[Zorn's Lemma]
  Suppose a partially ordered set $P$ has the property that every chain in $P$
  has an upper bound in $P$. Then there is an element $a\in P$ such that the
  only upper bound for $\{a\}$ is $a$.
\end{thm}
\begin{thm}[Well-Ordering Theorem]
  Every set has a well-ordering.
\end{thm}

\chapter{Cardinality}
\begin{defn}
  Two sets $A$ and $B$ are said to have the same \emph{cardinality} (written
  $|A|=|B|$) if there is a bijection $f:A\to B$.

  A set $A$ has cardinality at most the cardinality of $B$ ($|A|\le|B|$) if
  there is an injection $f:A\to B$.

  A set $A$ has cardinality less than the cardinality of $B$ ($|A|<|B|$) if
  $|A|\le|B|$ and $|A|\ne|B|$.
\end{defn}
\begin{thm}
  If $|A|\le|B|$ and $|B|\le|A|$ then $|A|=|B|$.
\end{thm}
\begin{thm}
  For any set $A$, $|\mathcal P(A)|>|A|$.
\end{thm}
\begin{defn}
  A \emph{cardinal number} is an ordinal number $\alpha$ such that for any
  ordinal number $\beta$ with $|\alpha|=|\beta|$ we have $\alpha\in\beta$.
\end{defn}
\begin{prop}
  For any set $S$, there is a unique cardinal number $\alpha$ with
  $|\alpha|=|S|$.
\end{prop}
\begin{defn}
  For these sets $S$ and $\alpha$ we define $|S|=\alpha$.
\end{defn}
\begin{defn}
  A set $A$ is said to be \emph{finite} if $|A|\in\omega$, and \emph{infinite}
  otherwise.
\end{defn}
\begin{prop}
  A set is infinite if and only if it has the same cardinality as some proper
  subset.
\end{prop}
\begin{defn}
  An infinite set $A$ is said to be \emph{countable} if $|A|=\omega$, and
  \emph{uncountable} otherwise.
\end{defn}
\begin{prop}
  A countable set does not have any uncountable subsets. An uncountable set has
  a countable subset.
\end{prop}

\chapter{Number Systems}
  \begin{defn}
    A \emph{binary operation} on $A$ is a function $\cdot:A\times A\to A$. We
    usually write $\cdot(a,b)=c$ as $a\cdot b=c$.

    It is \emph{associative} if $(a\cdot b)\cdot c=a\cdot(b\cdot
    c)$ for any $a,\ b,\ c$ in $A$.

    It is \emph{commutative} if $a\cdot b=b\cdot a$ for any $a,\ b$ in $A$.
  \end{defn}
  \begin{defn}
    A \emph{monoid} is an ordered pair $(A,\cdot)$ of a set $A$ and an
    associative binary
    operation $\cdot$ on $A$ such that there exists an element $1$, called the
    \emph{identity}, such that $a\cdot 1=1\cdot a=a$ for all $a$.
  \end{defn}
  \begin{rem}
    There are two main notations for monoid-type structures. These are
    \begin{itemize}
      \item 
        Multiplicative notation, in which the operation is notated $a\cdot b$ or
        simply $ab$, and the identity element is $1$; and
      \item Additive notation, in which the operation is notated $a+b$ and the
        identity element is $0$.
    \end{itemize}
  \end{rem}
  \begin{defn}
    A \emph{group} is a monoid $(A,\cdot)$ such that for each element $a$ of $A$
    there is an element $b$ of $A$ such that
    $ab=1=ba$.
    
    A group is \emph{abelian} if the operation is commutative.
  \end{defn}
  \begin{prop}
    If $ab=ba=1$ and $ac=1$ or $ca=1$ then $b=c$.
  \end{prop}
  \begin{defn}
    The element $b$ of $A$ such that $ab=ba=1$ is called the \emph{inverse} of $a$.
    In multiplicative notation, the inverse of $a$ is notated $a^{-1}$.
    In additive notation, the inverse of $a$ is notated $-a$.
  \end{defn}
  \begin{rem}
    We often define $\frac ab=ab^{-1}$ in multiplicative notation, and
    $a-b=a+(-b)$ in additive notation.
  \end{rem}
  \begin{defn}
    A \emph{ring} is an ordered triple $(A,+,\cdot)$ such that $(A,+)$ is an
    abelian group, $(A\setminus \{0\},\cdot)$ is a monoid, and the
    \emph{distributive laws} hold:
    \[a\cdot(b+c)=ab+ac\quad\text{and}\quad (a+b)\cdot c=ac+bc.\]
    It is \emph{commutative} if $\cdot$ is commutative.

    It is \emph{ordered} if there is a total order $\le$ on $A$ satisfying
    \begin{itemize}
      \item if $a\le b$ then $a+c\le b+c$, and
      \item if $0\le a$ and $0\le b$ then $0\le ab$.
    \end{itemize}
  \end{defn}
  \begin{defn}
    A \emph{field} is a commutative ring $(A,+,\cdot)$ such that $(A\setminus
    \{0\},\cdot)$ is a group.

    An \emph{ordered field} is a field that is also an ordered ring.
  \end{defn}
  \begin{defn}
    In an ordered ring $R$, the \emph{absolute value} $|a|$ of an element $a$
    of $R$ is $a$ if $0\le a$, otherwise $-a$.
  \end{defn}
  \begin{prop}
    $|a+b|\le|a|+|b|$.
  \end{prop}
  \begin{defn}
    Let $X$ and $Y$ be similar well-ordered sets, and let $A$ and $B$ be the
    least elements of $X$ and $Y$ respectively. Assume that all other elements
    of $X$ and $Y$ are operations on $A$ and $B$ respectively, and let $f$ be
    the similarity between $A$ and $B$.

    A function $\varphi:A\to B$ is said
    to be a \emph{homomorphism} if
    for every $a,b\in A$ and every $\cdot\in X\setminus \{A\}$ we
    have \[\varphi(a\cdot b)=\varphi(a) f(\cdot) \varphi(b).\]

    An \emph{isomorphism} is a bijective homorphism.

    If there exists an isomorphism from $A$ to $B$, then we say $A$ and $B$ are
    \emph{isomorphic}.
  \end{defn}
  \begin{prop}
    The property of being isomorphic is reflexive, symmetric and transitive.
  \end{prop}
  \begin{rem}
    We don't say that isomorphism is an equivalence relation, since it would
    imply there exists a set of all well-ordered sets of this type.

    Such a set does not exist because if it did it would contain
    $(S,\mathrm{Id}_S)$ for each set $S$. This would imply the existence of a
    set of all sets.
  \end{rem}
  \begin{thm}
    There exists a unique ordered ring $\Zz$ (up to isomorphism) such that
    $\{x\in\Zz:x\ge 0\}$ is well-ordered.

    $\Zz$ is commutative.
  \end{thm}
  \begin{defn}
    We call this set $\Zz$ the \emph{integers}. The \emph{non-negative
    integers} $\Zz_{\ge 0}$ are $\{n\in\Zz: n\ge 0\}$. The \emph{positive
    integers} $\Zz^+$ are $\Zz_{\ge 0}\setminus \{0\}$.
  \end{defn}
  \begin{rem}
    As a byproduct of our construction, we get a canonical bijection between
    $\omega$ and $\Zz_{\ge 0}$.
    In particular, the cardinality of a finite set is a nonnegative integer.
  \end{rem}
  \begin{rem}
    We avoid use of the term \emph{natural numbers}, and the symbol $\mathbb N$,
    since
    some use them to mean the positive integers and others use them to mean the
    nonnegative integers.
  \end{rem}
  \begin{prop}
    Every ordered ring contains a unique subring isomorphic to $\Zz$.
  \end{prop}
  \begin{defn}
    In $\Zz\times\Zz^+$, we define the operations
    \[(a,b)+(c,d)=(ad+bc,bd),\qquad (a,b)(c,d)=(ac,bd).\]
    We also define an equivalence relation $\sim$ where
    $(a,b)\sim (c,d)\iff ad=bc$.

    We define the \emph{rational numbers} $\Qq$ as the partition
    of $\Zz\times\Zz^+$ induced by this equivalence relation, with
    $[(a,b)]+[(c,d)]=[(ad+bc,ac+bd)]$ and $[(a,b)]\cdot [(c,d)]=[(ac,bd)]$.
  \end{defn}
  \begin{prop}
    The relation $\sim$ is an equivalence relation. Moreover, the operations $+$
    and $\cdot$ are independent of the representatives of each equivalence
    class. With these operations, $\mathbb Q$ is a field.
  \end{prop}
  \begin{prop}
    Every ordered field contains a unique subfield isomorphic to $\Qq$.
  \end{prop}
  \begin{defn}
    A partially ordered set $S$ is \emph{complete} if every nonempty subset that has
    an upper bound in $S$ has a least upper bound in $S$.
  \end{defn}
  \begin{prop}
    Let $S$ be a complete partially ordered set. Every nonempty subset that
    has a lower bound in $S$ has a greatest lower bound in $S$.
  \end{prop}
  \begin{thm}
    There exists a unique complete ordered field, up to isomorphism.
  \end{thm}
  \begin{defn}
    We call this field $\Rr$.
  \end{defn}
  \begin{defn}
    We define $\Qq_{\ge 0},\ \Qq^+,\ \Rr_{\ge 0},\ \Rr^+$ in an analogous way to
    $\Zz_{\ge 0}$ and $\Zz^+$.
  \end{defn}
  \begin{defn}
    We define the \emph{complex numbers} $\Cc$ as $\Rr^2$, with the operations
    \[(a,b)+(c,d)=(a+c,b+d),\qquad (a,b)\cdot(c,d)=(ac-bd,ad+bc).\]

    We usually write $(a,b)$ as $a+bi$. We define the \emph{conjugate} of $a+bi$
    to be $\overline{a+bi}=a-bi$.
  \end{defn}
  \begin{prop}
    $\Cc$ is a field under these operations.
  \end{prop}
  \begin{prop}
    There are unique homomorphisms $\Zz\to\Qq$, $\Qq\to\Rr$ and $\Qq\to\Cc$.
    There is also an isomorphism $\Rr\to\{x\in\Cc: x=\overline x\}$.
  \end{prop}
  \begin{rem}
    Because of this, we usually take $\Zz\subseteq\Qq\subseteq\Rr\subseteq\Cc$.
  \end{rem}
  \begin{prop}
    Let $a\in\Cc$. Then, $a\overline{a}\in\mathbb R_{\ge 0}$.
  \end{prop}
  \begin{prop}
    Let $b\in\Rr_{\ge 0}$. There exists a unique $x\in\Rr_{\ge 0}$ such that
    $x\cdot x=b$. 
  \end{prop}
  \begin{defn}
    We call $x$ the \emph{square root} of $b$, denoted $\sqrt b$.

    We call $\sqrt{a\overline a}$ the \emph{modulus} of $a$, denoted $|a|$.
  \end{defn}
  \begin{prop}
    $|a+b|\le|a|+|b|$.
  \end{prop}
  \begin{thm}
    $|\Zz^+|=|\Zz_{\ge 0}|=|\Zz|=|\Qq|=\omega$, but $|\Rr|=|\Cc|=|\mathcal
    P(\omega)|$.
  \end{thm}
  \begin{defn}
    A \emph{polynomial} over $S$ is an expression of the form
    \[p(z)=a_0+a_1z+a_2 z^2+\cdots+a_m z^m,\]
    for some integer $m$ and coefficients $a_i\in S$.

    We say the \emph{degree} of $p$ is $d$, where $d$ is the largest integer
    such that $a_d\ne 0$. If no such $d$ exists, the degree is $-\infty$.
  \end{defn}
  \begin{prop}[Division Algorithm]
    Suppose $p$ and $s$ are polynomials over a field $\Ff$
    with $s\ne 0$. There exist unique
    polynomials $q,r$ over $\Ff$ such that $p=sq+r$ and $\deg r<\deg s$.
  \end{prop}
  \begin{defn}
    A number $r\in\Ff$ is a root of a polynomial $p$ over $\Ff$ if $p(r)=0$.
  \end{defn}
  \begin{prop}
    A polynomial over a field $\Ff$ has at most as many roots as its degree.
  \end{prop}
  \begin{thm}[Fundamental Theorem of Algebra]
    Every nonconstant polynomial over $\Cc$ has a root.
  \end{thm}
  \begin{prop}
    If $p$ is a polynomial over $\Cc$ then it has a unique factorisation of the
    form $p(z)=c(z-r_1)\cdots(z-r_m)$, where all constants are complex numbers.
  \end{prop}
  \begin{prop}
    If $p$ is a polynomial over $\Rr$ then it has a unique factorisation of the
    form
    \[p(x)=c(x-r_1)\cdots(x-r_m)(x^2+b_1x+c_1)\cdots(x^2+b_n x+c_n),\]
    where all constants are real numbers such that $b_j^2<4c_j$ for each $j$.
  \end{prop}

%chktex-file 3
\chapter{Vector Spaces}
\begin{defn}
  Let $\Ff$ be a field. A \emph{vector space over} $\Ff$ is an abelian group
  $V$ (of \emph{vectors})
  together with a function $\cdot:\Ff\times V\to V$ (\emph{scalar
    multiplication}) such that
  \begin{itemize}
    \item $a(b\mathbf v)=(ab)\mathbf v$ (compatible),
    \item $1\mathbf v=\mathbf v$ (identity), and
    \item $a(\mathbf u+\mathbf v)=a\mathbf u+a\mathbf v$ and $(a+b)\mathbf
      v=a\mathbf v+b\mathbf v$ (distributive).
  \end{itemize}
\end{defn}
\begin{defn}
  Let $S$ be a subset of $V$. A \emph{linear combination} of elements of $S$ is
  a vector of the form \[\sum_{i=1}^n a_i\mathbf s_i,\] where each $s_i$ is a
  distinct element of $S$.
\end{defn}
\begin{defn}
  A subset $S$ of $V$ is \emph{linearly independent} if any linear combination
  of elements of $S$ that produces $\mathbf 0$ has all coefficients equal to $0$.
  Otherwise, it is \emph{linearly dependent}.
\end{defn}
\begin{defn}
  A \emph{subspace} $W$ of $V$ is a nonempty subset of $V$ which is also a
  vector space over $\Ff$.
\end{defn}
\begin{prop}
  A subset $W$ of $V$ is a subspace iff the following conditions hold:
  \begin{itemize}
    \item $W$ is nonempty;
    \item $W$ is closed under addition; and
    \item $W$ is closed under scalar multiplication.
  \end{itemize}
\end{prop}
\begin{defn}
  The \emph{span} of a subset $S$ of $V$ is the intersection of all linear
  subspaces of $V$ that contain $S$.  
\end{defn}
\begin{prop}
  The span of $S$ is the set of linear combinations of elements of $S$.
\end{prop}
\begin{defn}
  A subset $S$ of $V$ is a \emph{basis} if it is linearly independent and its
  span is $V$.
\end{defn}
\begin{thm}
  Assume $V$ has a finite basis. Then, any two bases of $V$ have the same
  cardinality.
\end{thm}
\begin{thm}
  If we take the Axiom of Choice, then
  \begin{itemize}
    \item Every vector space $V$ has a basis.
    \item Any two bases of $V$ have the same cardinality.
  \end{itemize}
\end{thm}
\begin{defn}
  The \emph{dimension} of $V$ is the cardinality of a basis of $V$.
\end{defn}
\begin{defn}
  An \emph{inner product space} is a vector space $V$ over a field $\Ff$ which
  is either $\Rr$ or $\Cc$, together with a function
  $\langle\cdot,\cdot\rangle:V\times V\to\Ff$ satisfying
  \begin{itemize}
    \item $\langle \mathbf x,\mathbf y\rangle=\overline{\langle \mathbf
      y,\mathbf x\rangle}$ (conjugate
      symmetry)
    \item $\langle a\mathbf x+b\mathbf y,\mathbf z\rangle=a\langle\mathbf
      x,\mathbf z\rangle+b\langle\mathbf y,\mathbf z\rangle$ (linearity in the
      first argument), and
    \item $\langle\mathbf x,\mathbf x\rangle=0\implies\mathbf x=\mathbf 0$.
  \end{itemize}
\end{defn}
\begin{defn}
  Let $S$ be a set, and let $n\in\Zz^+$. An \emph{ordered $n$-tuple} 
  $\{a_i\}_1^n$
  of elements of $S$ is a function $f:\{x\in\Zz^+:x\le n\}\to S$. We write
  $a_i=f(i)$ for each positive integer $i$ which is at most $n$.
\end{defn}
\begin{defn}
  For any $n\in\Zz^+$, we define $\Rr^n$ and $\Cc^n$ to be the set of ordered
  $n$-tuples of real and complex numbers respectively, with addition and
  scalar multiplication defined componentwise.
\end{defn}
\begin{prop}
  $\Cc^n$ is an inner product space over $\Cc$ with the inner product
  \[\langle a,b\rangle=\sum_{i=1}^n a_i\overline{b_i}.\] 

  $\Rr^n$ is an inner product space over $\Rr$ with the inner product
  \[\langle a,b\rangle=\sum_{i=1}^n a_i b_i.\]
\end{prop}
\begin{defn}
  A \emph{normed vector space} is a vector space $V$ over $\Rr$ or $\Cc$
  on which there is a \emph{norm}: a function $\|\cdot\|:V\to\Cc$ satisfying
  \begin{itemize}
    \item $\|\mathbf x\|\ge 0$,
    \item $\|\mathbf x\|=0$ implies $\mathbf x=\mathbf 0$,
    \item $\|a\mathbf x\|=|a|\|\mathbf x\|$, and
    \item $\|\mathbf x+\mathbf y\|\le\|\mathbf x\|+\|\mathbf y\|$ (the triangle
      inequality).
  \end{itemize}
\end{defn}
\begin{prop}
  If $V$ is an inner product space, then $\langle\mathbf x,\mathbf x\rangle$ is
  real for all $\mathbf x$.
  Moreover, $\|\mathbf x\|=\sqrt{\langle\mathbf x,\mathbf x\rangle}$ is a norm
  on $V$.
\end{prop}

\chapter{Metric Spaces}
\begin{defn}
  A \emph{metric space} is a nonempty set $M$ together with a function
  $d:M\times M\to\Rr$ (the \emph{metric}) such that
  \begin{itemize}
    \item $d(x,y)=0\iff x=y$,
    \item $d(x,y)=d(y,x)$ (symmetry), 
    \item $d(x,z)\le d(x,y)+d(y,z)$ (triangle inequality).
  \end{itemize}
\end{defn}
\begin{prop}
  In a normed vector space, the function $d(x,y)=\|x-y\|$ is a metric.
\end{prop}
\begin{defn}
  We call this the \emph{induced metric}.
\end{defn}
\begin{defn}
  In a metric space, the \emph{open ball} $B_r(x)$ with centre $x$ and radius $r$ is the
  set of all points $y$ with $d(x,y)<r$.

  The \emph{closed ball} $\overline{B_r(x)}$ with centre $x$ and radius $r$ is
  the set of all points $y$ with $d(x,y)\le r$.
\end{defn}
\begin{defn}
  Let $E$ be a subset of a metric space $M$.
  \begin{itemize}
    \item A point $p$ is a \emph{limit point} of $E$ if every open ball centred
      at $p$ contains a point $q\ne p$ such that $q\in E$.
    \item A point $p$ is an \emph{interior point} of $E$ if there is an open
      ball centred at $p$ which is a subset of $E$.
    \item $E$ is \emph{closed} if every limit point of $E$ is a point of $E$.
    \item $E$ is \emph{open} if every point of $E$ is an interior point of $E$.
    \item $E$ is \emph{bounded} if it is contained in some open ball.
    \item The \emph{complement} $E^c$ of a set $E$ is the set $M\setminus E$.
    \item The \emph{interior} of $E$ is the set of interior points of $E$.
    \item The \emph{boundary} $\partial E$ of $E$ is the set of points of $M$
      that are limit points of both $E$ and $E^c$.
  \end{itemize}
\end{defn}
\begin{prop}
  The interior and boundary of $E$ are disjoint, and their union is $E$.
\end{prop}
\begin{prop}
  The following are equivalent:
  \begin{itemize}
    \item $E$ is open.
    \item $E\cap\partial E=\emptyset$.
    \item $E^c$ is closed.
    \item $\partial E\subseteq E^c$.
  \end{itemize}
\end{prop}
\begin{prop}
  Every open ball is open; every closed ball is closed.
\end{prop}
\begin{prop}
  If $p$ is a limit point of $E$, then every open ball centred around $p$
  contains infinitely many points of $E$.
\end{prop}
\begin{prop}
  Any union of open sets is open; a finite intersection of open sets is open.

  Any intersection of closed sets is closed; a finite union of closed sets is
  closed.
\end{prop}
\begin{defn}
  The \emph{closure} of $E$ is the set $E\cup\partial E$.
\end{defn}
\begin{prop}
  The closure of $E$ is closed; the interior of $E$ is open.

  Any closed set which contains $E$ contains the closure of $E$. Any open set
  which is contained in $E$ is contained in the interior of $E$.
\end{prop}
\begin{prop}
  Suppose $X\subseteq M$ inherits the metric. A subset $E$ of $X$ is open
  relative to $X$ iff $E=X\cap Y$ for some open set $Y$.
\end{prop}
\begin{defn}
  An \emph{open cover} of $E$ is a set of open sets whose union contains $E$.
\end{defn}
\begin{prop}
  The following are equivalent:
  \begin{itemize}
    \item 
      Every open cover of $E$ contains a finite
      subset which is still an open cover of $E$.
    \item Every infinite subset of $E$ contains a limit point in $E$.
  \end{itemize}
\end{prop}
\begin{defn}
  Such a set is called \emph{compact}.
\end{defn}
\begin{prop}
  Suppose $X\subseteq M$ inherits the metric. A subset $E$ of $X$ is open
  relative to $X$ iff $E$ is compact relative to $M$.
\end{prop}
\begin{prop}
  A compact subset of a metric space is closed and bounded; a closed subset of a compact
  metric space is compact.
\end{prop}
\begin{prop}
  If $S$ is a collection of compact subsets of a metric space such that any
  finite intersection of elements of $S$ is nonempty, then $\bigcap S$ is
  nonempty.
\end{prop}
\begin{thm}[Heine-Borel]
  A subset of $\Rr^n$ is compact iff it is closed and bounded.
\end{thm}
\begin{thm}[Weierstrass]
  Every bounded infinite subset of $\Rr^n$ has a limit point.
\end{thm}
\begin{defn}
  Two subsets $A$ and $B$ of a metric space $X$ are \emph{separated} if both
  $A\cap\overline B$ and $B\cap\overline A$.

  A set $E$ is \emph{disconnected} if it is the union of two nonempty separated
  sets, and connected otherwise.
\end{defn}
\begin{prop}
  A metric space $M$ is connected iff the only sets which are both open and closed
  are the empty set and $M$.
\end{prop}
\begin{prop}
  A subset of $\Rr^1$ is connected iff it is an interval.
\end{prop}
\begin{defn}
  A sequence $\{a_n\}$ is \emph{convergent} if there is a point $L$ such that
  for any $\varepsilon>0$ there is an $N\in\Zz^+$ such that $n\ge N$ implies
  $d(a_n,L)<\varepsilon$. We write \[\lim_{n\to\infty} a_n=L.\]
\end{defn}
\begin{prop}
  Suppose $\{a_n\}$ and $\{b_n\}$ are sequences of complex numbers which
  converge to $a$ and $b$ respectively. Then the sequences $\{a_n+b_n\},\
  \ \{a_n b_n\},\ \{\frac{a_n}{b_n}\}$ converge to $a+b,\ ab,\ \frac ab$
  respectively (where in the last one we require $b_n\ne 0$ for each $n$).
\end{prop}
\begin{prop}
  A sequence in $\Rr^n$ or $\Cc^n$ converges iff it converges coordinatewise.
\end{prop}
\begin{defn}
  A sequence $\{p_n\}$ is \emph{Cauchy} if for every $\varepsilon>0$ there is an
  integer $N$ such that $d(p_n,p_m)<\varepsilon$ if $m,n\ge N$.

  A metric space is \emph{complete} if every Cauchy sequence converges.
\end{defn}
\begin{prop}
  Every convergent sequence is Cauchy.
\end{prop}
\begin{prop}
  Every compact metric space is complete.
\end{prop}
\begin{prop}
  $\Rr^n$ and $\Cc^n$ are complete.
\end{prop}
\begin{defn}
  Let $f:X\to Y$ be a function, where $Y$ is a metric space and $X$ is a 
  subset of a metric space $E$. Let $p$ be a limit point of $X$. We say that
  \[\lim_{x\to p}f(x)=q\] if for every sequence $\{x_n\}$ in $E$ which converges
  to $p$ but does not contain $p$, $f(x_n)$ converges to $q$.
\end{defn}
\begin{defn}
  We say that $f$ is \emph{continuous} at $p$
  if for every sequence $\{x_n\}$ in $E$ which converges
  to $p$, $f(x_n)$ converges to $f(p)$.

  We say that $f$ is \emph{continuous} on $X$, or simply \emph{continuous},
  if it is continuous at every point in $X$.
\end{defn}
\begin{prop}
  A function $f$ is continuous iff the inverse image of every open set is open.
\end{prop}
\begin{prop}
  If $f$ is continuous, then
  \begin{itemize}
    \item The image of a compact set is compact.
    \item The image of a connected set is connected.
  \end{itemize}
\end{prop}
\begin{cor}[Intermediate Value Theorem]
  If the codomain of $f$ is $\Rr$, then it is an interval. If the domain of $f$
  is a compact set, then the interval is closed.
\end{cor}
\begin{defn}
  A function $f$ is \emph{uniformly continuous} if for every $\varepsilon>0$
  there exists a $\delta>0$ such that if $d(a,b)<\delta$ then
  $d(f(a),f(b))<\varepsilon$.
\end{defn}
\begin{thm}
  Every continuous function on a compact set is uniformly continuous.
\end{thm}
\begin{thm}
  Let $S$ be an open subset of $\Rr^n$, and let $f:S\to\Rr^n$ be an injective
  continuous function. Then the image of $f$ is open.
\end{thm}
\begin{thm}[Fundamental Theorem of Algebra]
  Let $p:\Cc\to\Cc$ be a polynomial. Then the image of $p$ is $\Cc$.
\end{thm}
\subsection*{References}
\begin{itemize}
  \item \emph{Principles of Mathematical Analysis}, Rudin
\end{itemize}

\appendix
\chapter{Proofs}

\end{document}
